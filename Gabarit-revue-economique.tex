% !TeX root = Gabarit-revue-economique.tex
\documentclass{Revue-economique} % Classe de document pour la revue économique


\begin{document} % Contenu du document

\begin{Article}[%
    Titre=À la recherche du point de basculement sur le marché du logement : les résultats d’une expérience de terrain,
    Auteur={Sylvain Chareyron\thanks{Université Paris-Est Créteil, UGE, ERUDITE (EA 437), TEPP-CNRS (FR 2042). Correspondance : 61 avenue du général de Gaulle, 94010 Créteil, France. Courriel : sylvain.chareyron@u-pec.fr}, \par
    Samuel Gorohouna\thanks{Université de la Nouvelle-Calédonie, LARJE (EA 3329). Correspondance : BP R4, 98851 Nouméa, Nouvelle-Calédonie. Courriels : samuel.gorohouna@unc.nc}, \par
    Yannick L’Horty\thanks{Université Gustave Eiffel, UPEC, ERUDITE (EA 437), TEPP-CNRS (FR 2042). Correspondance : 5 boulevard Descartes, 77454 Marne-la-Vallée, France. Courriels : yannick.lhorty@univ-eiffel.fr}, \par
    Pascale Petit\thanks{Université Gustave Eiffel, UPEC, ERUDITE (EA 437), TEPP-CNRS (FR 2042). Correspondance : 5 boulevard Descartes, 77454 Marne-la-Vallée, France. Courriels : pascale.petit@univ-eiffel.fr}, \par
    Catherine Ris\thanks{Université de la Nouvelle-Calédonie, LARJE (EA 3329). Correspondance : BP R4, 98851 Nouméa, Nouvelle-Calédonie. Courriels : catherine.ris@unc.nc\\
    Cette recherche a bénéficié du soutien de l’Agence nationale de la recherche (projet DALTON ANR-15-CE28-0004) et de l’Université de Nouvelle-Calédonie. Nous sommes redevables à Tuamanaia Foimapafisi pour son assistance de recherche très utile dans la phase de collecte des données. Nous tenons à remercier les participants aux conférences JMA (Casablanca, 2019), EALE (Uppsala, 2019), LAGV (Aix-en-Provence, 2019) et AFSE (Orléans, 2019). Nous remercions le rédacteur en chef et les rapporteurs anonymes pour leurs précieux commentaires qui ont permis d’améliorer cet article. Toutes les erreurs sont de notre responsabilité.
}}
]

\begin{resume}
    Cet article étudie économétriquement la relation entre la discrimination liée à l’origine éthnique et la composition ethnique des quartiers. En particulier, nous recherchons la présence d’une non-linéarité dans la relation, qui pourrait être indicative d’un point de basculement dans la composition ethnique. Pour ce faire, nous mesurons la discrimination dans l’accès au logement en Nouvelle-Calédonie avec une vaste expérience de terrain. Entre 2015 et 2017, nous avons envoyé six demandes de location à 741 annonces immobilières, soit un total de 3 616 observations. Les différents candidats ont des noms européens, kanaks et wallisiens et montrent différents signaux de qualité. En utilisant les données du recensement, nous apparions la localisation de l’offre à la composition ethnique du quartier. Les résultats montrent une discrimination ethnique plus faible dans les quartiers où la part des minorités est supérieure à 40 \%. La discrimination reste cependant importante dans les quartiers composés principalement de minorités ethniques.
\end{resume}

\titrearticleENG{Seeking for tipping point in the housing market:
Evidence from a field experiment}

\begin{resumeENG}
    This study conducts an econometric investigation of the relationship between ethnic discrimination and the ethnic composition of neighborhoods. In particular, we look for the presence of non-linearity in the relationship that might be indicative of a tipping point in ethnic composition. To do this, we measure discrimination in access to housing in New Caledonia with a large field experiment. Between 2015 and 2017, we sent six rental applications to 741 real-estate advertisements, for a total of 3,616 observations. The various applicants have European, Kanak and Wallisian names and show different signals of quality. Using census data, we link the location of the offer to neighborhood ethnic composition. The results show lower ethnic discrimination in neighborhoods with minority share above 40\%. Discrimination remains, however, substantial in neighborhoods that are composed mainly of ethnic minorities.    
\end{resumeENG}

\motscles{discrimination, marché locatif, point de basculement, ségrégation, test de correspondance}
\keywords{discrimination, rental market, tipping-point, segregation, correspondence test}
\jelcode{J15, R23, C93}

\section{Introduction}

Si le nombre d’études mesurant la discrimination dans l’accès au logement est en augmentation, l’intérêt pour les déterminants de la discrimination est encore limité, notamment pour le rôle joué par la composition ethnique des habitants du quartier. Cette dimension pourrait avoir un effet important sur la dynamique de la ségrégation si la discrimination est plus prononcée lorsque des individus appartenant à une minorité ethnique emménagent dans un quartier composé principalement d’individus de l’ethnie dominante.

Cependant, la présence d’une relation ainsi que sa forme ne sont pas évidentes. On s’attend à ce que la discrimination statistique ne varie pas en fonction de la composition de la zone, car elle peut être exercée par des propriétaires appartenant aussi bien au groupe majoritaire qu’aux minorités. Toutefois, la notion de préjudice du client permet d’établir un lien entre la composition ethnique et la discrimination : dans ce cas, les propriétaires refuseront de louer à des minorités parce qu’ils estiment que leurs clients du groupe majoritaire ont des préjugés à l’égard de la diversité ethnique (Page [1995]). On peut alors s’attendre à ce que la discrimination diminue linéairement avec l’augmentation de la proportion d’habitants issus des minorités ethniques dans le quartier : plus cette proportion est élevée, plus le niveau de discrimination est faible. Toutefois, la relation pourrait être non linéaire si les individus de la majorité ethnique, tout en préférant l’homogénéité ethnique, ont une tolérance pour la diversité jusqu’à un certain seuil. Les propriétaires chercheront à maintenir l’homogénéité ethnique en dessous de ce seuil afin d’éviter de provoquer le départ des résidents du groupe ethnique principal et discrimineront davantage lorsque la composition du quartier se rapprochera de ce seuil. Des preuves de ce type de préférences individuelles existent mais, pour l’instant, elles ont été observées exclusivement aux États-Unis. Card, Mas et Rothstein [2008] ont montré que, dans ce pays, les populations blanches ont tendance à quitter les villes lorsque la part des minorités dans la population atteint entre 5 et 20~\%. Ils appellent ce seuil, « niveau de tolérance », en accord avec la notion de « point de basculement » de Schelling [1971]. Pour Schelling [1971], un « basculement » se produit lorsqu’une nouvelle minorité visible est suffisamment présente dans un quartier pour que les premiers résidents commencent à le quitter. Le point de basculement est le niveau à partir duquel les déménagements des résidents du groupe ethnique majoritaire se produisent.

L’objectif de notre étude est d’explorer la relation entre la discrimination ethnique et la composition ethnique des quartiers\footnote{Les données de recensement dont disposaient Bunel et al. [2019] ne permettaient que de subdiviser le Grand Nouméa en trois zones différentes.}. Cela nous permettra de comprendre les mécanismes en jeu et de mieux expliquer la discrimination. Cela permettra également de mieux comprendre la relation entre la discrimination et la ségrégation. En particulier, l’identification d’une non-linéarité dans la discrimination ethnique peut être le signe d’un point de basculement dans la composition ethnique et, par conséquent, d’une préférence des individus de la majorité ethnique pour des quartiers homogènes.

Nous considérons un modèle simple avec deux groupes : un groupe majoritaire (c’est-à-dire les Européens) et un groupe minoritaire (c’est-à-dire tous les groupes ethniques qui ne sont pas européens). Toutes les minorités ethniques sont regroupées car, en supposant que les Européens prennent leur décision de localisation en fonction de l’homogénéité ethnique du quartier, nous nous attendons à ce que chaque personne supplémentaire appartenant à la minorité ethnique augmente la diversité. En cas de point de basculement, nous nous attendons donc à ce que : 1) lorsque la part de la minorité est inférieure à un certain seuil (le point de basculement), et que la part de la minorité augmente, la discrimination augmente, 2) lorsque la part de la minorité est supérieure au point de basculement, la discrimination diminue.

Nous utilisons un ensemble de données de 3 616 observations issues d’une vaste expérience de terrain pour déterminer la relation entre la discrimination ethnique sur le marché du logement locatif et la composition ethnique des quartiers en Nouvelle-Calédonie. Les données de recensement de l’Institut de la statistique et des études économiques de Nouvelle-Calédonie (ISEE)\footnote{Nous utilisons ici les données du recensement de 2014.} sont utilisées pour associer à chaque offre de logement une information sur la composition ethnique du quartier dans lequel le logement est situé. Cela permet d’étudier économétriquement la relation entre la différence de réponses positives reçues par les différents profils qui demandent à visiter un logement et la composition du quartier dans lequel se trouve le logement.

À notre connaissance, cet article est le premier à entreprendre une étude économétrique de la non-linéarité de la relation entre discrimination et ségrégation ethnique sur le marché du logement locatif en dehors des États-Unis. L’étude européenne la plus proche est celle de Bunel et al. [2019], dont nous réutilisons en partie les données, et qui mesure la discrimination ethnique à Nouméa. Par rapport à cet article, notre étude présente trois améliorations majeures. Premièrement, l’échantillon est presque deux fois plus grand en raison d’une deuxième vague de tests. Deuxièmement, nous testons la discrimination envers un groupe ethnique supplémentaire, celui des Wallisiens. La situation en Nouvelle-Calédonie est remarquable dans le sens où l’un des groupes potentiellement exposés à la discrimination (à savoir les Kanaks, que nous considérons pour cette raison comme une minorité dans ce qui suit), n’est pas, numériquement parlant, un groupe minoritaire\footnote{Les Kanaks sont numériquement le groupe ethnique le plus important en Nouvelle-Calédonie mais seulement le deuxième dans le Grand Nouméa, où les Européens sont majoritaires.}. Bien qu’ils représentent une proportion non négligeable de la population, les Wallisiens sont minoritaires en Nouvelle-Calédonie et se distinguent des Kanaks en ce qu’ils ne sont pas autochtones de l’île de Nouvelle-Calédonie. Par conséquent, il est intéressant d’évaluer si les niveaux de discriminations observés pour les Kanaks sont similaires à ceux des Wallisiens. Troisièmement, nous fusionnons notre jeu de données expérimentales avec les données du recensement au niveau de l’IRIS afin d’obtenir des informations précises sur la composition ethnique des quartiers. Toutes ces améliorations nous permettent d’approfondir le lien entre discrimination et ségrégation ethnique et de tester l’hypothèse du point de basculement.

Le champ géographique de l’étude est le Grand Nouméa, capitale du territoire français d’outre-mer de la Nouvelle-Calédonie. Des candidats de trois origines différentes sont testés : kanake, wallisienne et européenne. En 2014, ces groupes ethniques représentent respectivement 23 \%, 12 \% et 34 \% de la population de l’agglomération\footnote{Le reste de la population de l’agglomération du Grand Nouméa est d’origine mixte (10 \%), d’autres groupes ethniques (17 \%) et non déclarée (3 \%).}. La Nouvelle-Calédonie, et plus particulièrement l’agglomération de Nouméa, présente plusieurs caractéristiques particulièrement intéressantes pour cette étude. Tout d’abord, c’est un territoire du Pacifique avec une histoire coloniale européenne et c’est l’un des rares territoires coloniaux où la population autochtone, les Kanaks, est de taille comparable à la population européenne. La discrimination dans l’accès au logement a été peu étudiée dans ce type de contexte colonial. Ensuite, la Nouvelle-Calédonie est le seul territoire de la République française, avec la Polynésie française, où l’on dispose de statistiques ethniques permettant de déterminer la répartition locale des ethnies. Enfin, l’agglomération de Nouméa, lieu du test, est un territoire où la concentration de la population potentiellement discriminée varie d’une petite minorité à une grande majorité selon les localités. Nous disposons donc d’une grande variété spatiale pour notre variable d’intérêt : la composition ethnique locale. Le niveau élevé de ségrégation dans l’agglomération de Nouméa la rend idéale pour étudier le lien entre l’environnement ethnique et la discrimination. Dans la partie sud de l’agglomération, la population kanake ne représente que 5 \%, alors qu’elle est de 50 \% dans la partie nord située à seulement 10 kilomètres. Deux signaux de qualité des candidats sont également introduits afin de distinguer les deux principaux types de discrimination : la discrimination statistique et la discrimination par les préférences.

Les résultats indiquent une forte discrimination à l’égard du demandeur kanak et encore plus à l’égard du demandeur wallisien. Nous montrons qu’un changement dans la discrimination se produit lorsque la part des minorités dépasse 40 \% : la discrimination contre le demandeur kanak est la plus faible de 7 points de pourcentage dans les quartiers où la part des minorités est supérieure à 40 \%. Cela suggère la présence d’un point de basculement dans la composition ethnique des quartiers qui peut conduire les propriétaires à discriminer davantage dans les quartiers en dessous de ce seuil. Nous observons également qu’il existe toujours un niveau substantiel de discrimination dans les quartiers où la part des minorités est élevée, en particulier lorsqu’il n’y a pas de signal de qualité, ce qui indique l’existence d’une discrimination statistique importante.

La section suivante présente nos hypothèses de recherche et discute des recherches précédentes sur la relation entre la discrimination ethnique sur le marché du logement et la composition des quartiers. La troisième section décrit le contexte de l’étude et la quatrième section présente le protocole expérimental et la procédure de collecte des données. La cinquième section présente les résultats de l’expérience et la sixième section discute de la robustesse des résultats. Les conclusions sont présentées dans la dernière section du document.

\section{HYPOTHÈSES}

La méthode du test par correspondance appliquée dans divers pays a permis de conclure à la présence d’une discrimination ethnique sur le marché du logement. Dans un survol récent, Flage [2018] recense 29 études scientifiques qui ont utilisé cette méthode dans 15 pays différents. Il conclut que les candidats qui suggèrent leur origine ethnique par un nom de famille à consonance étrangère ont en moyenne deux fois moins de chances que les candidats issus de l’ethnie principale du pays d’être invités à visiter un logement locatif. La plupart de ces travaux s’attachent à prouver l’existence d’une discrimination et à en mesurer l’intensité.

Pour les économistes, l’identification d’une discrimination signale une anomalie dans le fonctionnement du marché du logement. L’existence et l’ampleur d’une telle anomalie est un sujet intéressant en soi. Mais il est clair qu’il faut aller plus loin et comprendre les causes de ce type d’anomalie si l’on veut être en mesure de proposer des actions pour la combattre efficacement. Or il existe relativement peu de données sur les déterminants de la discrimination.

Nous nous concentrons ici sur un déterminant clé : la composition ethnique de la localité. La discrimination peut être reliée de plusieurs façons avec la ségrégation locale. D’une part, il est clair qu’une forte discrimination fondée sur l’origine ethnique peut renforcer et amplifier les processus qui créent et maintiennent la ségrégation ethnique locale. D’autre part, la ségrégation ethnique locale peut à son tour influencer l’intensité de la discrimination. Lorsque le groupe minoritaire, potentiellement discriminé, devient dominant dans une localité, cela favorise-t-il ou limite-t-il la discrimination ? Il nous semble que la réponse à cette question n’est pas évidente. Cependant, c’est une question importante si nous voulons savoir si les comportements discriminatoires dans l’accès au logement contribuent à amplifier ou atténuer la ségrégation spatiale.

Nous formulons trois hypothèses qui relient la part de la minorité visible dans un quartier à l’intensité de la discrimination que les personnes issues de la minorité peuvent subir dans l’accès au logement. Tout d’abord, l’intensité de la discrimination (D1) dépend de nombreux facteurs qui ne sont pas nécessairement liés à la composition ethnique du quartier de résidence. En particulier, on s’attend à ce que la discrimination statistique soit exercée par les propriétaires issus des différents groupes ethniques et, par conséquent, qu’elle existe indépendamment de la composition du quartier. La discrimination statistique découle du fait que les propriétaires agissent sur la base de la croyance que les individus appartenant à des groupes différents possèdent des caractéristiques pertinentes pour le marché qui sont différentes les unes des autres (Page [1995]). Par conséquent, même lorsque le groupe minoritaire est prédominant dans une zone, les personnes appartenant à la minorité ethnique peuvent être discriminées en raison de la discrimination statistique.

\end{Article}

\end{document}