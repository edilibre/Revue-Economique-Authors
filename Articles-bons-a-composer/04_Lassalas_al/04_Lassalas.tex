\selectlanguage{english}

\begin{Article}[Auteur={
    Marie Lassalas\thanks{SMART, Institut Agro, INRAE. \textit{Correspondence:} 65 rue de Saint-Brieuc, 35000 Rennes, France. \textit{Email:} marie.lassalas@agrocampus-ouest.fr}\\
    Alejandro  Plastina\thanks{Department of Economics, Iowa State University. \textit{Correspondence:} 260 Heady Hall, 518 Farm House Lane, 50011-1054 Ames, Iowa, USA. \textit{Email:} plastina@iastate.edu}\\
    Sergio H. Lence\thanks{Department of Economics, Iowa State University. \textit{Correspondence:} 260 Heady Hall, 518 Farm House Lane, 50011-1054 Ames, Iowa, USA. \textit{Email:} shlence@iastate.edu}}, 
    Titre={Does the Adoption of Environmental Contracts Affect Farms' Productivity and Efficiency?\\
    \soustitre{An Insight through the Adoption of Agri-Environmental Schemes (AES) and Organic Certification}}]

\label{Lassalas}

\begin{refsection}[Lassalas]

\remerciements{La recherche ayant donné lieu à cet article a
été partiellement financée par la Fondation de
L'Institut Agro Rennes-Angers, Fulbright et Rennes
Métropole.

Acknowledgments: The research leading to this article was
partially funded by Fondation de L'Institut Agro Rennes-Angers,
Fulbright, and Rennes Métropole.}

% On redéfinit localement le titre (courant) des titres d'articles trop longs
\renewcommand{\titrecourantarticle}{Does the Adoption of Environmental Contracts Affect Farms'
Productivity…?}

\begin{resumeENG}
The Common Agricultural Policy (CAP) must meet increasingly ambitious
environmental objectives while ensuring food supply and farmers' income.
We assess the impact of two instruments implemented in the CAP,
agri-environmental schemes (AES) and organic subsidies, on farms'
productivity and efficiency. We apply a stochastic frontier model and a
difference-in-differences estimator to data on French farms specialized
in crop production. Our results suggest that the adoption of AES and
organic farming has no effect on farms' productivity and efficiency
regarding their own production frontier. When compared to the entire
sector, results show a negative causal effect of the adoption of organic
practices on farms' efficiency between 2014 and 2020. Organic farms
operate under a lower production frontier than conventional and AES
farms regarding the metafrontier.
\end{resumeENG}

\titrearticleENG{L'adoption de contrats environnementaux affecte-t-elle la productivité\\ et l'efficacité des exploitations agricoles ? \\
Un éclairage à travers l'adoption de mesures agro-environnementales \\ 
et climatiques (MAEC) et de la certification biologique}
  
\begin{resume}
La Politique agricole commune (PAC) doit concilier objectifs
environnementaux, approvisionnement alimentaire et revenu des
agriculteurs. Nous évaluons l'effet des MAEC et de l'agriculture biologique (AB) sur la productivité et l'efficacité des exploitations. Nous estimons un modèle
de frontière stochastique et appliquons un estimateur de différences-en-différences sur des données d'exploitations françaises spécialisées en grandes cultures. L'adoption de MAEC et de l'AB ne semble pas affecter la productivité et l'efficacité des exploitations vis-à-vis de leur propre frontière de production. Lorsque comparés à l'ensemble du secteur via une métafrontière,
les résultats montrent un effet négatif de l'adoption de l'AB sur l'efficacité des exploitations entre 2014 et 2020. Les exploitations AB opèrent sous une frontière de production inférieure aux autres exploitations.
\end{resume}


\keywords{agri-environmental schemes, organic certification,
production function, productivity analysis, stochastic frontier
analysis, technical efficiency}

\motscles{mesures agro-environnementales et climatiques,
certification agriculture biologique, fonction de production, analyse de
productivité, analyse de frontière stochastique, efficacité technique}

\jelcode{D24, Q12, Q15, Q18}

\section{Introduction}

The intensification of agricultural practices is a major driver of
climate change and biodiversity loss. Worldwide, the agricultural sector
accounted for 23\% of total net anthropogenic greenhouse gases (GHG)
from 2007 to 2016 (\textcite{ipcc_2019}). The report of the Intergovernmental
Science-Policy Platform on Biodiversity and Ecosystem Services (IPBES)
notes that about one million species are threatened with extinction
(\textcite{ipbes_2019}), and that agriculture and overexploitation of wild
species are the main drivers of biodiversity decline (\textcite{maxwell_fuller_brooks_w_2016}; \textcite{tilman_etal_2017}; \textcite{ipbes_2019}). The
intensification of agricultural practices harms biodiversity through
various channels: habitat destruction and homogenization by spreading
monoculture and removing hedgerows, use of chemical inputs (pesticides
and fertilizers) and tillage (\textcite{benton2003farmland}; \textcite{hautier_niklaus_hector_2009}; \textcite{geiger_et_al_2010}; \cite{beketov2013pesticides}; \textcite{arslan2018}; \textcite{sanchez-bayo_wyckhuys_2019}; \textcite{raven_wagner_2021}).

A reduction in pesticides use by 50\%, a decrease in fertilizer use by
20\% and a conversion of 25\% of the agricultural land under organic
farming are the ambitious objectives set for 2030 by the European Union
in terms of environment protection in the Green Deal and its
Farm-to-Fork Strategy (\textcite{european_commission_2020}). The Farm-to-Fork Strategy report recalls that it is urgent for agricultural production to
reduce dependency on pesticides, reduce excessive use of fertilizers and
reverse the decline in biodiversity. The report also highlights that the
transition has to be sustained by the European Common Agricultural
Policy (CAP). There exist already various agri-environmental policy
instruments developed by the European Union to reduce the negative
environmental impacts of food production (\textcite{deboe_2020}). European
agri-environmental policies are mainly based on conditioning CAP
subsidies to environmental standards, subsidizing organic farming, and
agri-environmental schemes (AES)---voluntary contracts between
governments and farmers to produce environmental services in return for
annual payments.

Recent social concerns about the negative effects of agricultural
practices have brought new objectives to the CAP. However, historical
and still relevant objectives for the CAP are to ensure food supply,
support farmers' income and increase the competitiveness of the
agricultural sector. Thus, questions can be raised as to which
instruments future CAP programs will use to meet environmental
challenges and what effects they will have on farms' productivity. In
this paper, we aim at assessing the impact of two existing instruments
on farms' productivity that will probably be used by future CAP programs
to incentivize farmers to adopt environmental practices, namely, AES and
organic subsidies. We will refer to both instruments as environmental
contracts. From a public policy point of view, the goal is to understand
whether government support toward the voluntary adoption of
environmental practices affects the efficiency and the productivity of
the participating farms. Our analysis focuses on French crop farms from
the Poitou-Charentes region during the 2014--2020 CAP program. We
estimate farms' aggregate productivity using a modified Malmquist index
approach, and focus on the technical efficiency component. We adopt a
stochastic frontier (SF) model. To assess the effect of the adoption of
AES and organic farming, we estimate the average treatment effect on the
treated (ATT) using a difference-in-differences approach.

AES and organic contracts have similar characteristics: their adoption
is voluntary and farmers receive monetary incentives to do so. They also
differ on several points. First, the requirements for organic farming
are more stringent than the ones for AES contracts. Second, while AES
rely on five-year contracts, the adoption of organic farming is based on
a two-year conversion period that commits farms to production practices
in the long term. Finally, the monetary incentive mechanisms differ. In
the case of AES contracts, farmers receive subsidies to compensate for
the profit loss due to the adoption of environmental practices. For
organic farming, there are two monetary incentives. First, organic
producers in France receive subsidies during the conversion period to
partially offset the temporary reduction in profitability, and after
conversion (at lower rates) to reduce market risks and support organic
practices. Second, organic products receive a price premium thanks to
product differentiation on the market.

A vast literature aims at evaluating the impact of subsidies on farms'
technical efficiency (see \textcite{minviel_latruffe_2016} for a
meta-analysis of the effects of public subsidies on farms' technical
efficiency). There is no consensus in the literature on the impact of
the adoption of organic farming on technical efficiency despite the
existence of numerous efficiency analyses (see \textcite{lakner_breustedt_2017} for a review). Some studies show that, on average, organic
certification increases technical efficiency (\textcites{tzouvelekas_pantzios_fotopoulos_2001, tzouvelekas_pantzios_fotopoulos_2002}; \textcite{oude_lansink_2002}; \textcite{tiedemann_latacz-lohmann_2013}; \textcite{grovermann_et_al_2021}), while others find a negative impact (\textcite{sipilainen_oude_lansink_2005}; \textcite{kumbhakar_tsionas_sipilainen_2009}; \textcite{serra_goodwin_2009}). There are also studies that do not find a
significant difference (\textcite{mayen_balagtas_alexander_2010}; \textcite{sauer_2010}; \textcite{grovermann_et_al_2021}). The different
conclusions can be explained by many reasons. First, studies are
conducted on different kinds of productions (e.g., olive, crop, dairy)
in different western countries (European countries and the United
States). Second, different kinds of data are used: some studies rely on
panel data, whereas others consider cross-sectional data. Third, some
studies rely on SF analysis (\textcite{tzouvelekas_pantzios_fotopoulos_2001};
\textcite{tzouvelekas_pantzios_fotopoulos_2002}; \textcite{sipilainen_oude_lansink_2005}; \textcite{kumbhakar_tsionas_sipilainen_2009}; \textcite{mayen_balagtas_alexander_2010}; \textcite{tiedemann_latacz-lohmann_2013}; \textcite{grovermann_et_al_2021}), whereas other studies resort to alternative efficiency
estimation strategies such as data envelopment analysis (\textcite{oude_lansink_2002}; \textcite{sipilainen_huhtala_2013}). Fourth, while some
analyses consider a unique frontier for both technologies (\textcite{sipilainen_huhtala_2013}; \textcite{aldanondo2014}), most
of them consider two different frontiers. Finally, numerous studies do
not address the organic farming adoption-selection bias. \textcite{sipilainen_oude_lansink_2005} used the Heckman-type correction by appending
the inverse Mill's ratio to address selection bias. However, it has been
shown that this approach is not suitable for non-linear models such as
the SF approach (\textcite{greene_2010}). To take into account the
endogeneity of the decision to adopt organic practices, \textcite{kumbhakar_tsionas_sipilainen_2009} jointly estimate the production frontier and
the technology choice. \textcite{mayen_balagtas_alexander_2010} as well as \textcite{tiedemann_latacz-lohmann_2013} rely, respectively, on propensity score
matching and Euclidean-distance matching to address the selection bias.

A large share of the literature on AES aims at understanding the
determinants of AES adoption. The number of studies analyzing the impact
of AES adoption on productivity is small. \textcite{dakpo_latruffe_desjeux_jeanneaux_2022}
estimate technical efficiency for farms who adopted AES and conventional
farms, accounting for production heterogeneity with a latent class
model. \textcite{mary_2013} assesses the impact of subsidies on total factor
productivity (TFP) change and shows that AES payments do not have a
significant effect. Two recent studies, \textcite{barath2020effect}
and \textcite{mennig_sauer_2020}, aim at comparing TFP change between AES
adopters and non-adopters. They obtain slightly different results.
Baráth \emph{et al.} find no significant effect of AES
adoption on either TFP change or its components for Slovenian farms.
Mennig and Sauer find similar results for Bavarian dairy
farms, but highlight a positive effect of AES adoption on TFP change for
Bavarian arable farms. Moreover, they show that AES affect significantly
and positively the technical efficiency of arable farms.

Our paper contributes to the previous empirical literature mentioned
above by comparing the effect of two CAP environmental contracts,
organic farming and AES, on farms' productivity and efficiency. From a
public policy point of view, it is an important question. The objective
is to understand if these two instruments could allow achieving the
multiplicity of the CAP objectives such as environment conservation,
food supply and farmers' income security.

Our analysis also contributes to the existing literature from a
methodological point of view. First, we implement the stochastic
metafrontier techniques proposed by \textcite{huang_huang_liu_2014}. To
our \mbox{knowledge}, no study has applied stochastic metafrontier methods to
investigate the effect of AES or organic adoption on farms' efficiency.
We can only mention the study by \textcite{beltran2014}, which applies a metafrontier methodology based on data
envelope analysis. Metafrontier models allow one to compare technical
efficiency of farms within each group (conventional, organic and AES) as
well as between groups in relation to the sector as a whole. Second, in
contrast to previous literature on organic farming which compares a
group of conventional farms to a group of organic farms, we focus on
farms adopting organic farming during the period under study and examine
the evolution of their efficiency. Our third and fourth methodological
contributions concern exclusively the  literature about organic farming, as they have
already been implemented by previous studies for AES farms: we address
selection bias toward the adoption of organic farming with a
difference-in-differences estimator; and we extend the existing
literature by analyzing aggregate productivity change and its
components. It is of interest for policymakers to know how the adoption
of the organic certification affects farms' technical efficiency.
However, a broader knowledge of the effects of the adoption of the
organic certification on farms' productivity will allow policymakers to
better identify the channels through which the organic certification
exerts its impact. To the best of our knowledge, no previous study has
analyzed the effect of organic farming adoption on aggregate
productivity change and its components. The only paper related to this
issue is \textcite{balez2015sources}, which aims at determining the drivers of
TFP growth. He shows that the adoption of organic certification has no
significant effect on TFP change.

The rest of the article is organized as follows. After presenting the
background (section 2), we describe the data used and discuss summary
statistics (section 3). We explain our empirical strategy based on an SF
model and a difference-in-differences estimator (section 4). Finally, we
present and discuss our results (section 5) and provide concluding
remarks (section 6).

\section{Background}

The present study analyzes farms in the French region of
Poitou-Charentes. This region plays a significant role in crop
production within France. In 2014, 55\% of the arable land was dedicated
to the production of cereals and 18\% to the production of oleaginous crops (\textcite{agreste2015}).

\subsection{AES contracts and adoption in Poitou-Charentes}

AES are one of the CAP instruments to encourage the voluntary adoption
of environmental practices by farmers. They consist of five-year
contracts where farmers receive subsidies to compensate for the profit
loss due to the adoption of environmental practices. Farmers receive
payments on a yearly basis and controls can be conducted to verify that
the practices the farmers have committed to implement are indeed
carried out. The second pillar of the CAP finances AES to the extent of
75\%, while the remaining 25\% is co-financed by member-states
(\textcite{ministere_agriculture_2020}). AES have a long history in the CAP, becoming
mandatory for the member states' rural development plans in the 1992 CAP
reform. AES contracts have evolved over the years, gaining importance in
CAP programs. During the 2014--2020 CAP program, the total amount of
public aid devoted to AES doubled compared to the preceding CAP program
(2007--2013). AES corresponding to the 2014--2020 CAP program were
implemented in 2015.

The total AES budget for the Poitou-Charentes region was €177 million
for the 2014--2020 CAP (\textcite{region_nouvelle-aquitaine_2020}). From 2015
to 2019, there were 5,646 contracts signed by 3,428 farms. Thus, 13\% of
the farms in the region have signed an AES contract and 11\% of the region's
agricultural area is under an AES contract (\textcite{region_nouvelle-aquitaine_2020}). While an AES contract could be signed at
any time between 2015 and 2019, 60\% of the farms did so in 2015 when
AES contracts were first offered for the CAP program under study. In the
following years, the new AES contracts signed represented on average
10\% of the contracts for the analyzed period (\textcite{region_nouvelle-aquitaine_2020}). In Poitou-Charentes, producers received between €76 per ha
to €510 per ha depending on the contracted AES.

AES contracts can be classified into three main types: localized
measures, system measures, and measures to protect genetic resources.
Localized measures AES aim at preserving wetlands, biodiversity, water
quality, soils, or landscapes. These AES contracts can require a variety
of practices on the contracted plots: soil coverage and management,
grassland management, irrigation management, hedge and tree management,
crop protection and fertilization management. System measures are AES
contracts introduced in 2015 which, contrary to the localized measures,
are contracted on the entire farm area and aim at globally engaging the
farm towards the adoption of environmental practices by requiring crop
diversification, pesticides reduction and a rational fertilization
management. Finally, AES contracts based on measures to protect genetic
resources aim at protecting endangered breeds and plants, as well as
bees (\textcite{ministere_agriculture_2020}). This type of AES is very marginal in
Poitou-Charentes, as it represents 2\% of the total budget allocated to
AES and benefits only a few beekeepers and breeders. Localized AES and
system measures AES are predominant, as they account for 44\% of
financial resources for 30\% of the land area, and 53\% of financial
resources for 70\% of the land area, respectively. The two most adopted
localized AES contracts are related to the creation and maintenance of a
floral or faunal interest cover and the absence of fertilization of
grasslands. Thus, the adoption of an AES contract leads to a
modification of the technology compared to conventional technology, by
imposing restrictions on input use and the implementation of specific
practices.

\subsection{Organic farming in Poitou-Charentes}

In France, organic products are subject to strict regulatory
requirements and frequent inspections. To be certified as organic,
products must come from a certified organic farm. The organic
certification requires a two-year conversion process where farmers apply
organic farming practices before being allowed to label their products
as organic. French public policies recognized organic farming in 1981
and the requirements were harmonized at the European level in 1992.

The main constraints of the organic farming specifications are based on
the prohibition of the use of synthetic pesticides and fertilizers. The
adoption of organic practices leads to large modifications of the
farming system. Despite the early recognition of organic farming, it has
long remained a niche market. It is only recently that the dynamics of
conversion to organic agriculture have increased. In Poitou-Charentes,
6\% of the agricultural land was cultivated under organic practices in
2019, representing an increase of nearly 30\% of the area compared to
2018 (\textcite{interbio_2021}). Organic farms and farms
converting to organic practices received subsidies from the second
pillar of the CAP. During the previous CAP program, from 2014 to 2020,
farms situated in Poitou-Charentes received 300€/ha/year during the
two-year conversion period and 160€/ha/year once they were certified
(\textcite{Chambre2018}).

\section{Data}

\vspace{-5mm} % Exceptionnellement :-)

\begin{table}[!h]
\centering
\caption {Panel structure of the full database}
\label{Tablestructure}
\begin{tabular}{lrrrrrr}
\midrule
\textbf{Panel length (in years)} & 2 & 3 & 4 & 5 & 6 & 7\\
\midrule
\textbf{Frequency} & 1 & 15 & 30 & 53 & 145 & 446 \\
\textbf{Percentage} & \textless1\% & 2\% & 4\% & 8\% & 21\% & 65\% \\
\midrule
\end{tabular}
\end{table}


The database for the present study is composed of specialized crop farms
in Poitou-Charentes. In our sample, on average, crop production
represents 94\% of farms' total production in value. Anonymized
farm-specific data was obtained from the accounting firm CERFRANCE
Poitou-Charentes. Beyond traditional accounting information, the dataset
includes information on farms' adoption of organic farming and AES and
farms' characteristics. Unlike a census, farmers' engagement with the
accounting firm is voluntary; consequently, our full database is an unbalanced panel dataset from 2014 to 2020. Due to our choice of
identification strategy, which relies on a difference-in-differences
approach (see section 4.2), we only consider farms that are present in the sample
in both 2014 and 2020. This allows us to define for each farm a
pre-treatment period in 2014, when none of the farms had adopted an
environmental contract, and a post-treatment period in 2020, when some
of the farms were under an environmental contract. Our dataset includes
4,424 observations on 690 different farms. This sample represents 11\%
of the region's farms specializing in crop production
(\textcite{recensement_agricole_2020}). Table \ref{Tablestructure} presents the panel structure
of the database, showing that 65\% of the farms are present every year
from 2014 through 2020. Since the choice of accounting firm is not
related to the adoption of environmental contracts or to farms'
productivity, it seems reasonable to assume that there is no attrition
bias in our sample.

% {Table 1. Panel structure of the full database.}

% \begin{center}
%     \captionof{table}{titre de ce tableau}
%     \begin{tabular}{c|c}
%          &  \\
%          & 
%     \end{tabular}
% \end{center}

\newpage

Our focus on a particular region and farm specialization enables us to
compare farms operating within a more uniform environment and a more
homogeneous mix of production technologies than possible with the
national farm accountancy data network (FADN) database. The latter
database only contains information for around fifty farms specializing
in crop production in the Poitou-Charentes region. During the period
under study, 102 farms adopted an AES contract, representing 15\% of the farms in our sample. In addition, 34 farms adopted organic
contracts\footnote{The adoption of an organic contract coincides with a farm's conversion to organic practices. The organic certification is obtained two years later.} from 2015 onwards and 10 farms adopted
organic practices before this date. The latter were excluded from the
sample because our focus is on the effect of environmental contract
adoption. We refer to farms that did not adopt environmental contracts as conventional farms.


Tables 2 and 3 present descriptive statistics by groups of farms
(conventional, AES, and organic) in 2014 and 2020, respectively. Total
farm output (in terms of sales) and materials (which include
fertilizers, pesticides, energy, and other miscellaneous inputs) both
take into account inventory changes. All monetary values were deflated
by output or input price indices with base year 2015 (\textcite{insee_2022}).
Output price indices are available at the national level and input price
indices are available at the regional level. Due to the lack of
organic-specific price indices, identical indices are used for
conventional, AES, and organic farms. The land variable is the utilized
agricultural area (UAA) of each farm. We compute the
Herfindahl-Hirschman Index\footnote{The Herfindahl-Hirschman Index, also
  known as the Simpson diversity index in ecology, has been used in the
  agronomic literature to measure the extent of crop diversification
  (\textcite{basavaraj2016crop}; \textcite{adjimoti2018}) It is calculated
  with the following formula:
  \(Index = 1 - \sum_{i = 1}^{n}P_{i}^{2}\), where \(P_{i}\) is the
  acreage share of the crop \(i\) in the UAA and \(n\) the number of
  crops considered. The index increases as diversification increases.}
to assess the crop diversity of the farms.



\begin{table}[!h]
  \centering
  \tabcolsep=3pt
  \caption{Descriptive statistics -- 2014}
    \label{TableDesc2014}
    \scalebox{.85}{
  \begin{tabular}[]{>{\raggedright}p{2cm}
    D{2cm} D{1.25cm} C{1.25cm} D{1.25cm} D{1.25cm} }  
  \toprule
  \textbf{Variable} & 
  \multicolumn{1}{C{2cm}}{{\bfseries Conventional\par farms}} & 
  \multicolumn{2}{C{3cm}}{{\bfseries AES farms}} &
  \multicolumn{2}{C{3cm}}{{\bfseries Organic farms}} \\
  \midrule
  & \centering mean (standard deviation) 
  & \centering mean (standard deviation) 
  & \centering \emph{Comparison of mean with conventional farms}
  & \centering  mean (standard deviation) 
  & \centering \emph{Comparison of mean with conventional farms} \tabularnewline \midrule
   \# observations & 554 & 102 & & 34 & \\ \midrule
  \multicolumn{6}{l}{\emph{\textbf{Production function}}} \\\midrule
  Total output\textsuperscript{a} (€/ha) & 1,120.91 \varstats{291.49} & 1,140.42
  \varstats{292.53} & \emph{n.s.} & 1,073.47 \varstats{336.64} & \emph{n.s.} \\
  Labor (Man-unit) & 1.34
  
  \varstats{0.56} & 1.37
  
  \varstats{0.67} & \emph{n.s.} & 1.31
  
  \varstats{0.59} & \emph{n.s.} \\
  Land (UAA in ha) & 153.84
  
  \varstats{64.54} & 153.66 \varstats{55.05} & \emph{n.s.} & 145.88 \varstats{55.37} & \emph{n.s.} \\
  Materials\textsuperscript{a} (€/ha) & 928.45
  
  \varstats{223.67} & 906.08 \varstats{226.05} & \emph{n.s.} & 957.11 \varstats{273.13} &
  \emph{n.s.} \\
  Capital\textsuperscript{a} (€/ha) & 767.56
  
  \varstats{558.86} & 767.67 \varstats{578.49} & \emph{n.s.} & 696.80 \varstats{677.22} &
  \emph{n.s.} \\\midrule
  \multicolumn{6}{l}{\emph{\textbf{Source of within group inefficiency}}}  \\\midrule
  Share of employees (\%) & 7.75
  
  \varstats{16.23} & 6.13
  
  \varstats{13.96} & \emph{n.s.} & 10.82 \varstats{22.02} & \emph{n.s.} \\
  Irrigation (dummy) & 0.26
  
  \varstats{0.44} & 0.30
  
  \varstats{0.46} & \emph{n.s.} & 0.26
  
  \varstats{0.45} & \emph{n.s.} \\\midrule
  \multicolumn{6}{l}{\emph{\textbf{Source of between group inefficiency}}}  \\\midrule
  AES subsidies / Farm UAA (€/ha) & 0.00
  
  \varstats{0.00} & 0.00
  
  \varstats{0.00} & \emph{n.s.} & 0.00
  
  \varstats{0.00} & \emph{n.s.} \\
  Herfindahl-Hirschman Index & 0.81
  
  \varstats{0.09} & 0.81
  
  \varstats{0.09} & \emph{n.s.} & 0.81
  
  \varstats{0.12} & \emph{n.s.} \\\bottomrule
  \end{tabular}}
  
  \notedetableau{\textsuperscript{a}Values are deflated by yearly index prices. Wilcoxon rank-sum tests are performed; *, **, *** indicate significance at the 10\%, 5\% and 1\% levels respectively; n.s. stands for non-significant.}  

    \vspace{-\baselineskip}

\end{table}

In 2014, farms that would later adopt AES or organic contracts were
similar to the group of conventional farms (i.e., no significant
differences were detected between the two groups and conventional farms
over the selected characteristic in Table 2). This fact suggests that
farming systems are comparable before the adoption of environmental
contracts. These results differ from a branch of the literature
highlighting for AES and organic contracts differences on farm observed
characteristics before adoption (\textcite{mayen_balagtas_alexander_2010}; \textcite{barath2020effect}; \textcite{mennig_sauer_2020}; \textcite{dakpo_latruffe_desjeux_jeanneaux_2022}). This observation may be explained by two phenomena.
First, it may result from our sample choice, which only includes farms that are
highly specialized in crop production; therefore, farms are very
homogeneous before the adoption of an environmental contract. Second,
farms adopting AES contracts or organic farming after 2014 were not
pioneers. AES have existed for over 20 years in the CAP. In the region
under study, a wave of conversion to organic farming was observed
starting in 2015. In one year, from 2018 to 2019, the cultivated area in
organic farming increased by nearly 30\% (\textcite{interbio_2021}).

\newpage

In 2020, average output per hectare in organic farms was significantly
smaller than in conventional farms. This difference is observed despite
the fact that we have deflated output values for conventional and
organic farms by same price indexes due to lack of information on
organic-specific prices. As organic products typically receive price
premiums in the market, using the same price deflators for organic as
for conventional output should overestimate the true organic output
quantity. In addition, organic farms tended to own more capital per
hectare than conventional farms. Both results are expected, as it has
been shown that the adoption of organic practices negatively affects
crop yields (\textcite{de_ponti_rijk_van_ittersum_2012}; \textcite{seufert_ramankutty_foley_2012}; \textcite{reganold_wachter_2016}; \textcite{smith_etal_2020}) and organic practices rely more on agricultural machinery,
especially for weed control, to compensate for not using pesticides
(\textcite{turner_etal_2007}; \textcite{benaragama2013}).
In addition, the Herfindahl-Hirschman Index for organic farms was
significantly higher than for conventional farms, indicating higher crop
diversification among the former group. These observations imply that
the agricultural production system was altered by the adoption of
practices that comply with the specifications of organic farming.
However, for farms adopting AES contracts, we did not observe any
significant differences in 2020, after the adoption of the contract,
with respect to conventional farms (Table 3). The only exception was
subsidies, which is an expected result. The comparison of conventional farms shows a decrease in the value of
total output between 2014 and 2020, a result confirmed by an analysis of
main crop yields. At the same time, we observe an increase in the UAA,
which highlights a trend toward larger farms by area, and a decrease in
the value of capital. The observed decrease in capital can be explained
by several factors. Firstly, there are cycles of investment in
agriculture depending on the economic context and few investments
occurred during the period under analysis. Secondly, on average
conventional farms tended to outsource an increasing share of farm tasks
through hired custom work (84.88€/ha in 2014 to 102.67€/ha in 2020).
Thirdly, obsolescence and depreciation of durable assets reduce the
value of capital yearly.

% \begin{table}[!h]
\begin{center}
  \captionof{table}{Descriptive statistics -- 2020}
    \label{TableDesc2020}
    \tabcolsep=3pt
    \scalebox{.85}{
  \begin{tabular}[t]{@{}p{2.5cm} *{2}{D{1.5cm}} c D{1.5cm} c @{}}
  \toprule
  \textbf{Variable} & \multicolumn{1}{C{1.5cm}}{{\bfseries Conventional\par farms}} &
  \multicolumn{2}{C{3cm}}{\textbf{AES farms}} &
  \multicolumn{2}{C{3cm}}{\textbf{Organic farms}} \\
  & \multicolumn{1}{C{1.5cm}}{mean
  
  (standard deviation)} & \multicolumn{1}{C{1.5cm}}{mean
  
  (standard deviation)} & \multicolumn{1}{C{1.5cm}}{\emph{Comparison of mean with conventional farms}}
  & \multicolumn{1}{C{1.5cm}}{mean
  
  (standard deviation)} & \multicolumn{1}{C{1.5cm}}{\emph{Comparison of mean with conventional
  farms}} \\ \midrule
  \# observations & 554 & 102 & & 34 & \\ \midrule
\emph{\textbf{Production function}} & & & \\ \midrule
  Total output\textsuperscript{a} (€/ha) & 889.55 \varstats{285.77} & 885.71
  \varstats{257.74} & \emph{n.s.} & 772.42 \varstats{388.11} & ** \\
  Labor (Man-unit) & 1.27  \varstats{0.53} & 1.30  \varstats{0.57} & \emph{n.s.} & 1.27  \varstats{0.47} & \emph{n.s.} \\
  Land (UAA in ha) & 164.13  \varstats{73.15} & 164.32 \varstats{63.16} & \emph{n.s.} & 142.36 \varstats{48.50} & \emph{n.s.} \\
  Materials\textsuperscript{a} (€/ha) & 886.48 \varstats{238.34} & 850.77 \varstats{204.08}
  & \emph{n.s.} & 850.85 \varstats{283.69} & \emph{n.s.} \\
  Capital\textsuperscript{a} (€/ha) & 402.48 \varstats{397.72} & 390.83 \varstats{406.68} &
  \emph{n.s.} & 622.34 \varstats{564.25} & ** \\\midrule
  \multicolumn{3}{@{}>{\raggedright\arraybackslash}p{(\columnwidth - 10\tabcolsep) * \real{0.5625} + 4\tabcolsep}}{%
  \emph{\textbf{Source of within group inefficiency}}} & & & \\\midrule
  Share of employees (\%) & 6.47  \varstats{14.91} & 5.22 \varstats{12.21} & \emph{n.s.} & 7.00 \varstats{17.35} & \emph{n.s.} \\
  Irrigation (dummy) & 0.19  \varstats{0.40} & 0.22  \varstats{0.41} & \emph{n.s.} & 0.26  \varstats{0.45} & \emph{n.s.} \\\midrule
  \multicolumn{6}{@{}>{\raggedright\arraybackslash}p{(\columnwidth - 10\tabcolsep) * \real{1.0000} + 10\tabcolsep}@{}}{%
  \emph{\textbf{Source of between group inefficiency}}} \\\midrule
  AES subsidies / Farm UAA (€/ha) & 0.00 \varstats{0.00} & 39.40 \varstats{43.40} & \multicolumn{1}{l}{***} & 0.00 \varstats{0.00} & \emph{n.s.} \\
  Herfindahl-Hirschman Index & 0.85 \varstats{0.09} & 0.86 \varstats{0.07} & \emph{n.s.} & 0.91 \varstats{0.07} & *** \\\bottomrule
  \end{tabular}}
  
  \notedetableau{\textsuperscript{a}Values are deflated by yearly index prices. Wilcoxon rank-sum tests are performed; *, **, *** indicate
  significance at the 10\%, 5\% and 1\% levels, respectively; n.s. stands
  for non-significant.}
\end{center}



\section{Methodology}

  \subsection{Estimation of stochastic production frontier, aggregate
    productivity and its components}

We adopt the SF model to estimate production frontiers and measure
technical efficiency. It is a parametric method, first proposed by
\textcite{aigner1977} and \textcite{meeusen_van_den_broeck_1977}. The SF model is composed of two error terms and is specified
as follows:
\begin{align}
\ln Y_{it}^{k} = f\left( \ln X_{it}^{k};\beta^{k} \right) + v_{it}^{k} - u_{it}^{k}\
\end{align}
where \(Y_{it}^{k}\) denotes the output quantity of farm
\(i\) with the \(k\) technology at time \(t\), \(f(.)\) describes the
production frontier, \(X_{it}^{k}\) represents the vector of
transformations of input quantities, and \(\beta^{k}\) is a vector of
parameters to be estimated. The error term is composed of a two-sided
stochastic term \(v_{it}^{k}\) accounting for statistical noise, and a
nonnegative stochastic term \(u_{it}^{k}\) representing inefficiency. We
assume \(v_{it}^{k}\) is normally distributed,
\(v_{it}^{k}\sim N(0,\sigma_{vk}^{2})\) and \(u_{it}^{k}\) follows a
truncated normal
distribution \(u_{it}^{k}\sim N^{+}(\mu_{it}^{k},\sigma_{uk}^{2})\).
In addition, \(v_{it}^{k}\) and \(u_{it}^{k}\) are assumed to be
uncorrelated. Inefficiency is computed in an output-oriented model,
i.e., a farm is considered inefficient if a higher level of output could
be produced with a given quantity of input. We estimate a frontier for
each farm group \(k\) (conventional, AES, and organic) because, as
highlighted in section 2, they rely on different technologies due to
input restrictions and specific practices implementation.

For the production technology specification, we consider one output, the
total output of the farm in terms of sales \((Y)\), because the farms in
our sample are highly specialized in crop production (94\% of the total
production in value). In addition, we consider four inputs; namely,
total labor \({(X}_{W})\), land dedicated to agricultural activities
\({(X}_{L})\), materials \({(X}_{M})\) which include fertilizers,
pesticides, energy, and other miscellaneous inputs, and capital
\({(X}_{K})\). \(X_{W}\) is expressed in full-time equivalent annual
labor units, \(X_{L}\) in hectares, and \(X_{M}\) and \(X_{K}\) in
monetary value. As mentioned previously in the data section, all
monetary values were deflated by output or input price indices with base
year 2015 (\textcite{insee_2022}). 

We specify the technology as a translog
production function, as follows:
\begin{align}
\ln Y_{it}^{k} &= \beta_{0}^{k} + \sum_{n}^{}{\beta_{n}^{k}\ln X_{nit}^{k}} + \frac{1}{2}\sum_{n}^{}{\sum_{l}^{}{\beta_{nl}^{k}\ln X_{nit}^{k}\ln X_{lit}^{k}}} \notag \\
               & + \sum_{n}^{}{\beta_{nt}^{k}tlnX_{nit}^{k}} + \beta_{t}^{k}t + v_{it}^{k} - u_{it}^{k}
\end{align}
with \(\beta_{nl}^{k} = \beta_{\ln}^{k}\)for
\(n,l = W,L,M,K\).

We account for potential non-neutral technical change by including
interaction terms between time and input variables in the production
function.

It is a common practice in efficiency analysis to rely on inputs and
outputs expressed in monetary values because of their availability.
Nevertheless, when estimating efficiency scores with monetary values
instead of physical quantities, the efficiency score not only represents
technical efficiency, but also reflects allocative efficiency to a
certain extent. We deflated monetary values to control for the price
effect. Assuming that farmers face the same yearly prices for inputs and
output allows us to recover implicit physical quantities.

Besides the interest in estimating farms' inefficiency scores, from a
public policy point of view, it is beneficial to identify the drivers of
inefficiency. Several models accounting for the determinants of
inefficiency have been developed (\textcite{kumbhakar_ghosh_mcguckin_1991};
\textcite{reifschneider_stevenson_1991}; \textcite{huang_liu_1994}). Here,
we follow the most commonly applied model proposed by \textcite{battese1995model}. It states that the technical inefficiency can be explained
by a set of variables \(z_{it}^{k}\), as follows:
\begin{equation}
u_{it}^{k} = z_{it}^{k}\delta^{k} + w_{it}^{k}    
\end{equation}

\looseness = -1{where \(\delta^{k}\) is a vector of parameters to be estimated and
\(w_{it}^{k}\) is defined by the truncation of the normal distribution
\(N(0,\sigma_{uk}^{2})\), such that the point of truncation is
\(- z_{it}^{k}\delta^{k}\). It follows that
\(u_{it}^{k}\sim N^{+}(z_{it}^{k}\delta,\sigma_{uk}^{2})\). In our
setting, the explanatory variables included in \(z_{it}^{k}\) are 14
regional dummies, share of employed workers (as opposed to family
labor), usage of irrigation (dummy), and size of the farm (in~ha).}

The SF model (equation 2) and the model accounting for the technical
inefficiency effects (equation 3) are estimated simultaneously by
maximum likelihood method using the frontier R package developed by
\textcite{coelli2013frontier}.

\looseness = +1 {To measure productivity change (PC), we follow the methodology developed
by \textcite{orea_2002}\footnote{Equations 5 to 8 are derived in detail
  in \textcite{orea_2002} and \textcite{coelli_et_al_2005}, p.~291--293 and
  300--302.} that allows for the decomposition of a Malmquist
productivity index (MPI) into technical efficiency (TE), technical
change (TC), and scale economies. While the MPI is the most popular
productivity measurement instrument in the absence of price data (\textcite{balk2020toolbox}) it is not an indicator of TFP because it is
not a multiplicatively-complete\footnote{\textcite{odonnell_2012}, p.~875
  defines a multiplicatively-complete TFP index as one that can be
  derived from any TFP index that can be written as a ratio of an output
  quantity index to an input quantity index. The Malmquist productivity
  index first introduced by \textcite{caves1982economic} is
  not multiplicatively-complete.} index (\textcite{odonnell_2012}). The TE of farm \(i\) with the \emph{k}-th technology at
time \(t\) is defined~as:
\begin{equation}
TE_{it}^{k} = \exp\left( - u_{it}^{k} \right)\
\end{equation}}

Thus, the first component, the technical efficiency change (TEC) of farm
\(i\) between the base period \(s\) and the reference period \(t\) is
calculated as follows:
\begin{equation}
TEC_{i,t/s}^{k} = \frac{TE_{it}^{k}}{TE_{is}^{k}}\
\end{equation}

TC for farm \(i\) with the \(k\) technology between period \(s\) and
\(t\) is the geometric mean of the partial derivatives with respect to
time of the production function:
\begin{equation}
TC_{i,t/s}^{k} = \exp\left( \frac{1}{2}\left\lbrack \frac{\partial lnY_{is}^{k}}{\partial s} + \frac{\partial lnY_{it}^{k}}{\partial t} \right\rbrack \right)\
\end{equation}

The last component of PC is the scale change (SC) introduced by \textcite{orea_2002}. SC for farm \(i\) with the \(k\) technology between period
\(s\) and \(t\) is defined as follows:
\begin{equation}
SC_{i,t/s}^{k} = \exp\left( \frac{1}{2}\sum_{n = 1}^{4}{\left\lbrack \varepsilon_{nis}^{k}SF_{is}^{k} + \varepsilon_{nit}^{k}SF_{it}^{k} \right\rbrack*\ln\left( \frac{x_{nit}^{k}}{x_{nis}^{k}} \right)} \right)\
\end{equation}
where \(SF_{is}^{k} = (\varepsilon_{is}^{k} - 1)/\varepsilon_{is}^{k}\),
\(\varepsilon_{is}^{k} = \sum_{n = 1}^{N}\varepsilon_{nis}^{k}\) and
\(\varepsilon_{nis}^{k} = \frac{\partial lnY_{is}^{k}}{\partial{lnX}_{nis}^{k}}\)
which is the output elasticity with respect to each input.

Finally, PC is the sum of the three components:
\begin{equation}
PC_{i,t/s}^{k} = TEC_{i,t/s}^{k} + TC_{i,t/s}^{k} + SC_{i,t/s}^{k}
\end{equation}

\subsection{Estimation of stochastic metafrontier, technology gap ratio
  and technical efficiency with respect to the metafrontier}

Most of the studies comparing TE for organic farms and conventional
farms define two different frontiers for each group and thus measure
differences in efficiencies in relation to the production frontier of
the farms' own type (\textcite{lakner_breustedt_2017}). This type of
analysis enables the discussion of farms' TE in
relation to their respective frontiers within each group, but it does
not allow for comparisons between groups with respect to the entire
sector. On the other hand, with the exception of \textcite{dakpo_latruffe_desjeux_jeanneaux_2022}, most studies comparing the TE of AES farms and conventional
farms assume that both types of farms have identical technologies and,
therefore, compare them based on a single frontier. To avoid making this
assumption while being able to compare farms' TE results between
different groups not sharing the same technology, we apply a stochastic
metafrontier approach. The metafrontier can be described as the frontier
that covers all production technologies. This methodology has been
previously used in agricultural economics analysis (\textcite{melo-becerra_orozco-gallo_2017}; \textcite{alem2019}; \textcite{martinez_cillero_wallace_thorne_breen_2021}; \textcite{delay_thompson_mintert_2022}; \textcite{owusu_bravo-ureta_2022}).

The empirical application of stochastic metafrontiers was first proposed
by \textcites{battese_rao_odonnell_2004,battese_rao_odonnell_2008}. Their approach is based on a two-step procedure. In the
first step, they estimate group-specific frontiers using an SF model,
and in the second step, they utilize mathematical programming techniques
to estimate the metafrontier. We apply the methodology developed by
\textcite{huang_huang_liu_2014}, who estimate the metafrontier in the
second step as stochastic. This approach has two advantages over the
aforementioned methodologies. First, it incorporates stochastic factors
and statistical noise into the estimation of the metafrontier. Second,
the metafrontier estimates obtained from this approach possess
meaningful statistical properties.

In the first step, a frontier is estimated for each group of farms
(conventional, AES, and organic) following the methodology described in
section 4.1. In the second step, the metafrontier \(M\) is estimated as
follows:
\begin{equation}
{\widehat{f}}^{k}\left( \ln X_{it}^{k};\beta^{k} \right) = f^{M}\left( \ln X_{it}^{k};\beta^{M} \right) + v_{it}^{kM} - u_{it}^{kM}\
\end{equation}
where \({\widehat{f}}^{k}\left( \ln X_{it}^{k};\beta^{k} \right)\) is
the estimate of each group frontier from the first step in equation~1.
In equation~9, \(X_{it}^{k}\), \(i\),\(t\) and \(k\) are defined in
the same way as in equation~1. \(f^{M}(.)\) represents the
metafrontier and \(\beta^{M}\) is a vector of the metafrontier
parameters to be estimated. The error term is composed of a two-sided
stochastic term \(v_{it}^{kM}\) accounting for statistical noise and a
nonnegative stochastic term \(u_{it}^{kM}\) representing the technology
gap. The technology gap component is modeled as a set of explanatory
variables like in equation 3 following the Battese and Coelli model. In our setting, we include the Herfindahl-Hirschman
Index (which accounts for farms' crop diversity) and the value of AES
subsidies per hectare as explanatory variables. We define TE in the same
way as in equation~4, indicating the distance of each farm to its
respective technology frontier.\footnote{As previously discussed,
  equation 4 measures TE in relation to the respective frontier within
  each group, but it does not allow for comparisons between groups with
  respect to the entire sector.}

To compare technology efficiencies we compute the technology gap ratio
(TGR), which represents the distance between the technology frontier and
the metafrontier. It is specified as follows:
\begin{equation}
TGR_{it}^{k} = exp\left( - u_{it}^{kM} \right)
\end{equation}

Finally, to compare farms using different technologies in relation to
the entire sector, we calculate the distance of each farm to the
metafrontier, which is referred to as MTE. It is specified as follows:
\begin{equation}
MTE_{it}^{k} = TGR_{it}^{k}*TE_{it}^{k}\
\end{equation}

\subsection{Identification strategy with a difference-in-differences approach}

We estimate separately the effects of adopting AES and organic contracts
on farms' efficiency and productivity, but we follow the same
identification strategy, presented hereafter. The treated group
\((D = 1)\) is composed of farms that adopted an environmental contract
after 2014, and the control group \((D = 0)\) consists of conventional
farms.

Selection bias arises because the adoption of environmental contracts is
voluntary. A large literature has studied the determinants of the
adoption of AES and organic farming (for meta-analyses and literature
reviews see \textcite{latruffe_nauges_2013}; \textcite{lastra-bravo_hubbard_garrod_tolon-becerra_2015}; \textcite{zimmermann_britz_2016}; \textcite{Dessart_barreiro-hurlé_van_bavel_2019}). This literature highlights several categories of
determinants, including farm structure (e.g. farm size, farm location,
share of family labor) and farmer characteristics (e.g. age, education
level, risk aversion, sensitivity to environmental issues). In addition,
\textcite{pietola_lansink_2001} show that the adoption of organic farming
is more likely among farms having low yields. These determinants of the
adoption of environmental contracts also affect farms' productivity.
They act as confounding factors and need to be controlled for in the
identification strategy. Within these potential confounding factors,
some are observable while others are unobservable. Descriptive
statistics in 2014, presented and discussed in the previous section,
highlight that farming systems are comparable before the adoption of
environmental contracts. This is a strength of our sample, which allows
us to avoid employing a selection bias control method on observable
characteristics. To control for selection bias on unobservable
confounding factors, such as risk aversion or sensitivity to
environmental issues, we apply a difference-in-differences approach and
estimate the ATT.

For both the treated and the control group, we define two potential
outcomes: \(Y(1)\) if the farmer adopts an environmental contract during
the period under study and \(Y(0)\) if the farmer does not. We also
define \(pre\) as the period before the implementation of the treatment
and \(post\) as the period after its implementation. According to the
difference-in-differences approach, the effect of the adoption of an
environmental contract can be identified by comparing the change in the
expected outcome of the treated group between periods \(pre\) and
\(post\),
\(E\left\lbrack Y_{post}(1) - Y_{pre}(0) \middle | D = 1 \right\rbrack\),
to the counterfactual change in the expected outcome if farmers would
have not adopted the environmental contract:
\[
E\left\lbrack Y_{post}(0) - Y_{pre}(0) \middle| D = 1 \right\rbrack.
\]

The counterfactual case is not observed, but under the parallel trend
assumption we obtain the following equality:
\[
E\left\lbrack Y_{post}(0) - Y_{pre}(0) \middle| D = 1 \right\rbrack = E\left\lbrack Y_{post}(0) - Y_{pre}(0) \middle| D = 0 \right\rbrack.
\]

According to the parallel trend assumption, the expected outcome
difference between the treated and control group would have been
constant over time in the absence of treatment. It states that the
adoption of environmental contracts does not depend on time-varying
factors. It seems reasonable to consider as fixed over the short period
under study the determinants of the adoption of environmental contracts
identified in the literature, such as farms' structure and farmers'
characteristics. Therefore, with a difference-in-differences approach
the ATT can be defined as follows:
\begin{equation}
ATT = E\left\lbrack Y_{post}(1) - Y_{pre}(0) \middle| D = 1 \right\rbrack - E\left\lbrack Y_{post}(0) - Y_{pre}(0) \middle| D = 0 \right\rbrack
\end{equation}

To recover the ATT effectively by a difference-in-differences approach,
the Stable Unit Treatment Value Assumption (SUTVA) should also be met.
The SUTVA assumption requires the treatment to be homogenous among the
treated group, in our case the adoption of either AES or organic
farming. The SUTVA also requires the treatment to have no effect on the
non-treated unit outcome. The adoption of environmental contracts is low
worldwide, the scale at which agricultural prices are determined. We
believe that the adoption of environmental contracts has no effect on
the prices of inputs and outputs and, hence, no effect on input and
output choices and thus neither on their productivity. In addition, AES
contracts are designed to respect the World Trade Organization green box
requirements (i.e. subsidies must not distort trade).

In our framework, 2014 is the pre-treatment period and 2020 is the
post-\linebreak treatment period. For each environmental contract, we compare
efficiency and productivity for adopters to conventional farms. We apply
the difference-in-difference estimator to compute the ATT and test for
the significance of the effect of environmental contract adoption by
applying a Wilcoxon rank-sum test.

\section{Results and discussion}

\looseness = -1
To test whether the three groups have different technologies and thus
justify the estimation of a production frontier per group, we perform
the likelihood-ratio test proposed by \textcite{battese_rao_odonnell_2004}.
Since the statistical value of the test is significant (142.765***), we
reject the null hypothesis that conventional, AES and organic farms have
the same technology. We also perform the test solely on conventional and
AES farms, which is significant (79.379***) and therefore indicates that
different frontiers should be estimated for conventional and AES farms.

The estimates from the SF models for conventional, AES and organic farms
are presented in Table 4. For the three models, the variance of the
inefficiency term represents more than 85\% of the total variance,
supporting the implementation of SF models. Data were mean-corrected
before estimation to allow the interpretation of first-order parameters
as the elasticities at the sample mean. Elasticities at the sample mean
for all groups follow a similar pattern. The largest elasticity at the
sample mean data point corresponds to materials, followed by land. The
sum of the four production elasticities is larger than one for the
conventional technology (1.04), suggesting slightly increasing returns
to scale, whereas it is lower than one for AES and organic technology
(respectively 0.90 and 0.98), indicating moderate to slightly decreasing
returns to scale. Time coefficients are significant and suggest a
technical regress of the same order of magnitude for conventional farms
and farms adopting the AES technology over the period of study,
respectively 1.50\% and 1.40\% per year. Technical regress is twice as
large for farms adopting the organic technology. The interaction term
between time and materials is significantly positive for the
conventional technology, indicating that TC has been materials-saving.
To a lesser extent, TC has also been capital-saving for the conventional
technology.
Regarding the inefficiency term, results show that for the conventional
technology, a higher share of hired labor and the absence of an
irrigation system are associated with higher inefficiency scores. To a
lesser extent, inefficiency is negatively affected by farmland size,
highlighting that bigger farms tend to be more efficient. The latter
result also applies to the AES farms. Most regional dummies parameters
are significant, indicating soil quality effects on inefficiency;
although they are not reported to save space. No significant effect is
found to explain the inefficiency scores for the organic technology with
the exception of some regional dummies.

{
% \scriptsize
\tabcolsep=2pt
\begin{longtable}{
    >{\raggedright}p{2.5cm} 
    l 
    >{\raggedleft} p{1.3cm} 
    l 
    >{\raggedleft} p{1.3cm} 
    l 
    >{\raggedleft} p{1.3cm} 
    l}
\caption{Maximum likelihood estimation estimates of the
stochastic frontier model for conventional, AES and organic farms} \tabularnewline\tabularnewline
\toprule
\textbf{Variables} & \textbf{Parameters} &  \multicolumn{6}{c}{\textbf{Coefficients} (std. err.)} \tabularnewline\cmidrule{3-8}
 & & \multicolumn{2}{c}{Conventional} & \multicolumn{2}{c}{AES} & \multicolumn{2}{c}{Organic}  \tabularnewline
\midrule
\endfirsthead
%\multicolumn{7}{c}{{\bfseries \tablename\thetable{} -- suite}} \tabularnewline
\midrule
\textbf{Variables} & \textbf{Parameters} &  \multicolumn{6}{c}{\textbf{Coefficients} (std. err.)} \tabularnewline\cmidrule{3-8}
 & & \multicolumn{2}{c}{Conventional} & \multicolumn{2}{c}{AES} & \multicolumn{2}{c}{Organic}  \tabularnewline
\midrule %
\endhead
\bottomrule 
%\multicolumn{7}{r}{{Suite à la page suivante}} \tabularnewline
\endfoot

\midrule
\endlastfoot

\textit{Stochastic frontier}                             &         &                        &         &                        &         & \tabularnewline
Constant                                        & $\beta_0$       & 0.144\varstats{0.013}  & ***     & 0.192\varstats{0.030}  & ***     & 0.090\varstats{0.031}  & *** \tabularnewline
Ln labor                                        &     $\beta_W$    & 0.061\varstats{0.011}  & ***     & 0.033\varstats{0.026}  &         & 0.014\varstats{0.081}  & \tabularnewline
Ln land                                         &   $\beta_L$      & 0.374\varstats{0.018}  & ***     & 0.289\varstats{0.047}  & ***     & 0.386\varstats{0.096}  & *** \tabularnewline
Ln materials                                    &   $\beta_M$      & 0.551\varstats{0.014}  & ***     & 0.524\varstats{0.033}  & ***     & 0.521\varstats{0.062}  & *** \tabularnewline
Ln capital                                      &      $\beta_K$   & 0.049\varstats{0.003}  & ***     & 0.049\varstats{0.006}  & ***     & 0.063\varstats{0.013}  & *** \tabularnewline
(Ln labor)$^2$                                  &    $\beta_{WW}$     & $-$0.027\varstats{0.029} &         & 0.054\varstats{0.076}  &         & $-$0.331\varstats{0.417} & \tabularnewline
(Ln land)$^2$                                   &    $\beta_{LL}$     & 0.007\varstats{0.111}  &         & $-$0.067\varstats{0.203} &         & 0.243\varstats{0.495}  & \tabularnewline
(Ln materials)$^2$                              &     $\beta_{M_M}$    & 0.127\varstats{0.072}  & *       & $-$0.191\varstats{0.180} &         & $-$0.063\varstats{0.248} & \tabularnewline
(Ln capital)$^2$                                &    $\beta_{KK}$     & 0.009\varstats{0.001}  & ***     & 0.006\varstats{0.002}  & ***     & 0.007\varstats{0.004}  & * \tabularnewline
Ln labor * ln land                              &   $\beta_{WL}$      & $-$0.026\varstats{0.046} &         & $-$0.141\varstats{0.096} &         & $-$0.337\varstats{0.192} & * \tabularnewline
Ln labor * ln materials                         &    $\beta_{WM}$     & $-$0.030\varstats{0.039} &         & $-$0.062\varstats{0.087} &         & 0.099\varstats{0.173}  & \tabularnewline
Ln labor * ln capital                           &    $\beta_{WK}$     & 0.015\varstats{0.004}  & ***     & 0.043\varstats{0.011}  & ***     & 0.091\varstats{0.041}  & ** \tabularnewline
Ln land * ln materials                          &    $\beta_{LM}$     & $-$0.120\varstats{0.078} &         & $-$0.013\varstats{0.155} &         & $-$0.163\varstats{0.298} & \tabularnewline
Ln land * ln capital                            &    $\beta_{LK}$     & 0.007\varstats{0.008}  &         & $-$0.037\varstats{0.022} & *       & 0.016\varstats{0.061}  & \tabularnewline
Ln materials * ln capital                       &   $\beta_{MK}$      & $-$0.010\varstats{0.007} &         & 0.023\varstats{0.017}  &         & 0.011\varstats{0.034}  & \tabularnewline
Time                                            &    $\beta_t$     & $-$0.015\varstats{0.002} & ***     & $-$0.014\varstats{0.004} & ***     & $-$0.030\varstats{0.008} & *** \tabularnewline
Ln labor * time                                 &   $\beta_{Wt}$      & 0.003\varstats{0.005}  &         & $-$0.003\varstats{0.010} &         & $-$0.025\varstats{0.021} & \tabularnewline
Ln land * time                                  &   $\beta_{Lt}$      & $-$0.015\varstats{0.008} & *       & 0.019\varstats{0.020}  &         & $-$0.034\varstats{0.031} & \tabularnewline
Ln materials * time                             &   $\beta_{Mt}$      & 0.018\varstats{0.007}  & ***     & 0.003\varstats{0.016}  &         & 0.068\varstats{0.023}  & \tabularnewline
Ln capital * time                               &   $\beta_{Kt}$      & 0.002\varstats{0.001}  & ***     & 0.000\varstats{0.002}  &         & 0.001\varstats{0.004}  & \tabularnewline
\midrule
\textit{Inefficiency model}                              &         &                        &         &                        &         & \tabularnewline
Constant                                        & $\delta_0$       & $-$0.559\varstats{0.376} & ***     & 0.371\varstats{0.190}  & *       & 0.231\varstats{3.068}  & \tabularnewline
Dummies region                                  & $\delta_1$       & yes                    &         & yes                    &         & yes                    & \tabularnewline
Employees share                                 & $\delta_2$       & 0.416\varstats{0.126}  & ***     & $-$0.467\varstats{0.560} &         & 1.793\varstats{4.141}  & \tabularnewline
Land                                            & $\delta_3$       & $-$0.002\varstats{0.001} & ***     & $-$0.003\varstats{0.001} & **      & $-$0.014\varstats{0.010} & \tabularnewline
No irrigation dummy                             & $\delta_4$       & 0.283\varstats{0.103}  & ***     & 0.138\varstats{0.114}  &         & $-$0.145\varstats{2.286} & \tabularnewline
\midrule
$_v \alpha^2 = _u\alpha^2 + \alpha^2$           &         & 0.182\varstats{0.053}  & ***     & 0.110\varstats{0.033}  & ***     & 1.048\varstats{0.576}  & * \tabularnewline
$\gamma = \alpha^2_u/(\alpha^2_v + \alpha^2_u)$ &         & 0.888\varstats{0.029}  & ***     & 0.855\varstats{0.040}  & ***     & 0.986\varstats{0.009}  & *** \tabularnewline
\midrule
Log-likelihood                                  &         & 784.755                &         & 197.883                &         & 43.201                 & \tabularnewline
\midrule
\# of observations                              &         & 3,558                  &         & 651                    &         & 215                    & \tabularnewline
\midrule
\multicolumn{2}{L{3.8cm}}{\% of observation\par respecting monotonicity}       &  \multicolumn{2}{c}{94.52\%}           &  \multicolumn{2}{c}{71.12\%}                 &  \multicolumn{2}{c}{66.51\%} \tabularnewline
\end{longtable}

\vspace{-2\baselineskip}

\notedetableau{*, **, *** indicate significance at the 10\%, 5\% and 1\% levels, respectively.}
}

In Table 5, we present ATT for AES and organic farming on the components
of PC over the period 2014--2020. Results show a decrease in TE for AES,
organic and conventional farms over the period under study. The decrease
is of the same order of magnitude for conventional and AES farms,
respectively $-$2.42\% and $-$3.01\%. The point estimate of the decrease in
TE for farms that adopted the organic technology is more than twice as
large as for conventional farms, at 7.59\%; however, the ATT is not
statistically significant. As previously highlighted when discussing
results from Table 4, the TC over the period is negative for all groups.
The technical regress is twice as large for farms adopting the organic
technology compared to conventional farms, and the ATT is significant.
It shows that in order to generate the same PC as their conventional
counterparts, organic farms needed to increase their technical
efficiency or their scale economies faster than if they had remained
conventional farms. SC is slightly negative for the conventional
technology and positive for both technologies including the voluntary
adoption of environmental practices. Farms adopting AES and organic
certification increase their scale efficiency on average from 2014 to
2020. However, the ATT analysis shows that the effect is not
statistically significant at the 5\% level. Finally, PC from 2014 to
2020 for all groups are negative.


\begin{table}[!hb]
    \centering
    \tabcolsep=1pt
  \caption{Average treatment effect on the treated for AES and
  organic farming on the components of productivity change (2014--2020)}
\begin{tabularx}{\linewidth}{@{} X *{3}{D{2cm}} l @{}}
    \tabularnewline\toprule
    \textbf{Variables} & 
    \multicolumn{1}{C{3cm}}{\textbf{Treated group:}\par \emph{mean (s.d.)}} & 
    \multicolumn{1}{C{3cm}}{\textbf{Control group:}\par conventional farms \emph{mean (s.d.)}} & 
    \multicolumn{2}{C{3cm}}{\textbf{Treatment effects:}\par  \emph{ATT (std. err.)}} \tabularnewline\midrule
    \multicolumn{4}{@{}L{6cm}}{\emph{Treated group: AES farms}} & \\
    TEC (\%) & $-$3.01 \varstats{12.59} & $-$2.42 \varstats{11.90} & $-$0.59 \varstats{1.35} & \\
  TC (\%) & $-$1.38 \varstats{0.71} & $-$1.46 \varstats{0.71} & 0.08 \varstats{0.08} & \\
  SC (\%) & 1.02 \varstats{6.58} & $-$0.18 \varstats{1.95} & 1.20 \varstats{0.66} & *\\
  PC (\%) & $-$3.37 \varstats{14.59} & $-$4.06 \varstats{12.11} & 0.69 \varstats{1.53} & \\
  \multicolumn{4}{@{}L{6cm}}{\emph{Treated group: organic farms}} & \\
  TEC (\%) & $-$7.59 \varstats{24.45} & $-$2.42 \varstats{11.90} & $-$5.17 \varstats{4.22} & \\
  TC (\%) & $-$3.10 \varstats{1.97} & $-$1.46 \varstats{0.71} & $-$1.64 \varstats{0.34} & ***\\
  SC (\%) & 5.96 \varstats{40.25} & $-$0.18 \varstats{1.95} & 6.14 \varstats{6.9} & \\
  PC (\%) & $-$4.72 \varstats{49.63} & $-$4.06 \varstats{12.11} & $-$0.66 \varstats{8.53} & \\\bottomrule
  \end{tabularx}
  
  \notedetableau{TEC = technical efficiency change; TC = technical change; SC =
  scale change; PC = productivity change.
  \emph{t}-tests are performed; *, **, *** indicate significance at the
  10\%, 5\% and 1\% levels, respectively.}
  
\end{table}

The ATTs for PC being non-significant highlights that overall the
adoption of AES or organic certification has had no effect on
differential farm PC. However, as we utilize identical output price
indices for conventional and organic farms, we are likely overestimating
organic PC. To our knowledge, there is no previous study on the effect
of the adoption of organic farming on PC; hence, it is not possible to
compare our results to previous economic literature. Regarding AES
contracts, \textcite{barath2020effect} also found no significant
effect of AES adoption on farms' productivity for Slovenian farms. On
the contrary, \textcite{mennig_sauer_2020} found that AES contracts
increase farms' productivity by 1.2\% per year for crop farms. However,
it is difficult to compare results from different case studies; Mennig and Sauer's dataset is composed of German farms during the
previous CAP program (2007-2013) and crop farms are less specialized as
they consider farms for which crops share in total sales is greater than
66\%. Furthermore, their approach failed to take into account that
technologies could differ between AES and conventional farms. In the
results presented so far, we analyzed agricultural productivity and
efficiency in relation to each farm's own technology. To further compare
farms using different technologies against the sector as a whole, a
metafrontier is considered.

Results of the metafrontier estimation are presented in Table 6. The
variance of the inefficiency term representing 98\% of the total
variance supports the implementation of an SF model to estimate the
metafrontier. Regarding the inefficiency model, farms with higher crop
diversity (larger Herfindahl-Hirschman Index) and higher AES subsidies
per hectare operate on average under a lower production frontier.
Regarding AES subsidies, this is not surprising because input
restriction and environmental practices implementation stringency
increase as AES subsidies per hectare grow. With regard to crop
diversity, the effect is not clear \textit{a priori}. On the one hand, low
diversity can lead to gains from specialization and achieving economies
of scale. On the other hand, crop diversity can generate economies of
scope, as the agronomic literature highlights positive effects,
particularly in pest management and nutrient management such as
nitrogen. Crop diversity is even more important for organic farms as
chemical inputs are prohibited. Our results highlight that the negative
effects of diversification outweigh the positive ones.

\begin{table}
\centering
\renewcommand{\tabletextsize}{%
    \renewcommand{\arraystretch}{.6}% Ajuste l'interligne des tableaux
    \fontsize{8}{9}\selectfont% Définit la taille du texte à 8 points avec un interligne de 9 points
}
\renewcommand{\varstats}[1]{%
    \par
    \raisebox{1.2ex}{\smaller (#1)}% Texte de taille réduite dans une boîte surélevée
}
\caption{Maximum likelihood estimation estimates of the
    metafrontier model}
  \begin{tabularx}{\linewidth}{@{} X c D{1.5cm} l @{}}
    \toprule
  \textbf{Variables} & \textbf{Parameters} &
  \multicolumn{1}{C{1.5cm}}{\textbf{Coefficients}\par (std. err.)} \\
  \midrule
  \emph{Stochastic frontier} & & \\
  Constant & \(\beta_{0}\) & 0.16& ***\\
  && \varstats{0.001} \\
  Ln labor & \(\beta_{W}\) & 0.05& ***\\
  && \varstats{0.002} \\
  Ln land & \(\beta_{L}\) & 0.37& *** \\
  && \varstats{0.003}\\
  Ln materials & \(\beta_{M}\) & 0.54& *** \\
  && \varstats{0.003}\\
  Ln capital & \(\beta_{K}\) & 0.04& *** \\
  && \varstats{0.001}\\
  (Ln labor)\textsuperscript{2} & \(\beta_{WW}\) & $-$0.03& ***\\
  && \varstats{0.007}\\
  (Ln land)\textsuperscript{2} & \(\beta_{LL}\) & 0.04& **\\
  && \varstats{0.019}\\
  (Ln materials)\textsuperscript{2} & \(\beta_{MM}\) & 0.10& *** \\
  && \varstats{0.018}\\
  (Ln capital)\textsuperscript{2} & \(\beta_{KK}\) & 0.00& ***\\
  && \varstats{0.000}\\
  Ln labor * ln land & \(\beta_{WL}\) & $-$0.03& *** \\
  && \varstats{0.010}\\
  Ln labor * ln materials & \(\beta_{WM}\) & $-$0.02& *** \\
  && \varstats{0.008}\\
  Ln labor * ln capital & \(\beta_{WK}\) & 0.01& *** \\
  && \varstats{0.001}\\
  Ln land * ln materials & \(\beta_{LM}\) & $-$0.13& *** \\
  && \varstats{0.015}\\
  Ln land * ln capital & \(\beta_{LK}\) & $-$0.002 \\
  && \varstats{0.002}\\
  Ln materials * ln capital & \(\beta_{MK}\) & $-$0.00& ** \\
  && \varstats{0.002}\\
  Time & \(\beta_{t}\) & $-$0.01& *** \\
  && \varstats{0.000}\\
  Ln labor * time & \(\beta_{Wt}\) & 0.000 \\
  && \varstats{0.001}\\
  Ln land * time & \(\beta_{Lt}\) & $-$0.01& *** \\
  && \varstats{0.002}\\
  Ln materials * time & \(\beta_{Mt}\) & 0.01& *** \\
  && \varstats{0.001}\\
  Ln capital * time & \(\beta_{Kt}\) & 0.00& *** \\
  && \varstats{0.000}\\
  \emph{Inefficiency model} & & \\
  Constant & \(\delta_{0}\) & $-$3.84& *** \\
  && \varstats{0.210}\\
  Herfindahl-Hirschman Index & \(\delta_{5}\) & 3.22& *** \\
  && \varstats{0.177}\\
  AES subsidies / ha & \(\delta_{6}\) & 0.00& *** \\
  && \varstats{0.000}\\
  & & \\
  \(\sigma^{2} = \sigma_{v}^{2} + \sigma_{u}^{2}\) & & 0.02& *** \\
  && \varstats{0.001}\\
  \(\gamma = \sigma_{u}^{2}/(\sigma_{v}^{2} + \sigma_{u}^{2})\) & & 0.97& *** \\
  && \varstats{0.001}\\
  Log-likelihood & & 9,545.840 \tabularnewline
  \# of observations & & 4.424 \\
  \multicolumn{2}{@{}>{\raggedright\arraybackslash}p{(\columnwidth - 6\tabcolsep) * \real{0.6422} + 2\tabcolsep}}{%
  \% of observation respecting monotonicity} & 94.10\% \\\bottomrule
  \end{tabularx}
  
  \notedetableau{*, **, *** indicate significance at the 10\%, 5\% and 1\% levels,
  respectively.}
\end{table}

Table 7 presents ATT on MTEC over the period 2014--2020 for AES and
organic farming. MTEC represents farms' TE with respect to the
metafrontier. Results highlight a negative causal effect of 11.55\% on
MTEC resulting from the adoption of organic certification. It is
interesting to note that this estimate is within the range of the
agronomic literature studying the effect of organic practices on crop
yields ($-$8\% to $-$25\%) (\textcite{de_ponti_rijk_van_ittersum_2012}; \textcite{seufert_ramankutty_foley_2012}; \textcite{reganold_wachter_2016}). As
previously shown in Table 5, adopting organic practices does not
significantly affect farms' TE with respect to their own frontier. The
loss of efficiency in organic farms is explained by the deviation of
their production frontier from the metafrontier: the ATT on the TGR is
$-$7.36\%. These results are complementary to the existing literature on
organic certification and efficiency. They highlight that the adoption
of organic certification leads to a deviation from the production
frontier compared to the metafrontier. Due to the constraints organic
farms face in terms of input restriction and environmental practices
implementation, it is not surprising that they operate under a lower
production frontier. However, organic farms are generally efficient
relative to their own technology. Yet, our results might slightly
underestimate the effect of adopting organic practices on farm
efficiency, as we deflated monetary values with identical output price
indices for conventional and organic farms. \clearpage

\begin{table}[h]
    \centering\tabcolsep=2pt
  \caption{Average treatment effect on the treated for AES and
    organic farming on efficiency (2014--2020)}
  \begin{tabular}[]{@{}
    l *{2}{D{1.8cm}} D{2cm} l
    @{}}
    \toprule
  \textbf{Variables} & 
  \multicolumn{1}{C{2cm}}{\textbf{Treated group}\par \emph{mean (s.d.)}} & 
  \multicolumn{1}{C{2cm}}{\textbf{Control group:}\par conventional farms\par \emph{mean (s.d.)}} &
  \multicolumn{2}{C{2.25cm}}{\textbf{Treatment effects:} \par \emph{ATT (std. err.)}} \\
  \midrule
  \emph{Treated group: AES farms} & \\
  TEC (\%) & $-$3.01 \varstats{12.59} & $-$2.42 \varstats{11.90} & $-$0.59 \varstats{1.35} & \\
  TGR (\%) & 0.34 \varstats{3.36} & $-$0.03 \varstats{0.49} & 0.37 \varstats{0.33} & \\
  MTEC (\%) & $-$2.78 \varstats{12.25} & $-$2.44 \varstats{11.93} & $-$0.34 \varstats{1.31} & \\
  \emph{Treated group: organic farms} & \\
  TEC (\%) & $-$7.59 \varstats{24.45} & $-$2.42 \varstats{11.90} & $-$5.17 \varstats{4.22} & \\
  TGR (\%) & $-$7.39 \varstats{7.20} & $-$0.03 \varstats{0.49} & $-$7.36 \varstats{1.24} & *** \\
  MTEC (\%) & $-$13.99 \varstats{23.63} & $-$2.44 \varstats{11.93} & $-$11.55 \varstats{4.08} & *** \\\bottomrule
  \end{tabular}
  
  \notedetableau{TEC = technical efficiency change; TGR = technology gap ratio;
  MTEC = technical efficiency change with respect to metafrontier. \textit{t}-tests are performed; *, **, *** indicate significance at the
  10\%, 5\% and 1\% levels, respectively.}
\end{table}

Regarding AES, results show no significant effect of the adoption on
farms' efficiency, neither in terms of efficiency relative to their own
production frontier nor relative to the metafrontier. As discussed in
section 2, there are a variety of AES contracts, which are more or less
stringent and apply to a more or less significant portion of the farm.
Thus, depending on the AES contracts and the level of commitment, their
effects on farm efficiency can be heterogeneous. Unfortunately, our
database does not allow us to differentiate among AES contracts.
Previous studies encountered the same methodological issue (\textcite{barath2020effect}; \textcite{mennig_sauer_2020}; \textcite{dakpo_latruffe_desjeux_jeanneaux_2022}). AES subsidies per hectare capture farms' AES
engagement either by contracting on a large part of farmland and/or by
choosing a stringent contract as subsidies increase with environmental
requirements. Despite the limitations of our database, we attempt to
explore the heterogeneity of AES contracts by considering two subgroups
based on whether their AES subsidies are below or above 40€/ha (the
average value perceived within the sample). Results are presented in
table 8. Farms receiving 40€/ha or more in AES subsidies experience a
larger decrease in TEC and MTEC over the study period. However, the ATTs
are not significantly different from zero. In summary, no causal effect
of the adoption of differential AES contracts on efficiency is found.

\begin{table}
  \renewcommand{\tabletextsize}{%
    \renewcommand{\arraystretch}{.8}% Ajuste l'interligne des tableaux
    \fontsize{8}{10}\selectfont% Définit la taille du texte à 8 points avec un interligne de 10 points
}
\centering
  \caption{Average treatment effect on the treated for AES
    sub-groups farms on efficiency (2014--2020)}
  \begin{tabular}[]{@{}lrrr@{}}
    \toprule
  \textbf{Variables} & \multicolumn{1}{m{2cm}}{\centering\textbf{Treated group:}\par \emph{mean (s.d.)}} & 
  \multicolumn{1}{m{2cm}}{\centering\textbf{Control group:}\par conventional farms \emph{mean (s.d.)}} & 
  \multicolumn{1}{m{2cm}}{\centering\textbf{Treatment effects:}\par \emph {ATT (std. err.)}} \tabularnewline \midrule
  \emph{Treated group: AES farms} \\
  TEC (\%) & & & \\
  \emph{AES subsidies/ha ≥ 40} (n~=~37) & $-$4.70 & $-$2.42 & $-$2.28 \\
  & \varstats{9.64} & \varstats{11.90} & \varstats{1.66}\\
  \emph{AES subsidies/ha \textless{} 40} (n~=~65) & $-$2.05 & $-$2.42 &  0.37 \\
  & \varstats{13.98} & \varstats{11.90} & \varstats{1.81}\\
  TGR (\%) & & & \\
  \emph{AES subsidies/ha ≥ 40} (n~=~37) & 0.10 & $-$0.03 & 0.13 \\
  & \varstats{2.13} & \varstats{0.49} & \varstats{0.35} \\
  \emph{AES subsidies/ha \textless{} 40} (n~=~65) & 0.48 & $-$0.03 & 0.51 \\
  & \varstats{3.90} & \varstats{0.49} & \varstats{0.48} \\
  MTEC (\%) & & & \\
  \emph{AES subsidies/ha ≥ 40} (n~=~37) & $-$4.63 & $-$2.44 & $-$2.19 \\
  & \varstats{9.68} & \varstats{11.93} & \varstats{1.67} \\
  \emph{AES subsidies/ha \textless{} 40} (n~=~65) & $-$1.73 & $-$2.44 & 0.71  \\
  & \varstats{13.46} & \varstats{11.93} & \varstats{1.74}\\
  \bottomrule
  \end{tabular}
    \notedetableau{TEC = technical efficiency change; TGR = technology gap ratio;
  MTEC = technical efficiency change with respect to metafrontier.
  \textcolor{red}{\textit{t}-tests are performed: *, **, *** indicate significance at the
  10\%, 5\% and 1\% levels, respectively.}}
\end{table}

Our results highlight a different effect of the two types of
environmental contracts on farms' efficiency. While the adoption of an
AES contract has no effect, the adoption of organic farming drastically
reduces farms' efficiency by 11.55\% between 2014 and 2020. This
observation can be explained by several factors. AES contracts
requirements are lower than the organic requirements. AES contracts can
concern only some plots of the farm as opposed to organic certification,
which requires redesigning the production system. In addition, this
result can be explained by the AES design itself. AES have been
criticized in the literature because of windfall effects: subsidizing
practices that would have been adopted even in the absence of the
subsidies (\textcite{chabe2013}). In our study, we
cannot control whether the adoption of AES contracts coincides with the
adoption of the practices required by the contracts. Farmers may have
implemented the practices before the adoption of the AES contract,
taking advantage of the windfall offered by the subsidies to increase
their income. This is unlikely to be the case for organic farming,
because the adoption of organic practices is not profitable in the
absence of market price premiums paid by consumers.

\section{Conclusion}

Evaluating the effect of AES and organic contracts on farms'
productivity and efficiency is important from a policy point of view,
especially in a context where the CAP is required to achieve a variety
of objectives, both economic and environmental. We found no effect of
the adoption of AES contracts. From a public policy point of view, this
finding could be encouraging as it promotes the adoption of
environmental practices without affecting negatively the economic
objectives of the CAP. However, this finding should be nuanced as
previous studies highlight that AES suffer from some drawbacks (\textcite{uthes_matzdorf_2013}; \textcite{dupraz_guyomard_2019}). While AES are
usually effective, they can be very expensive in light of the
environmental results achieved (\textcite{batary2015role}). It is
partly explained by the windfall effect captured by the farmers
(\textcite{chabe2013}). Given the variety of AES
contracts, it would be important for future research to conduct this
kind of analysis by clearly identifying the type of AES contracted in
order to estimate their respective impacts on farm efficiency. This
requires access to databases with the relevant information and a
sufficiently large number of contract holders per AES to conduct a
statistical analysis. This is particularly relevant in the context of
the implementation of the new instrument of the 2023--2027 CAP, the
eco-schemes. They are funded by the first pillar of the CAP, but their
adoption is voluntary, similar to AES. The French eco-schemes
incorporate requirements from the system AES with lower stringency and
are structured in a similar manner, involving commitments for the entire
farm.

The adoption of organic farming appears to have reduced farms'
efficiency by 11.55\% over the period 2014--2020. This results from the
fact that organic farms, restricted in input use such as pesticides and
fertilizers, operate under a lower production frontier than conventional
farms. Our results show that organic farms are efficient regarding their
own production frontier, and that the efficiency loss is due to the
switch from conventional to organic production systems. It is important
to consider this in perspective with the objectives of the Green Deal,
whose goal is to convert 25\% of EU agricultural land into organic
farming by 2030. An important decline in farms' productivity could raise
questions regarding food security, farmers' income, and competitiveness
of the EU agricultural sector. In addition, if the instruments used to
achieve the Farm-to-Fork objectives lead to a reduction in production,
there is a possibility of leakage effects occurring in regions outside
the EU (\textcite{arvanitopoulos2021}; \textcite{henning_witzke_2021}; \textcite{wesseler_2022}).

In this article, we assessed the impact of adopting AES and organic
contract on farms' technical efficiency. It must be noted that while the
use of inputs like pesticides and fertilizers may increase crop yields,
it can also result in bad outputs such as biodiversity decline and
nitrogen leaching. However, due to the lack of data, we were not able to
take into account the production of bad outputs. A growing body of
literature focuses on developing methods to account for the production
of bad outputs and thus estimate the environmental efficiency of farms
(\textcite{dakpo_jeanneaux_latruffe_2016}). To complement this study, it would be
valuable to evaluate the effect of adopting AES and organic contracts on
farms' environmental efficiency. Implementing these methodologies
requires data on the production of negative outputs; otherwise, strong
and possibly unrealistic assumptions must be made (\textcite{dakpo_lansink_2019}; \textcite{aitsidhoum2020}; \textcite{aitsidhoum2023}).

\looseness = -1
For future research on AES and organic contracts, it would also be
worthwhile to study the effect of the timing of adoption on farms'
productivity. It would be interesting to examine the evolution of
productivity for farms that adopted an environmental contract over a
longer period to determine whether there is a learning effect over time
that allows them to reach efficiency levels closer to those before
adoption.


\begingroup
\setlength{\emergencystretch}{3em}
\printbibliography
\endgroup
\end{refsection}

\end{Article}