\documentclass[11pt]{article}

\usepackage{amsmath}
\usepackage{amssymb}
\usepackage{amsthm}
\usepackage{latexsym}
\usepackage{color}
\usepackage{graphicx}
\usepackage{appendix}
\usepackage{enumerate}
\usepackage{natbib}
\usepackage{eurosym}
\usepackage{caption}
\usepackage{longtable}
\usepackage{multirow, multicol,booktabs}
\usepackage[a4paper,bindingoffset=1cm,%
            left=2.5cm,right=3cm,top=3cm,bottom=3cm,%
            footskip=1cm]{geometry}

\bibliographystyle{agsm}

\DeclareMathOperator*{\argmax}{arg\,max}

\usepackage[french]{babel}
\usepackage[utf8]{inputenc}
\usepackage[T1]{fontenc}
%\DeclareUnicodeCharacter{2061}{}
\usepackage{hyperref}
\usepackage{setspace}
\usepackage{pdfpages}
\usepackage{ulem}
\doublespacing

\def\sym#1{\ifmmode^{#1}\else\(^{#1}\)\fi}

\usepackage{titlesec}
\setlength{\parindent}{0pt} 
\setcounter{secnumdepth}{4}

\titleformat{\paragraph}
{\normalfont\normalsize\bfseries}{\theparagraph}{1em}{}
\titlespacing*{\paragraph}
{0pt}{3.25ex plus 1ex minus .2ex}{1.5ex plus .2ex}

\providecommand{\keywords}[1]
{
  \small	
  \textbf{\textit{Keywords---}} #1
}

\date{}

\renewcommand{\baselinestretch}{1.2}
\newcommand{\commentaire}[1]{}
\addto\captionsfrench{\renewcommand{\tablename}{Tableau}}



\title{L'auto-protection influence-t-elle les choix d'assurance des individus ? Une étude expérimentale}


\date{}

\begin{document}


\maketitle

\begin{center}
\textbf{Tableaux et figures}
\end{center}

\section{Tableaux}


\begin{longtable}[c]{ccc}
\caption{Valeurs des paramètres}
\label{tab:parameters}\\

\hline\hline
Paramètre&Définition&Valeurs\\
\hline
\endfirsthead

\hline
Paramètre&Définition&Valeurs\\
\hline
\endhead

\hline
\endfoot

\hline\hline
\endlastfoot
$T$&Nombre de périodes&12\\
$W$&Richesse initiale&1500\\
$\bar{p}$&Probabilité de perte&\{0,2 ; 0,5\}\\
$L$&Montant de la perte&\{600 ; 1500\}\\
$J$&Taille du menu&4\\
$\alpha$&Taux de couverture&\{0,4 ; 0,65 ; 0,80 ; 0,90\}\\
$e$&Effort d'auto-protection&\{0,1,...,15\}\\
$\lambda$&Taux de chargement&\{-0,2 ; 0 ; 0,2\}\\
$c(e)$&Fonction de coût d'auto-protection&$c(e)=c e $ avec $
c = \{6 ; 15\}$\\
$p(e)$&Fonction de diminution du risque &$p(e)=\bar{p} - a e $ avec $ a=0,01$\\
\end{longtable}

\newpage


\footnotesize

\begin{longtable}[c]{cccc}
\caption{Description des options disponibles selon les traitements}
\label{tab:treatment_option} \\
\hline
       Traitement & Choix de l'assurance & Taux de couverture & Auto-protection \\

\hline
\endfirsthead

\hline\hline
         Traitement & Choix de l'assurance & Taux de couverture & Auto-protection \\

\hline
\endhead

\hline
\endfoot

\hline \hline
\endlastfoot
\textit{Assurance + Auto-protection} & Choix du contrat  &  $\alpha \in \{0,40;0,65;0,80;0,90\}$ & Disponible \\
\textit{Assurance} & Choix du contrat &  $\alpha \in \{0,40;0,65;0,80;0,90\}$ & Non disponible \\
\textit{Auto-protection} & Contrat imposé &  $\alpha \in \{0,40;0,65;0,80;0,90\}$ & Disponible \\

\end{longtable}
\normalsize

\newpage


\footnotesize

\begin{longtable}[c]{ccc}
\caption{Description des programmes d'optimisation selon les traitements}
\label{tab:treatment_opti} \\
\hline
       Traitement & Programme d'optimisation & Variables de décision \\

\hline
\endfirsthead

\hline\hline
         Traitement & Programme d'optimisation & Variables de décision \\

\hline
\endhead

\hline
\endfoot

\hline \hline
\endlastfoot
\textit{Assurance + Auto-protection} & $(\alpha^*,e^*)=argmax~EU(\alpha,e)$ & $\alpha \in \{0,40;0,65;0,80;0,90\}$ et $0 \le e \le 15$ \\
\textit{Assurance} & $\alpha^*=argmax~EU(\alpha,0)$ &$ \alpha \in \{0,40;0,65;0,80;0,90\}$\\
\textit{Auto-protection} & $e^*=argmax~EU(\bar{\alpha} ,e)$ &  $0 \le e \le 15$\\
\end{longtable}
{\footnotesize Notes : $\bar{\alpha}$ est attribué à chaque participant de manière aléatoire, avec une probabilité équivalente ($1/4$) parmi les valeurs possibles de $\alpha \in \{0,40; 0,65; 0,80; 0,90\}$. \\
\normalsize
\newpage

\footnotesize

\begin{longtable}[c]{lcccccccc}
\caption{Description de l'échantillon en fonction du traitement}
\label{tab:sample} \\
\hline\hline
        & \multicolumn{1}{c}{N}& \multicolumn{1}{c}{N. Obs.}& \multicolumn{2}{c}{Âge} & \multicolumn{2}{c}{Sexe} & \multicolumn{2}{c}{Niveau d'éducation}\\
        & & & \textit{Moy.} & \textit{Écart-type} & \textit{Homme} & \textit{Femme} & \textit{Licence} & \textit{Master} \\
\hline
\endfirsthead

\hline\hline
          & \multicolumn{1}{c}{N}& \multicolumn{1}{c}{N. Obs.}& \multicolumn{2}{c}{Âge} & \multicolumn{2}{c}{Sexe} & \multicolumn{2}{c}{Niveau d'éducation}\\
        & & & \textit{Moy.} & \textit{Écart-type} & \textit{Homme} & \textit{Femme} & \textit{Licence} & \textit{Master} \\
\hline
\endhead

\hline
\endfoot

\hline\hline
\endlastfoot
\textit{Assurance + Auto-protection} & 47 & 564 & 23,01 & 2,53 & 63,8\% & 36,2\% & 32\% & 68\% \\
\textit{Assurance} & 33 & 396 & 22,50 & 2,75 & 72,7\% & 27,3\% & 39\% & 61\%\\
\textit{Auto-protection} & 43 & 516 & 22,33 & 2,40 & 65,1\% & 34,9\% & 33\% & 67\% \\
Total & 123 & 1476 & 22,66 & 2,55 & 66,7\% & 33,3\% & 34\% & 66\%\\
\end{longtable}

\normalsize


\newpage



\footnotesize

\begin{longtable}[c]{lc@{\hspace{0mm}}lc@{\hspace{0mm}}lc@{\hspace{0mm}}l}
\caption{Analyses multivariées du choix d'assurance ($\alpha$)}
\label{tab:insurance_main_results} \\
\hline\hline
        &\multicolumn{2}{c}{Modèle 1}&\multicolumn{2}{c}{Modèle 2}&\multicolumn{2}{c}{Modèle 3}\\
        &\multicolumn{2}{c}{Choix}&\multicolumn{2}{c}{P($\alpha=0,40$)}&\multicolumn{2}{c}{P($\alpha=0,90$)}\\
        &\multicolumn{2}{c}{d'assurance}&\\
\hline
\endfirsthead

\hline\hline
         &\multicolumn{2}{c}{Modèle 1}&\multicolumn{2}{c}{Modèle 2}&\multicolumn{2}{c}{Modèle 3}\\
        &\multicolumn{2}{c}{Choix}&\multicolumn{2}{c}{P($\alpha=0,40$)}&\multicolumn{2}{c}{P($\alpha=0,90$)}\\
        &\multicolumn{2}{c}{d'assurance}&\\
\hline
\endhead

\hline
\endfoot

\hline\hline
\endlastfoot
\textbf{\textit{Caractéristiques individuelles}} \\
\textbf{Genre} \\
Femme & Réf. && Réf. && Réf. &\\
Homme       &           0,091   &      &       0,784&\sym{**}       &       0,490    &     \\
                    &     (0,289)  &             &     (0,356)      &          &     (0,351)  &       \\

 \\
  \textbf{Aversion au risque}                &          0,118     &          &     -0,050   &      &       0,188    &    \\
                  &     (0,099)   &           &     (0,186)    &     &     (0,120)       &  \\
\hline
\textit{\textbf{Paramètres du risque}} \\
\textbf{Probabilité de perte ($\bar{p}$)} \\
Faible probabilité ($\bar{p}=0,2$) & Réf.& & Réf.& & Réf.& \\
Forte probabilité  ($\bar{p}=0,5$)        &       0,461&\sym{***}&      -1,069&\sym{***}&     -0,001   &     \\
                  &     (0,124)     &        &     (0,102)    &    &    (0,158)   &      \\

\textbf{Montant de perte ($L$)} \\
Faible perte ($L=600$) & Réf.& & Réf. && Réf. &\\
Forte perte ($L=1500$)  &       0,645&\sym{***}        &      -0,546&\sym{***} &       0,648&\sym{***}\\
                  &     (0,124) &    &     (0,100)&       &    (0,160)      &   \\
\hline
\textbf{\textit{Caractéristiques du contrat}} \\
\textbf{Taux de chargement ($\lambda$)} \\
Nul ($\lambda=0$) & Réf.& & Réf.& & Réf. &\\
Négatif  ($\lambda=-0,2$)&        0,331&\sym{**} &        -0,469&\sym{**} &      0,124   &     \\
                  &     (0,151)    &            &     (0,218)  &     &     (0,189)    &     \\
Positif ($\lambda=0,2$)&         -0,424&\sym{***}&        0,270  &       &          -0,715&\sym{***}\\
                  &     (0,150)      &          &     (0,204)     &     &     (0,199)  &       \\
\hline
\textbf{\textit{Traitement}} \\
\textit{Assurance} & Réf. && Réf.& & Réf. &\\
\textit{Assurance + Auto-protection}       &      -0,226      &          &       0,715&\sym{**}                      &     -0,081    &     \\
                           &     (0,276)  &       &                        (0,330)  &       &              (0,329)    &     \\
\hline
Constante      &        -      &      &      -1,341&\sym{*}        &      -2,486&\sym{***}\\
&              &       &     (0,809)   &      &   (0,828)         \\
Seuil 1     &      -0,326& &           -    &      &            -     &     \\
                  &     (0,682)    &     &                         \\

Seuil 2            &       0,831    &     &        -         &    &          -      &          \\
                    &     (0,682)   &      &                             \\
Seuil 3            &       1,540&\sym{***}&       -       &       &           -        &        \\
                   &     (0,383)    &     &                                   \\
\hline
\(N\)            &         \multicolumn{2}{c}{960}                &        \multicolumn{2}{c}{960}            &      \multicolumn{2}{c}{960}           \\
\end{longtable}
{\footnotesize Notes: Significativité : $^{*}$ = 10\% $^{**}$ = 5\% $^{***}$ = 1\%. Les écart-types sont entre parenthèses. Toutes les régressions comprennent des effets aléatoires. } \\
\normalsize


\newpage


\footnotesize
\begin{longtable}[c]{l*{1}{c}@{\hspace{0mm}}l}
\caption{Analyses multivariées du niveau d'effort d'auto-protection $e$}
\label{tab:sp_main_results} \\

\hline\hline
            &\multicolumn{2}{c}{$e$}\\
\hline
\endfirsthead

\hline\hline
            &\multicolumn{2}{c}{$e$}\\
\hline
\endhead

\hline
\endfoot

\hline \hline
\endlastfoot
\textit{\textbf{Caractéristiques individuelles}} & \\
\textbf{Genre} & \\
Femme  & Réf. \\
Homme            &       1,126         \\
                  &     (0,778)            \\
 & \\
\textbf{Aversion au risque}      &      -0,367              \\
                  &     (0,319)         \\
\hline
\textit{\textbf{Paramètres du risque}} & \\
\textbf{Probabilité de perte ($\bar{p}$)} &\\
Faible probabilité ($\bar{p}=0,2$) & Réf. \\ 
Forte probabilité ($\bar{p}=0,5$)    &       0,646&\sym{**}  \\
               &     (0,290)        \\
\textbf{Montant de la perte ($L$)} & \\
Faible perte ($L=600$)  & Réf. \\
Forte perte ($L=1 500$)  &      -1,700&\sym{***}\\
                 &     (0,290)               \\
\hline
\textit{\textbf{Caractéristiques du contrat}} & \\
\textbf{Taux de chargement ($\lambda$) }& \\
Nul ($\lambda=0$)  & Réf. \\
Négatif ($\lambda=-0,2$)         &       0,739&\sym{**} \\
                  &     (0,352)             \\
Positif ($\lambda=0,2$)       &       0,748&\sym{**} \\
                  &     (0,351)               \\
\textbf{Taux de couverture ($\alpha$)} & \\
40\% & Réf. \\

65\% &    -2,609&\sym{***}\\
                &     (0,588)          \\
80\%  &      -3,782&\sym{***} \\
                 &     (0,653)                \\
90\%  &       -4,619&\sym{***}      \\
                  &     (0,595)      \\
%\textbf{Traitement }& \\
%Traitement \textit{Assurance}  & Réf.  \\
%Traitement \textit{Assurance + Prévention} & -0,944 \\
%& (0,875) \\
\textbf{Taux de couverture $\times$ Traitement} \\
40\% $\times$ Traitement \textit{Assurance + Auto-protection}      &   -1,602&\sym{*}    \\
                  &     (0,909)             \\
65\% $\times$ Traitement \textit{Assurance + Auto-protection} &      0,359\\
                 &     (0,934)     \\
80\% $\times$ Traitement \textit{Assurance + Auto-protection} &     -0,228 \\
                  &     (0,953)               \\
90\% $\times$ Traitement \textit{Assurance + Auto-protection} &     -0,933 \\
                  &     (0,889)               \\

\hline
Constante     &            10,458&\sym{***}\\
                &     (2,073)            \\
\hline
N             &      \multicolumn{2}{c}{1080}        \\
\end{longtable}
{\footnotesize Notes: Significativité : $^{*}$ = 10\% $^{**}$ = 5\% $^{***}$ = 1\%. Les écart-types sont entre parenthèses. La régression inclut des effets aléatoires. } \\

\normalsize
\newpage


\begin{longtable}[c]{c|cccc|c|cccc|c|c}
\caption{Élicitation de l'aversion au risque}
\label{tab:Holt_Laury}\\

\hline\hline
Décision &\multicolumn{4}{c}{Option A} &E(A)&\multicolumn{4}{c}{Option B} & E(B) & E(A)-E(B)\\
\hline
\endfirsthead

\hline
Décision &\multicolumn{4}{c}{Option A} & E(A)&\multicolumn{4}{c}{Option B} & E(B) &E(A)-E(B)\\


\hline
\endhead

\hline
\endfoot

\hline\hline
\endlastfoot
1 & 10\% & -400 & 90\% & -450 &-445 & 10\% & -100 & 90\% & -800 & -820 & 375\\
2 & 20\% &-400 & 80\% & -450 & -440 &20\% & -100 & 80\% & -800 & -740 & 300\\
3 & 30\% &-400 & 70\% & -450 & -435 &30\% & -100 & 70\% & -800 & -660 & 225\\
4 & 40\% &-400 & 60\% & -450 & -430 & 40\% & -100 & 60\% & -800 & -580 & 150\\
5 & 50\% &-400 & 50\% & -450 & -425 & 50\% & -100 & 50\% & -800 & -500 & 75\\
6 & 60\% &-400 & 40\% & -450 & -420 & 60\% & -100 & 40\% & -800 & -420 & 0\\
7 & 70\% &-400 & 30\% & -450 & -415 & 70\% & -100 & 30\% & -800 & -340 & -75\\
8 & 80\% &-400 & 20\% & -450 & -410 & 80\% & -100 & 20\% & -800 & -260 & -150\\
9 & 90\% &-400 & 10\% & -450 & -405 & 90\% & -100 & 10\% & -800 & -180 & -225\\
10 & 100\% &-400 & 0\% & -450 & -400 & 100\% & -100 & 0\% & -800 & -100 & -300\\
\end{longtable}





\normalsize


\newpage


\footnotesize



\footnotesize
\begin{longtable}[c]{l*{1}{c}@{\hspace{0mm}}l}
\caption{Analyses multivariées de l'effort d'auto-protection $P(e=0)$}
\label{tab:control_supply} \\
\hline\hline
            &\multicolumn{2}{c}{$P(e=0)$}\\
\hline
\endfirsthead
\hline\hline
            &\multicolumn{2}{c}{$P(e=0)$}\\
\hline
\endhead
\hline
\endfoot
\hline \hline
\endlastfoot
\textit{\textbf{Caractéristiques individuelles}} & \\
\textbf{Genre} & \\
Femme  & Réf. \\
Homme            &       -0,167          \\
                  &     (0,417)            \\
 & \\
\textbf{Aversion au risque}     &      0,209              \\
                  &     (0,170)         \\
\hline
\textit{\textbf{Paramètres du risque}} & \\
\textbf{Probabilité de perte ($\bar{p}$)} &\\
Faible probabilité ($\bar{p}=0,2$) & Réf. \\ 
Forte probabilité ($\bar{p}=0,5$)    &       -0,141 \\
               &     (0,162)        \\
\textbf{Montant de perte ($L$)} & \\
Faible perte ($L=600$)  & Réf. \\
Forte perte ($L=1 500$)  &      0,676&\sym{***}\\
                 &     (0,164)               \\
\hline
\textit{\textbf{Caractéristiques du contrat}} & \\
\textbf{Taux de chargement ($\lambda$) }& \\
Nul ($\lambda=0$)  & Réf. \\
Négatif ($\lambda=-0,2$)         &       -0,381&\sym{**} \\
                  &     (0,197)             \\
Positif ($\lambda=0,2$)       &       -0,351&\sym{*} \\
                  &     (0,197)               \\
\textbf{Taux de couverture ($\alpha$)} & \\
40\% & Réf. \\
65\% &    0,599&\sym{**}\\
                &     (0,256)          \\
80\%  &      1,185&\sym{***} \\
                 &     (0,265)                \\
90\%  &       2,271&\sym{***}      \\
                  &     (0,257)      \\
\textbf{Traitement }& \\
Traitement \textit{Assurance}  & Réf.  \\
Traitement \textit{Assurance + Auto-protection} & -0,401 \\
& (0,399) \\
\hline
Constante     &           -2,777&\sym{***}\\
                &     (1,107)            \\
\hline
N             &        1 080          \\
\end{longtable}
{\footnotesize Notes : Significativité : $*$ = 10\% $**$ = 5\% $***$ = 1\%. Les écart-types sont entre parenthèses. La régression inclut des effets aléatoires. } \\

\newpage

\section{Figures}

\begin{center}
\captionof{figure}{Choix d'assurance ($\alpha$) en fonction du traitement}
\includegraphics[scale=0.5]{Figures/distrib_couv_treatment_english.png}
\label{fig:sp_couv}
\end{center}

\newpage

\begin{center}
    \captionof{figure}{Niveau moyen d'effort d'auto-protection ($e$) en fonction du traitement}
    \includegraphics[scale=0.5]{Figures/insuranceandprevandprev__effort.png}
    \label{fig:sp_treatment}
\end{center}

\end{document}

