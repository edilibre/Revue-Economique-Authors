\begin{Article}%
[Auteur={Thierry Kamionka\\ Sandrine Spaeter-Loehrer}, 
Titre={Avancées récentes\\ en microéconomie appliquée}]

% Active le positionnement absolu sur toute la page
\begin{textblock*}{50mm}(100mm,46mm) 
    \Large Introduction
\end{textblock*}

\label{intro}

\begin{refsection}[Intro]
\selectlanguage{french}

\lettrine{C}{e numéro spécial} de la \emph{Revue économique} est composé d'une
sélection d'articles présentés aux 39\textsuperscript{es} Journées de
microéconomie appliquée (JMA) qui se sont déroulées à l'Université de
Strasbourg, les 8 et 9 juin 2023. Ces journées ont été organisées par le
Bureau d'Économie Théorique Appliquée (le BETA), avec le soutien de
l'ITI Makers, de l'INRAE et de l'Université de Strasbourg. Elles ont
accueilli de nombreux chercheurs -- jeunes, moins jeunes et doctorants -- dont
la présentation des travaux originaux en présentiel, dans un contexte post-confinements, a été un grand succès.

\medskip

La conférence inaugurale a été donnée par le professeur Alexandra
Spitz-Oener de la Humboldt Universität zu Berlin. Ses recherches s'inscrivent
principalement dans l'économie du travail. Elles portent sur la
compréhension des défis posés aux travailleurs par les changements
technologiques, le commerce et l'immigration. Alexandra
Spitz-Oener a publié dans les plus grandes revues internationales. Elle
est aujourd'hui directrice adjointe de la ROOKWOOL Foundation
Berlin Institute for the Economy and the Future of Work (RF Berlin), et
chercheuse associée à l'Institut de la Recherche sur
l'Emploi (Institute for Employment Research, IAB)
à Nuremberg. Elle est également chercheuse titulaire du Centre de
recherche collaborative/TransRegio «Rationalité et concurrence» (CRC
TRR 190), financé par la German Sciences Foundation.

\medskip

Les articles publiés dans ce numéro spécial ont été présentés à la
conférence, et soumis pour publication à la \emph{Revue économique}. Ils
ont été acceptés au terme d'une procédure associant un éditeur de la
revue, la responsable strasbourgeoise de l'organisation des JMA2023, et
deux rapporteurs anonymes choisis pour leurs compétences méthodologiques
et/ou leur connaissance du domaine de recherche.

\medskip

\textbf{Le premier article} de ce numéro spécial est issu de la
présentation inaugurale d'Alexandra Spitz-Oener. Dans cet article, est
étudiée la mobilité professionnelle en Allemagne de l'Est (RDA) pendant
la période de transformation politique appelée «Wende»
(1989 à 1992). Cette étude repose sur des données administratives
individuelles avant et après la date de la réunification des deux
Allemagnes (3~octobre 1990). La démarche consiste à comparer les
mobilités professionnelles de salariés localisés en RDA avant la
réunification avec celles de salariés situés en RFA avant ce tournant
historique et appartenant aux mêmes cohortes de naissance. Il s'agit
d'étudier au niveau des individus les conséquences sur le marché du travail  
induites par les transformations politiques, économiques et sociales profondes
liées à la réunification allemande. Les données utilisées sont
originales et remarquables dans la mesure où elles mobilisent des
informations pour les salariés de la RDA avant la réunification, issues
du «stockage de données sur le travail social» (Datenspeicher
Gesellschaftliches Arbeitsvermogen) provenant des archives fédérales d'Allemagne. Ces données sont appariées avec les biographies
d'emploi intégrées (données IEB). Les données IEB
contiennent les trajectoires complètes en matière d'emploi et de
rémunération des travailleurs couverts par l'assurance maladie en
Allemagne. Les données sur la RDA permettent d'obtenir
des informations sur le marché du travail et des caractéristiques
individuelles pour l'année 1989. L'appariement a été réalisé sur la base
du nom, du sexe et de la date de naissance. Le groupe de comparaison
pour la RFA provient de l'échantillon représentatif SIAB (de
l'institut IAB) et représente 2~\% de toutes les personnes employées
relevant de la sécurité sociale en Allemagne (hors fonctionnaires et
indépendants). À partir de ce dernier fichier, Alexandra Spitz-Oener a
sélectionné les personnes en emploi en 1989 dans les mêmes cohortes de
naissance que les cohortes issues des données provenant des archives fédérales
d'Allemagne. Trois cohortes de naissance sont retenues~: la cohorte
1939-1941 des personnes nées au début de la Seconde Guerre mondiale, la
cohorte 1951-1953 des individus nés pendant une période où RDA et RFA étaient séparées et confrontées à la reconstruction post-conflit, et la cohorte 1959-1961 des personnes nées pendant une période de tensions liées à la Guerre froide. La
mobilité professionnelle est caractérisée par six catégories provenant
de la classification des professions de Blossfeld (Schimpl-Neimanns [2003]). Les résultats issus de la comparaison des mobilités des salariés
des deux parties de l'Allemagne sont nombreux et intéressants. En
particulier, la proportion d'emplois qualifiés ou non qualifiés était
plus élevée chez les salariés de la RDA (mécaniciens, bouchers,
travailleurs du bâtiment) et, inversement, la part d'emplois
professionnels et semi-professionnels était plus grande chez les
salariés de la RFA (ingénieurs, techniciens, enseignants). Concernant
les trajectoires sur le marché du travail, les résultats indiquent
notamment l'existence de mobilités professionnelles ascendantes ou
descendantes plus prononcées pour les salariés issus de la RDA et d'une
cohorte donnée par rapport aux salariés de la même cohorte provenant de
la RFA. À l'inverse, les salariés issus de la RFA présentent des
trajectoires professionnelles plus stables. Cette étude permet de mettre
en évidence des structures d'emploi pré-réunification très
différentes lorsque l'on compare la RDA et la RFA. L'article met aussi
en évidence que la mobilité professionnelle des salariés présente une
hétérogénéité selon la partie de l'Allemagne où ils étaient localisés en
1989 (RDA vs. RFA).

\medskip

Cinq autres articles viennent compléter ce numéro spécial. Ils portent
sur des domaines variés de la microéconomie appliquée, tels que
l'économie de la dépendance, l'économie du risque, la théorie des jeux,
l'économie agricole ou celle des incitations à l'effort. Les techniques
utilisées relèvent de l'économétrie, de l'économie expérimentale, du
traitement de données d'enquêtes. L'approche théorique est également
mobilisée.

\medskip

\textbf{L'article de Gisèle Umbhauer} s'inscrit dans le domaine de la
théorie des jeux. L'analyse des comportements des joueurs dans le cadre
d'un jeu de devinette permet d'apprendre sur le processus de
raisonnement des joueurs et de son hétérogénéité. Un de ces jeux a été
proposé par John~M. Keynes [1936] et est souvent référencé comme un
«concours de beauté». Ce type de jeu a déjà été étudié dans la
littérature, notamment par Nagel [1995]. Ce dernier considère un jeu où
un grand nombre de joueurs doivent annoncer un nombre entier compris
entre 0 et une limite supérieure. Le gagnant est celui qui propose le
nombre le plus proche de la moyenne des annonces multipliée par p, où p
est un nombre fixé positif ou nul mais plus petit que 1. Ce nombre est
connu de l'ensemble des joueurs. Le gagnant obtient un montant fixé à
l'avance. Dans le cas de la présence d'\emph{ex aequo}, les gagnants se
partagent ce montant par parts égales. Les autres joueurs ne reçoivent
rien. Il existe un et un seul équilibre de Nash dans ce jeu. Il consiste
pour tous les joueurs à proposer la même somme égale à 0. Dans ce cas,
ils reçoivent tous le même montant. Dans la pratique, la mise en œuvre
de ce jeu peut conduire les joueurs à annoncer des nombres entiers
distincts du nombre 0. Une explication à ce comportement réside dans la
façon dont chaque joueur perçoit le processus de raisonnement des auteurs
joueurs. Un joueur de niveau 0 choisira un nombre important pour lui. Un
joueur de niveau 1 considérera que les autres sont des joueurs de niveau
0 et choisira d'annoncer le nombre entier qui maximise son gain en
réponse à ce comportement des autres joueurs. Enfin, un joueur de
niveau-\emph{k} va considérer que les autres joueurs sont des joueurs de niveau-\emph{k}$-$1 et détermine sa meilleure réponse dans ce contexte. \emph{k} caractérise
alors la profondeur du raisonnement. Le dispositif de Gisèle Umbhauer se
distingue de ce jeu de devinette en ce qu'il comporte deux joueurs et
que leur annonce de montant est comprise entre 11 et 20. Tous les
joueurs reçoivent le montant qu'ils demandent respectivement. Celui dont
le montant annoncé est plus petit d'une unité par rapport au montant demandé par
l'autre joueur reçoit un bonus de 20 unités (on parle de jeu 11-20/bonus
20). Ce jeu qui se distingue du jeu de devinette a été aussi considéré
par Arad et Rubinstein [2012]. Il conserve la possibilité d'analyser
la profondeur du raisonnement tout en étant plus simple à mettre en
œuvre. Ce jeu ne comporte pas de stratégie dominée ni d'équilibre de
Nash en stratégies pures à la différence de ce qui se passe dans le jeu
de devinette. En effet, dans le jeu de devinette, le raisonnement de
niveau-\emph{k} va converger vers l'équilibre de Nash. Un autre avantage du jeu
11-20/bonus 20 est que les montants reçus par les joueurs sont fonctions
des montants annoncés. Aussi les joueurs adverses au risque peuvent-ils
annoncer des montants élevés. Il est difficile dans le jeu 11-20/bonus
20 d'anticiper les montants annoncés par les autres joueurs, de sorte que
les décisions des joueurs peuvent entraîner un regret. Une autre
originalité de l'article est d'aborder le concept de minimax
regret en stratégies mixtes, car il constitue une alternative à un
raisonnement de niveau-\emph{k}. Une expérience en classe a été mise en œuvre
sur la base du jeu 11-20/bonus 20 et ses résultats étudiés.

\medskip

\textbf{L'article de Liliane Bonnal, Pascal Favard et Thomas Maurice}
vise à étudier l'utilisation des aides à domicile par les personnes
âgées en Europe. On observe en France comme dans les autres pays
européens un vieillissement de la population. L'aide à domicile peut
comporter, d'une part, une composante sous forme d'une aide domestique
et, d'autre part, une composante sous forme d'une aide à la personne.
Les personnes âgées peuvent recourir à une aide formelle fournie
par des professionnels et à une aide informelle prodiguée par des proches non
rémunérés. Une des originalités de l'article est d'étudier la relation
existant entre les différentes aides à domicile en tenant compte de
leurs caractéristiques (formelle vs. informelle, à la personne vs.
domestique). Il s'agit d'une analyse économétrique réalisée à partir de
l'enquête longitudinale SHARE collectée dans vingt-six pays européens
entre octobre 2019 et mars 2020. L'analyse économétrique porte sur les
personnes âgées de plus de 65 ans vivant seules à leur domicile. Les
problèmes de santé sont évalués à partir de cinq ensembles d'information~: les limitations à la vie quotidienne (comme des difficultés à
s'habiller), les problèmes de mobilité (comme des difficultés à se
pencher), l'utilisation éventuelle d'une aide au déplacement (comme un
fauteuil roulant), les maladies chroniques, la force de préhension. Par
ailleurs, la perception qu'a l'individu de son état de santé est
précisée à travers quatre catégories de variables: le réseau social, la
santé mentale, le sentiment de solitude et les limitations d'activités
ressenties. L'analyse économétrique est basée sur différentes
spécifications ayant en commun de reposer sur trois équations~: la
première décrit le recours ou non à une aide formelle, la seconde
indique s'il y a utilisation ou non d'une aide informelle et la
dernière spécifie le nombre d'heures d'aide formelle utilisées.
Certaines spécifications ne distinguent pas la nature de l'aide
(domestique ou personnelle), d'autres portent sur une de ces deux
composantes d'aide à la personne spécifiquement. Parmi les résultats, on
note que le degré d'autonomie physique est un déterminant important à la
fois de l'aide formelle et informelle, quelle que soit sa composante
(personnelle ou domestique). L'intensité de la limitation physique est
un déterminant important de l'aide formelle. Les personnes qui reçoivent
une aide informelle ont une probabilité significativement plus faible de
recevoir une aide formelle. Recevoir un nombre d'heures d'aide formelle
élevé diminue significativement la probabilité de recevoir une aide
informelle. Un des apports du texte est de montrer que les deux types
d'aides à domicile --~formelle et informelle~-- sont plutôt substituables.

\medskip

\textbf{L'article de Marie Lassalas, Alejandro Plastina et Sergio H.
Lence} aborde le thème de l'effet de l'adoption de contrats
environnementaux sur la productivité et l'efficacité des exploitations
agricoles. Les pratiques agricoles peuvent avoir un impact sur les
émissions de gaz à effet de serre et sur la biodiversité. L'Union
européenne a adopté un triple objectif à l'horizon 2030 de réduire
l'utilisation de pesticides, de diminuer celle de fertilisants et
d'augmenter la conversion des surfaces agricoles à l'agriculture
biologique. L'Union européenne a conditionné des aides à l'agriculture à
l'adoption de contrats environnementaux. Cette étude a pour objet
d'évaluer les effets de deux instruments de la Politique agricole
commune~: les mesures agro-environnementales et climatiques (MAEC) et
l'agriculture biologique. L'engagement des agriculteurs dans le cadre de
ces contrats est volontaire et ils reçoivent en contrepartie de
l'adoption ou du maintien de pratiques plus respectueuses de
l'environnement des subventions annuelles. L'Union européenne cherche à
combiner à la fois performance environnementale et efficacité économique
dans la mesure où ses objectifs conjuguent le respect de l'environnement,
la satisfaction des besoins alimentaires et le soutien aux revenus des
agriculteurs. Les deux contrats étudiés reposent sur des engagements
différents des agriculteurs. En effet, les MAEC reposent sur un contrat
pour une durée de cinq ans alors que l'adoption de l'agriculture biologique démarre
par une période de conversion qui dure deux ans et incite les
agriculteurs à mettre en place des pratiques environnementales sur le long
terme. Cette évaluation repose sur deux techniques économétriques~: un
modèle de frontière de production stochastique et l'obtention de
l'estimateur ATT de l'effet de l'adoption de pratiques environnementales
par une approche de différence de différences. L'estimation d'un modèle
avec frontière de production stochastique s'attache aussi à dégager les
déterminants de l'inefficacité technique. L'analyse économétrique est
conduite à partir de données qui concernent un échantillon composé de fermes
produisant des cultures spécialisées dans l'ancienne région
Poitou-Charentes sur la période 2014 à 2020. Les résultats attestent que
la mise en œuvre de l'agriculture biologique a un impact négatif sur
l'efficacité des exploitations agricoles comparée au reste des
exploitations de 11,55 \% sur la période 2014-2020. Ce résultat peut être
expliqué par le fait que les agriculteurs dans ce cadre s'engagent à
restreindre l'usage des pesticides et des fertilisants. L'adoption de
MAEC, par contre, n'a pas d'effet significatif en terme d'efficacité des
exploitations.

\medskip

\looseness = -1
{\textbf{Dans le cinquième article}, Marc
Lebourges et David Masclet étudient des mécanismes d'incitation à
l'effort dans un contexte de travail en équipe en entreprise. Le travail
en équipe est fréquemment utilisé par les entreprises. Cependant, il
peut conduire à des comportements de type «passager clandestin» car il
est difficile pour quelqu'un d'extérieur à l'équipe de déterminer
l'implication de chacun des membres de celle-ci. Afin d'apporter une solution
à ce problème, des mécanismes incitatifs peuvent
être mis en place. Il peut s'agir de mécanismes décentralisés basés sur
la pression des pairs. Dans ce cas, les membres de l'équipe peuvent
observer leurs contributions respectives et se sanctionner
réciproquement. Il peut s'agir de mécanismes incitatifs centralisés sous
la forme d'un objectif de performance à obtenir pour l'équipe ou,
alternativement, d'une mise en compétition des équipes de travail dans
un contexte de tournois collectifs. Les mécanismes centralisés et
décentralisés mis en œuvre pour résoudre ce problème de passager
clandestin ont été étudiés séparément jusqu'ici. Marc Lebourges et David
Masclet étudient l'efficacité de ces mécanismes dans le cadre d'une même
expérience de laboratoire. Cette expérience comporte quatre types de
traitement. Le premier traitement consiste en un jeu répété où les
membres de l'équipe choisissent leur niveau d'effort et où la production
est partagée. Le deuxième traitement est similaire au premier mais
comporte une étape où la pression des membres de l'équipe peut
s'exercer. Un troisième traitement est similaire au premier mais
comporte un objectif de production. Dans le cadre du quatrième
traitement, les équipes sont mises en compétition. L'étude montre que
dans le premier traitement le niveau d'effort est relativement faible.
Dans le deuxième traitement, de type «pression par les pairs», le
niveau d'effort est plus important. Le troisième traitement, celui qui
comporte un objectif de performance, permet d'obtenir un niveau d'effort
élevé. Le dernier traitement, celui comportant une compétition entre
équipes de travail, arrive en second du point de vue du niveau d'effort.
La somme des profits de la firme et des gains des travailleurs est plus
importante dans le cadre des mécanismes centralisés : ceux comportant un
objectif de performance ou une compétition des équipes de travail.}

\medskip

\textbf{Le sixième et dernier article} est coécrit par Claire Mouminoux,
Morgane Plantier et Jean-Louis Rullière. L'asymétrie de l'information
est une caractéristique commune à tous les marchés (Chiappori \emph{et al.} [2006]). Cependant, cette asymétrie n'a en général pas de conséquence
dans la mesure où toute l'information pertinente est résumée par le
prix. Malheureusement, ce n'est pas le cas pour tous les marchés. En
effet, Rothschild et Stiglitz [1976] montrent que l'information privée
détenue par une des parties pour certaines transactions peut avoir un
impact sur le revenu de l'autre partie. Ainsi, dans le cadre des
contrats d'assurance, le profit de la compagnie qui vend le contrat va
dépendre de l'information relative au risque encouru par la personne qui
s'assure. De ce fait, le marché de l'assurance est propice à la présence
d'une sélection adverse. Ce marché est par suite aussi favorable à
l'existence d'un aléa moral car l'assuré peut adopter ou pas une
démarche de réduction de risque. Dans ce contexte, la littérature
économique s'est intéressée par exemple à la corrélation existant entre
la couverture du risque par le contrat et le niveau de risque. Dans un
contexte de sélection adverse et d'aléa moral on doit s'attendre à
trouver un corrélation positive entre le niveau de risque et la
couverture de ce risque (Chiappori \emph{et al.} [2006]). Dans ce cadre,
l'apport de Claire Mouminoux, Morgane Plantier et Jean-Louis Rullière
consiste a étudier les deux instruments de pilotage du risque que sont
l'assurance -- qui garantit le paiement d'une indemnité en contrepartie
de celui d'un prime -- et l'autoprotection, qui permet de réduire la
probabilité de survenue du risque. Par exemple, dans le domaine de
l'assurance habitation, une mesure d'autoprotection peut consister à
installer un extincteur incendie. La littérature économique s'est
intéressée à la relation entre le niveau de couverture (assurance) et celui
de l'autoprotection. Il n'existe pas en la matière de consensus parmi les
arguments pouvant être avancés pour justifier une corrélation négative
ou positive, ce que confirment les applications empiriques qui isolent
l'effet des caractéristiques individuelles pour étudier la relation
existant entre niveau de couverture (assurance) et niveau de risque.
Cette absence de consensus est retrouvée aussi dans la littérature
économétrique qui étudie directement la relation entre assurance et
autoprotection. L'originalité de l'article est d'utiliser une
expérience de laboratoire pour analyser l'impact d'une option
d'autoprotection sur le choix d'assurance. Un autre apport de l'article
consiste à vérifier si les individus qui ont fait le choix de l'autoprotection modifient \emph{ex post} leur effort en matière de réduction de la
probabilité de survenue du risque.

\clearpage
\nocite{*}
\printbibliography
\end{refsection}

% Arad A., Rubinstein A. {[}2012{]}, «The 11-20 money request game: a
% level-k reasoning study», The American economic review, 102, N°7,
% 3561-3573.

% Chiappori P.-A., Julien B., Salanié B. et F. Salanié {[}2006{]},
% «Asymmetric Information in Insurance: General Testable Implications»,
% RAND Journal of Economics, 37, N°4, 783-798.

% Keynes John M. {[}1936{]}, Théorie Générale de l'Emploi, de l'Intérêt et
% de la Monnaie, traduit par Jean de Largentaye, Payot, Paris.

% Nagel R. {[}1995{]}, «Unraveling in Guessing Games: An Experimental
% Study», The American Economic Review, vol. 85, n°5, 1313-1326

% Rothschild M. et Stiglitz J., «Equilibrium in Competitive Insurance
% Markets: An Essay on the Economics of Imperfect Information», Quarterly
% Journal of Economics, 90, 629-649

% Schimpl-Neimanns B. {[}2003{]}, «Umsetzung der Berufsklassifikation von
% Blossfeld auf die Mikrozensen 1973-1998», ZUMA-Methodenbericht, 2003/10.

\end{Article}
