\begin{Article}[Titre=Aides formelles et informelles pour les seniors : substituabilité ou complémentarité ?, Auteur={Liliane Bonnal\thanks{Université de Poitiers, LÉP; Université de Toulouse 1 Capitole, TSE-R}, Pascal Favard\thanks{Université de Tours, IRJI François-Rabelais. \textit{Correspondance :} pascal.favard@univ-tours.fr}, Thomas Maurice\thanks{Université de Poitiers, LÉP}}]

\label{Bonnaletal}

\selectlanguage{french}

\begin{refsection}[Bonnal]
    
\begin{resume}
\`A partir des données de la vague 8 de l'enquête SHARE (Survey on Health, Ageing and Retirement) nous étudions l’utilisation des aides à domicile par les personne âgées en Europe. Ces aides peuvent être utilisées par une personne âgée parce qu’elle est dans l’incapacité d'accomplir certaines tâches domestiques ou personnelles. Nous estimons simultanément les probabilités de recevoir de l’aide formelle et informelle et le nombre d’heures d’aide formelle reçues, en tenant compte des interactions réciproques entre les aides. Nous montrons que la relation est très significativement négative mais cela n'est pas suffisant pour en conclure que l'aide formelle est un substitut de l'aide informelle puisque ces aides sont une agrégation de services rendus hétérogènes. Nous conduisons une étude différenciée suivant que l’aide est domestique ou personnelle. En considérant des services plus homogènes la substituabilité est très significativement établie.
\end{resume}

\motscles{aide formelle, aide informelle, aide domestique, aide personnelle, état de santé, séniors}

\titrearticleENG{Formal and Informal Care for Elders~: Substitutability or Complementarity?}
    
\begin{resumeENG}
From the data of the 8th wave of the SHARE (Survey on Health, Ageing and Retirement) survey, we investigate the use of home care by elderly individuals in Europe. These aids could be utilized by an elderly person because they are unable to perform certain household or personal tasks. We simultaneously estimate the probabilities of receiving formal and informal assistance and the number of hours of formal assistance received, taking into account the reciprocal interactions between these forms of help. We show that the relationship is significantly negative, but it is not sufficient to conclude that formal care is a substitute for informal care since these aids represent an aggregation of heterogeneous services. We conduct a differentiated study based on whether the assistance is domestic or personal. By considering more homogeneous services, the substitutability is strongly established.
\end{resumeENG}

\keywords{formal care, informal care, domestic care, personal care, health, elders}
\jelcode{C35, I1, J14}

%%%%%%%%%%%%%%%%%%%%%%%%%%%%%%%%%%%%%%%%%%%%%%%%%%%%%
							\section{Introduction}
%%%%%%%%%%%%%%%%%%%%%%%%%%%%%%%%%%%%%%%%%%%%%%%%%%%%%
Dans tous les pays européens, on observe un vieillissement de la population et une augmentation des problèmes de santé et de handicap chez les 50-65 ans. En France, par exemple, les démographes prévoient qu’en 2060, une personne sur trois sera âgée de 60 ans ou plus, contre une personne sur cinq en 2005. De plus, des études montrent que l’espérance de vie sans incapacité connaît depuis quelques années une stagnation \textcite{MOISY2018} voire une diminution \parencite{ROBINE2017}. Ce constat est identique au Japon \parencite{HAN2014,SHIMADA2013} mais aussi aux Etats-Unis où l’incapacité chez les 50-70 ans a augmenté \parencite{MARTIN2009,REYNOLDS2010,SEEMAN2010}. À ces problèmes de santé et d’incapacité se rajoute celui de la dépendance. Le risque de dépendance ou de perte d’autonomie correspond à un état de santé d’une personne qui nécessite le recours à un tiers pour accomplir les actes simples de la vie quotidienne comme se déplacer dans une pièce, s'habiller, se nourrir, se laver, de manière autonome, etc. (\parencite{DUPUY2016}). Ce tiers peut fournir de l’aide formelle (donnée par des professionnels rémunérés) ou informelle (donnée par des proches non rémunérés), domestique ou liée aux soins du corps. Pour un même individu ces différentes aides peuvent être reçues ou non. A priori, l’aide (prise au sens large, quelle que soit sa forme) sera d’autant plus forte que la dépendance est sévère et donc que l’incapacité d’accomplir les activités de la vie quotidienne est élevée.

Qu’elle soit formelle ou informelle, l'aide a un coût. Parce qu'elle est payante, l’aide formelle a un coût monétaire supporté de manière plus ou moins forte par l’aidé. L’aide informelle quant à elle, a plutôt un coût pour les aidants. Ces coûts peuvent être directement quantifiables mais, il existe aussi des coûts d’opportunité difficilement quantifiables et sans doute non négligeables (temps investi par l'aidant, impacts sur sa vie professionnelle et personnelle). Ces coûts peuvent être monétaires à travers la participation au marché du travail par exemple \parencite{FONTAINE2010,SOULLIER2011,VANHOUTVEN2013} ; psychologiques avec des risques de dépression ou d’anxiété \parencite{MAJEROVITZ1995}. Mais, ils peuvent aussi être mesurés à travers une dégradation de la santé physique occasionnant des maladies chroniques comme l’hypertension, les maladies cardiovasculaires ou encore des problèmes de dos \parencite{GRANT2002,MAUSBACH2007} ; ou de la santé mentale \parencite{COLOMBO2011,LEIGH2010}. L'aidé peut aussi avoir des réticences à faire appel à son entourage pour ne pas faire peser des contraintes sur celui-ci. De façon générale ces aides ont aussi à la fois un coût direct et un coût indirect pour la société.

Bien comprendre le recours aux aides (formelle et informelle) et la relation qui existe entre l’aide formelle et l’aide informelle est important pour avoir une meilleure idée de la façon d’appréhender et de quantifier ces coûts.

Un certain nombre d’études s'est intéressé à l’impact de l’aide informelle sur l’aide formelle à partir de modèles à variables instrumentales. Les instruments les plus souvent utilisés sont les caractéristiques des enfants tels que le nombre ou la proportion de filles, la distance (en kilomètres) entre le domicile de l'aidé et celui de l'aidant le plus proche, les caractéristiques de l’aîné, etc. (\parencite{VANHOUTVEN2004,BOLIN2008,BONSANG2009}). Pour les \'États-Unis, en utilisant le nombre d’enfants et le fait que l’ainé soit une fille, \textcite{VANHOUTVEN2004} trouvent un impact négatif de l’aide informelle sur la probabilité de recevoir de l'aide formelle. En considérant les mêmes instruments \textcite{BOLIN2008}, à partir des données européennes de la première vague de l’enquête SHARE, mettent en évidence la même relation. Toutefois, ils montrent que l’aide informelle augmente la probabilité de recevoir de l’aide formelle sous la forme de soins médicaux effectués par des professionnels de santé. Encore à partir de l’enquête SHARE, \textcite{BONSANG2009} s’est intéressé pour l'aide formelle à la distinction entre aide domestique et aide à la personne. Il montre que l’aide informelle augmente la probabilité d’avoir de l’aide formelle à la personne, en particulier lorsque l’aidé souffre de plusieurs limitations mais elle n'a pas d'effet sur l'aide formelle domestique.

Un autre ensemble d’études a mesuré l’effet de l’aide formelle sur l’aide informelle en instrumentant cette fois l’aide formelle. En utilisant le reste à charge pour les soins formels comme instrument, \textcite{ARNAULT2017} montrent, sur des données françaises, que la quantité d’aide informelle décroît lorsque la quantité d’aide médicale formelle augmente et lorsque le prix de l’aide médicale formelle diminue. \textcite{ROQUEBERT2017}, à partir d’une base de données différentes mais toujours pour la France, obtiennent des résultats très similaires. \textcite{CARRINO2018} ont quant à eux travaillé sur les vagues~1 et 2 de SHARE pour quatre pays~: l’Allemagne, l’Autriche, la Belgique et la France. Avec comme variable instrumentale le fait d’être éligible à un programme d’aide médicale, ils mettent en évidence que l’utilisation de l'aide formelle et la quantité d’heures reçues augmentent le recours à l'aide informelle. Enfin, pour la France, en considérant le montant des aides départementales, \textcite{PERDRIX2021} montrent qu’un accroissement de l’aide formelle va entrainer une baisse de l’aide informelle. Pour la Chine, \textcite{LIU2021} met en évidence que l’aide formelle a un impact négatif sur l’aide informelle lorsqu’il s’agit de soins médicaux (hospitalisation par exemple) mais de manière générale, l’aide informelle à tendance à varier dans le même sens que l’aide formelle, en particulier dans les familles à bas revenus, ou vivant dans des quartiers avec un développement économique faible et des conditions de vie dégradées, pour lesquelles, le peu d’aide formelle utilisée vient accroître significativement l’aide informelle surtout quand la famille n'est pas une famille nombreuse.

Peu de travaux, à notre connaissance, ont tenu compte de la double causalité. \textcite{GEERTS2012} se sont intéressés à des processus de transitions, entre différentes aides, sur deux périodes de temps (les vagues~1 et 2 de SHARE). Les aides étant considérées comme complémentaires si elles sont observées en même temps et substituables si l’on observe une seule des deux aides ou si la personne change d’aide entre deux périodes. La probabilité de recevoir les deux types d’aides augmentent entre les deux dates. À partir de données françaises, \textcite{BARNAY2016} estiment un modèle bivarié d’aide formelle (avec nombre d'heures) et informelle sans mesurer directement l’effet d’une des aides sur l’autre. La corrélation entre les deux aides est négative : les variables omises du modèle agiraient en sens inverse sur les aides. Enfin, \textcite{BALIA2014} ont travaillé sur la vague~1 de SHARE et pour l’ensemble de pays enquêtés. En estimant un modèle tenant compte à la fois des aides (formelle et informelle) et du nombre d’heures d’aide dans chaque catégorie, ils montrent trois relations~: l’aide informelle varie i) en sens inverse de l’aide formelle ; ii) en sens inverse de l’aide formelle associée aux tâches ménagères ; iii) dans le même sens que l’aide formelle associée aux soins du corps. Ils concluent donc que l'aide informelle et l'aide formelle associées aux tâches ménagères sont substituables et que l'aide informelle et l'aide formelle associées aux soins du corps sont complémentaires. Les effets estimés trouvés sont en valeurs absolues très faibles (moins d'une heure par mois). Toutefois, l'aide informelle n'ayant pu être décomposée en aide associée aux tâches domestiques et aide associée aux soins du corps, il est difficile de parler de substituabilité ou de complémentarité, les services n'étant pas homogènes.
Comment pourrait-on expliquer cette hétérogénéité des résultats~? Bien que les études ne concernent pas forcément les mêmes pays, les bases de données utilisées sont relativement homogènes puisqu’elles sont majoritairement constituées de personnes seules de 65 ans. Les principaux points de divergences portent sur les méthodes (prise en compte ou non de l’endogénéité des aides), la périodicité de ces aides (annuelle, mensuelle, au cours des dernières semaines) et sur les définitions même de de ces aides. Selon les études l’aide formelle peut ou non distinguer domestique et personnelle. Cette distinction est très peu souvent faite pour l’aide informelle. Nous pouvons rajouter à ces éléments l’utilisation des notions d’aides substituables et complémentaires. Parler de substituabilité ou de complémentarité des aides revient économiquement à supposer que les aides sont des facteurs de production permettant de produire un service (tâches domestiques, soins personnels donné à l’aidé, etc.) homogène. Par conséquent, les aides domestiques et personnelles ne sont pas des facteurs dans la production d'un même service. En revanche, l’aide formelle domestique et l’aide informelle domestique sont deux facteurs qui sont fortement substituables parce qu’elles sont associées à un même service rendu, l'accomplissement des tâches ménagères. Il en va de même pour les aides formelle et informelle personnelles associées à un même service, les soins du corps. Toutefois, même pour un service homogène, observer une utilisation conjointe de l’aide formelle et informelle ne permet pas de dire que ces aides sont complémentaires. Les aidants informels peuvent avoir des disponibilités limitées qui font que l’aidé reçoit aussi de l’aide formelle. De façon symétrique l'aide formelle peut être disponible en quantité limitée. Mais aussi, la contrainte budgétaire de l’aidé peut ne pas lui permettre de financer de façon formelle la totalité de l’aide dont il a besoin. Dans ces cas, l’aidé fait face à un problème de prix relatif non constant par rapport aux quantités d'aides utilisées. Dans ce travail, pour tenir compte des remarques précédentes nous étudions séparément les aides domestiques et les aides personnelles. Dans la lignée de \textcite{BALIA2014} et de \textcite{BARNAY2016}, l’objectif de ce travail de recherche est donc préciser la relation qui existe entre les aides en mettant en évidence un lien direct entre ces aides. Ces effets directs vont nous permettre de vérifier si c'est l'effet technique (substituabilité) qui prédomine. Une description de la base de données, des variables utilisées et du modèle estimé sera faite dans la section 2. Les résultats des estimations seront donnés dans la section 3 et la section 4 conclut ce travail.

%A priori, d'un point de vue productif, l'aide formelle domestique et l'aide informelle domestique sont des inputs fortement substituables. Il en va de même pour l'aide personnelle. Observer une utilisation conjointe de ces inputs peut être expliqué par des prix non linéaires ou des disponibilités limitées. Dans ce cas, dire que ces inputs sont complémentaire serait impropre. Même si dans la littérature appliquée l'analyse en terme de production n'est pas toujours aussi explicite c'est, à notre connaissance, le point de vue qui est retenu.



%%%%%%%%%%%%%%%%%%%%%%%%%%%%%%%%%%%%%%%%%%%%%%%%%%%%%
\section{Données, variables et méthode}
%%%%%%%%%%%%%%%%%%%%%%%%%%%%%%%%%%%%%%%%%%%%%%%%%%%%%

%%%%%%%%%%%%%%%%%%%%%%%%%%%%%%%%%%%%%%%%%%%%%%%%%%%%%
\subsection{Données et échantillon}
%%%%%%%%%%%%%%%%%%%%%%%%%%%%%%%%%%%%%%%%%%%%%%%%%%%%%

Les données utilisées sont celles de l'enquête SHARE \footnote{Pour plus de détails, voir \textcite{SHARE8}, \textcite{BORSCHSUPAN2013}, \textcite{BERGMANN2021}.}. Cette enquête, Survey on Health Ageing and Retirement in Europe, longitudinale multidisciplinaire et internationale porte sur plus de 80 000 européens âgés de 50 ans et plus. L’enquête est réalisée tous les deux ans depuis 2004. La vague 8 a été réalisée entre octobre 2019 et mars 2020 dans vingt-six pays européens (avec une harmonisation des questionnaires). Pour cette vague 46 498 personnes ont été interrogées. Parmi elles, 32 780 ont 65 ans et plus. Le fait de vivre seul ramène l’échantillon à 9 263 personnes. Nous avons écarté de cet échantillon 464 personnes n’ayant pas répondu aux questions concernant les aides (formelles et/ou informelles) ou les variables associées aux ADL\footnote{Activities of Daily Living.}. Cela supprime de fait les personnes en incapacité de répondre (nécessitant un proxy) qui sont au nombre de 187. Par conséquent, l’échantillon utilisé compte 8799 personnes vivant seules dans leur logement. L’enquête contient les réponses à de nombreuses questions sur les caractéristiques des ménages, le niveau de vie, le soutien social, la santé physique et mentale notamment des enquêtés. Nous considérons seulement les personnes de plus de 65 ans vivant seules à leur domicile. Le choix de cette population se justifie pour deux raisons. D'une part, les 65 ans et plus sont pour la plupart retraités et ils ont plus de chance que les plus jeunes d'avoir des problèmes de santé ou de dépendance. D'autre part, le rôle d'aidant d'un conjoint étant difficile à mesurer, nous avons fait le choix (comme dans \textcite{BALIA2014}) de ne travailler que sur les personnes vivant seules.

%%%%%%%%%%%%%%%%%%%%%%%%%%%%%%%%%%%%%%%%%%%%%%%%%%%%%
\subsection{Variables}
%%%%%%%%%%%%%%%%%%%%%%%%%%%%%%%%%%%%%%%%%%%%%%%%%%%%%

\subsubsection{Variables dépendantes}

\noindent Les variables d’aide sont définies à partir des informations obtenues dans la vague 8 uniquement. Quatre indicatrices d’aide ont été créées (une définition précise de ces aides est donnée dans l’annexe \ref{Construction des variables d’aide formelle informelle})~:
\begin{itemize}
\item  aide domestique formelle (ADF) qui prend la valeur 1 lorsque la personne déclare recevoir de l’aide liée aux tâches domestiques (ménage, repassage, cuisine, bricolage, jardinage, transport, courses) par un professionnel ;
\item  aide à la personne formelle (APF) qui prend la valeur 1 lorsque la personne déclare recevoir de l’aide liée aux soins personnels (s'habiller, \linebreak prendre un bain ou une douche, manger, se coucher ou se lever du lit, utiliser des toilettes) par un professionnel. L’aide prise en compte ne concerne pas les soins médicaux, accomplis généralement par des professionnels de santé qualifiés et donc ne pouvant être fournis par l’entourage ;
\item  aide domestique informelle (ADI) qui prend la valeur 1 lorsque la personne déclare recevoir de l’aide liée aux tâches domestiques de son entourage ;
\item  aide à la personne informelle (API) qui prend la valeur 1 lorsque la personne déclare recevoir de l’aide liée aux soins personnels de son entourage.
\end{itemize}

Les deux indicatrices d'aide informelle, ADI et API, ne tiennent compte que des services reçus au moins une fois par semaine\footnote{\textcite{BALIA2014} prennent en compte tout l'aide reçue même si celle-ci n'est pas hebdomadaire.}. Notre objectif est de ce focaliser sur les aides qui sont \enquote{indispensables} à la personne âgée pour continuer de vivre à domicile. Si une personne âgée reçoit de l'aide chaque semaine on peut penser que de ne pas recevoir cette aide la conduirait à quitter son domicile. Des aides moins récurrentes font partie d'un système d'échange de services intrafamiliaux ou intracommunautaires.

À partir de ces quatre indicatrices deux autres indicatrices sont créées. L'aide formelle (AF) prend la valeur 1 si ADF ou APF a pour valeur 1 et l'aide informelle\footnote{Contrairement à la majorité des études tenant compte de  l’aide informelle, nous ne considérons pas l’ensemble des aides déclarées mais seulement l’aide informelle reçue au moins une fois par semaine, c’est à dire régulière. Nous comptons dans l'échantillon 3115 personnes déclarant de l'aide informelle, soit 34 \% de l'échantillon. Ce pourcentage passe à 18,6 lorsque l'on restreint l'aide informelle à de l'aide hebdomadaire.} (AI) prend la valeur 1 si ADI ou API a pour valeur 1.
Le nombre d'heures d'aide formelle (hAF) reçues par semaine est connu et décomposable suivant qu'il s'agit d'heures affectées au tâches domestiques (hADF) ou aux soins personnels (hAPF). Aucune information sur le nombre d'heures associé à l'aide informelle n'est disponible.


\subsubsection{Variables indépendantes}

\noindent Des indicatrices de pays, pouvant être vues comme des effets fixes, sont crées pour tenir compte de l'hétérogénéité en termes de politiques de santé mais aussi de cultures \parencite{REHER1998}.\\

\noindent {\bf Variables socio-économiques}


%\vspace{-0.5cm}
\begin{itemize}
\item  L'âge (en continu), le genre (indicatrice), le nombre d'enfants (en continu).
\item  Le niveau d'éducation (indicatrices associées aux modalités : primaire, collège, lycée, supérieur, non réponse), et le niveau de revenu (normalisé en parité de pouvoir d'achat ; utilisé en continu et en logarithme).
\item  La zone d'habitation (indicatrices associées aux modalités : très grande ville, banlieue, grande ville, petite ville, zone rurale, non réponse).
\end{itemize}

\noindent {\bf Variables de santé}\\

\noindent Pour la vague en cours (vague 8) nous mesurons le degré de dépendance à partir de cinq types d'information : les limitations de la personne dans la vie quotidienne (ADL) ; la mobilité ; l'aide aux déplacements, les maladies chroniques et la force de préhension.
\begin{itemize}
\item  Les limitations de la vie quotidienne (ADL) sont définies à partir de la question suivante : \frquote{Avez-vous des difficultés à réaliser ces activités : s'habiller (y compris chaussures et chaussettes), traverser une pièce, prendre un bain ou une douche, manger, entrer ou sortir du lit, utiliser des toilettes ?}. Six ADL maximum peuvent être déclarées. Suite à une analyse factorielle en composantes multiples\footnote{L'analyse factorielle a été réalisée avec le logiciel R.} trois indicatrices ont été créées : la personne a au moins déclaré avoir des difficultés à réaliser elle-même sa toilette (bain ou douche) ; la personne a déclaré au moins une autre difficulté (mais pas pour la toilette) ; aucun problème n'a été déclaré.
\item  Trois indicatrices de mobilité ont été considérées : la personne a  des problèmes de mobilité des membres inférieurs si elle a déclaré avoir des difficultés à se pencher, s’agenouiller
ou s’accroupir ; la personne a des problèmes de mobilité des membres inférieurs et supérieurs  
si, en plus d’avoir des problèmes de mobilité du bas du corps, elle déclare avoir des
difficultés à lever ou étendre les bras au-dessus du niveau de l’épaule ou encore à saisir
une petite pièce de monnaie posée sur une table ; la personne a d'autres problèmes de mobilité ou n'a pas déclaré de problème particulier. 
\item  L'aide aux déplacements est définie à partir des outils d'aide à la marche utilisés par la personne. Une indicatrice a été définie : l'utilisation ou non d'un fauteuil roulant ou d'un déambulateur. 
\item  Les maladies chroniques ont été regroupées comme le suggèrent \textcite{KALWIJ2008} : les maladies graves et les autres maladies. Ont été considérées comme graves, susceptibles d'empêcher l'exécution des tâches de la vie quotidiennes, les maladies suivantes : arthrose, polyarthrite rhumatoïde, autres rhumatismes, accident ou maladie vasculaire cérébral, syndrome cérébral, sénilité ou tout autre trouble grave de la mémoire, troubles affectifs ou émotionnels y compris l'anxiété, problèmes nerveux ou psychiatriques, la maladie de Parkinson, maladie d'Alzheimer, démence. Trois indicatrices de maladie chronique ont été créées  : les maladies chroniques graves (si au moins une des maladies citées est déclarée)\footnote{Plus de 90 \% des personnes ayant une maladie chronique grave déclarent avoir des rhumatismes ou de l’arthrose.} ; les maladies chroniques légères (du type hypertension, hypercholestérolémie, diabète, asthme, problèmes d'estomac, etc.) et aucune maladie chronique déclarée.
\item  La force de préhension est une indicatrice qui prend la valeur 1 lorsque le niveau de force est inférieur à 27 pour les hommes et 16 pour les femmes, voir \textcite{dodds2016}.
\end{itemize}
%\item  Dans le futur : nous  repérons dans la vague 7, c'est-à-dire deux ans après l'enquête, le lieu d'habitation de la personne. Cinq situations sont possibles : décès, toujours à son domicile, chez un proche, en institution, non présent dans la base. Cette variable mesure, comme dans \textcite{BALIA2014} une certaine proximité avec la mort.\\
%\end{itemize}



\noindent {\bf Variables de comportement et de perception}\\

\noindent Ces variables caractérisent la vie sociale des personnes et leur ressenti (y compris par rapport à leur santé).

\begin{itemize}
\item  Le réseau social est mesuré à partir du nombre de personnes de confiance déclaré. Trois indicatrices ont été créées : aucune, moins de quatre, quatre et plus. Le nombre de \enquote{confidents} a été déterminé en testant plusieurs indicatrices.
\item  La santé mentale est mesurée par une indicatrice prenant la valeur 1 lorsque l'indice EuroD est supérieur à 4 (voir \parencite{PRINCE1999}).
\item  Le sentiment de solitude est mesuré à l'aide de trois indicatrices : sentiment ressenti souvent, parfois ou jamais.
\item  La limitation ressentie (GALI\footnote{Global Activity Limitation Indicator.}) est mesurée par trois indicatrices basées sur la réponse à la question : \enquote{Dans quelle mesure êtes-vous limité dans vos activités normales par des problèmes de santé qui durent depuis au moins 6 mois~?} Trois réponses sont possibles~: fortement limité, limité mais pas fortement, pas limité.
\end{itemize}


%%%%%%%%%%%%%%%%%%%%%%%%%%%%%%%%%%%%%%%%%%%%%%%%%%%%%
\subsection{Modélisation}
%%%%%%%%%%%%%%%%%%%%%%%%%%%%%%%%%%%%%%%%%%%%%%%%%%%%%


Lorsque les aides formelles et informelles sont mesurées de façon dichotomique (aide ou pas), il n’est techniquement pas possible de mesurer simultanément l’effet réciproque d'un type d'aide sur l'autre par des modèles standards de type probit bivarié. Cela explique pourquoi la double causalité entre l'aide formelle et l'aide informelle a très peu été étudiée. À notre connaissance, seule\footnote{On peut aussi citer \textcite{Greene1983} même si c'est dans un contexte très particulier. Les personnes bénéficient d’un « Community Services System ». L’objectif de ce programme est de fournir des services à domicile aux personnes fragiles (80 \% ont plus de 60 ans) pour éviter ou retarder l’entrée en maison pour personnes âgées. L’accès à l’aide formelle est donc facilité.} la modélisation de \textcite{BALIA2014} tient compte simultanément des deux types d'aide à partir d'un modèle à quatre équations : deux associées à la probabilité de recevoir de l'aide (formelle et informelle) et deux autres associées aux heures d'aides reçues par mois. Plus précisément, le nombre d'heures d'aide informelle reçues est supposé avoir un effet sur la probabilité de recevoir de l'aide formelle et sur la quantité d'heures d'aide formelle reçues et inversement, le nombre d'heures d'aide formelle est supposé avoir un effet sur la probabilité de recevoir de l'aide informelle et sur la quantité d'heures d'aide informelle reçues\footnote{Cette quantité est disponible dans la vague~1 mais plus dans la vague~8 de SHARE. Toutefois, pour ces heures, les données de la vague~1 ne permettent pas de distinguer les heures informelles suivant qu'elles sont domestiques ou personnelles. La vague~8 a été préférée car elle est plus récente et contient des informations sur des variables psychologiques.}. Ce modèle aurait pu être estimé à partir d’une loi normale multivariée à 4 équations avec variables censurées mais cette modélisation est un peu complexe à mettre en place. L’alternative choisie est une modélisation à variables latentes discrètes qui permet de rendre dépendantes entres elles les équations. L’erreur de mesure associée à chaque équation est composite et additive~: une partie spécifique et une partie commune associée au facteur d’hétérogénéité inobservée. Cette dernière dépend de 2 variables aléatoires discrètes indépendantes supposées suivre une loi de Bernoulli. Cela revient à supposer que la partie inobservée de chaque équation est distribuée selon une loi discrète à 4 points de support (définis par les couples possibles des valeurs prises par les 2 variables aléatoires). Ce type de modélisation semi-paramétrique peut être estimé par maximum de vraisemblance et permet d’obtenir empiriquement des résultats comparables à ceux obtenus en supposant un facteur d’hétérogénéité inobservé commun distribué selon une loi normale \parencite{bonnal1997}. Dans ce travail, nous avons donc choisi de retenir ce type de modélisation. Trois équations sont estimées~: la probabilité de recevoir de l'aide formelle ; le nombre d'heures d'aide formelle utilisé et la probabilité de recevoir de l'aide informelle. Le fait de recevoir de l'aide informelle est supposé avoir un effet sur la probabilité de recevoir de l'aide formelle et sur la quantité d'heures d'aide formelle reçue tandis que cette quantité d'heures est supposée avoir un effet sur la probabilité de recevoir de l'aide informelle. La fréquence des aides est hebdomadaire. Les deux probabilités de recevoir de l'aide sont estimées à l'aide d'un modèle Probit. Le nombre d'heures d'aide est supposé suivre une loi gamma. Enfin, la prise en compte de facteurs d'hétérogénéité inobservée permet d'estimer une interaction réciproque entre les aides formelle et informelle.

Les aides formelle et informelle sont respectivement définies par deux indicatrices $Y_{l}$ et $Y_{k}$\footnote{Dans un souci de simplification des notations, l'indice $j$ de la personne âgée a été omis.} :
\begin{equation}\label{prob_aideFormelle}
	Y_{l} = 
	\begin{cases}
		1 \text{ si } Y_{l}^* = X_{l}\beta_{l} + \gamma_{l}Y_{k} + \rho_{l}u + \delta_{l}v + \varepsilon_{l} \geq 0, \vspace{.3cm}\\
		0 \text{ sinon,}
	\end{cases}
\end{equation}
\begin{equation}\label{prob_aideInformelle}
	Y_{k} = 
	\begin{cases}
		1 \text{ si } Y_{k}^* = X_{k}\beta_{k} + \gamma_{k}Y_{hl} + \rho_{k}u + \delta_{k}v + \varepsilon_{k} \geq 0, \vspace{.3cm}\\
		0 \text{ sinon.}
	\end{cases}
\end{equation}

$Y_{hl}$ caractérise, quant à elle, le nombre d'heures d'aide formelle utilisées~: 
\begin{equation}\label{heures_aideFormelle}
	Y_{hl} = 
	\begin{cases}
		Y_{hl}^* = X_{hl}\beta_{hl} + \gamma_{hl}Y_{k} + \rho_{hl}u + \delta_{hl}v + \varepsilon_{hl} \text{ si } Y_{l} = 1, \vspace{.3cm}\\
		0 \text{ si } Y_{l} = 0.
	\end{cases}
\end{equation}

\noindent Pour les relations \eqref{prob_aideFormelle}, \eqref{prob_aideInformelle}, \eqref{heures_aideFormelle}, $Y_{b}^*$ est la variable latente associée à $Y_{b}$, $b=(l,k,hl)$ avec  $(l,k)=\{(AF,AI);(ADF,ADI);(APF,API)\}$. Cette variable latente dépend des caractéristiques individuelles observées $X_{b}$, de $\beta_{b}$ le vecteur de paramètres à estimer associé à ces caractéristiques et d'un terme d'erreur composite, défini par $\tau_b + \varepsilon_b=\rho_bu + \delta_bv + \varepsilon_b$, où $\varepsilon_b$ est un terme d'erreur idiosyncratique. Les paramètres $\gamma_{l}$ et $\gamma_{hl}$ mesurent l'effet de l'aide informelle reçue sur respectivement la probabilité de recevoir de l'aide formelle et sur la quantité d'heures d'aide formelle. Le paramètre $\gamma_{k}$ mesure, quant à lui, l'effet du nombre d'heures d'aide formelle sur la probabilité de recevoir de l'aide informelle. Concernant l’erreur composite, les facteurs latents $\tau_b$ caractérisent l’hétérogénéité inobservée et sont définis par deux variables aléatoires additives $u$ et $v$. Les $\varepsilon_b$ sont supposés être indépendants entre eux et indépendants des covariables, des variables à expliquer et des facteurs latents. Les variables $u$ et $v$ sont distribuées selon une loi de Bernoulli, elles prennent la valeur $1$ avec une probabilité $p_u$ et $p_v$ et la valeur $0$ avec une probabilité $(1-p_u)$ et $(1-p_v)$\footnote{Les points de support des deux variables aléatoires sont fixés à $0$ et $1$. Cette normalisation est nécessaire afin d'identifier les constantes des différentes équations.}. Ces probabilités sont définies par $p(u=1)=p_u=\frac{\exp{(\theta_u)}}{1+\exp{(\theta_u)}}$ et $p(v=1)=p_v=\frac{\exp{(\theta_v)}}{1+\exp{(\theta_v)}}$, $\theta_u$ et $\theta_v$ ainsi que les couples $(\rho_b,\delta_b)$ sont des paramètres à estimer.


Le modèle va être estimé par maximum de vraisemblance. Pour une personne donnée, quatre contributions à la vraisemblance sont possibles~:
\begin{enumerate}
	\item la personne ne reçoit aucune aide : $(Y_{l} = 0, Y_{hl} = 0, Y_{k} = 0)$
	\begin{align*}
		\Pr(Y_{l}&=0|X_{l},Y_{k},u,v) \times \Pr(Y_{k}=0|X_{k},Y_{hl},u,v)\\
		& = [1 - \Phi(X_{l}\beta_{l} + \rho_{l}u + \delta_{l}v)] \times [1 - \Phi(X_{k}\beta_{k} + \rho_{k}u + \delta_{k}v)]~;
	\end{align*}
	\item la personne reçoit de l'aide formelle seulement : $(Y_{l} = 1, Y_{hl} > 0, Y_{k} = 0)$
	\begin{align*}
		\Pr(Y_{l} &=1|X_{l},Y_{k},u,v) \times f(Y_{hl}|X_{hl},Y_{k},u,v) \times \Pr(Y_{k}=0|X_{k},Y_{hl},u,v) \\
		& = \Phi(X_{l}\beta_{l} +\rho_{l}u + \delta_{l}v) \times f(X_{hl}\beta_{hl} + \rho_{hl}u + \delta_{hl}v) \times \\
		&\ \ \ \ [1 - \Phi(X_{k}\beta_{k} + \gamma_{k}Y_{hl} + \rho_{k}u + \delta_{k}v)]~;
	\end{align*}
	\item la personne reçoit de l'aide informelle seulement : $(Y_{l} = 0, Y_{hl} = 0, Y_{k} = 1)$
	\begin{align*}
		\Pr(Y_{l}&=0|X_{l},Y_{k},u,v) \times \Pr(Y_{k}=1|X_{k},Y_{hl},u,v) \\
		& = [1 - \Phi(X_{l}\beta_{l} +\gamma_{l}Y_{k}+ \rho_{l}u + \delta_{l}v)]  \times \Phi(X_{k}\beta_{k} +  \rho_{k}u + \delta_{k}v)~;
	\end{align*}
	\item la personne reçoit les deux aides simultanément : $(Y_{l} = 1, Y_{hl} > 0, Y_{k} = 1)$
	\begin{align*}
	\Pr(Y_{l} &=1|X_{l},Y_{k},u,v) \times f(Y_{hl}|X_{hl},Y_{k},u,v) \times \Pr(Y_{k}=1|X_{k},Y_{hl},u,v) \\
		&= \Phi(X_{l}\beta_{l} + \gamma_{l}Y_{k} +  \rho_{l}u + \delta_{l}v) \times f(X_{hl}\beta_{hl} + \gamma_{hl}Y_{k} +  \rho_{hl}u + \delta_{hl}v) \times \\
		&\ \ \ \ \Phi(X_{k}\beta_{k} + \gamma_{k}Y_{hl} +  \rho_{k}u + \delta_{k}v).
	\end{align*}
\end{enumerate}
$\Phi$ est la fonction de répartition d'une loi normale centrée réduite. $f$ est la fonction de densité de la loi gamma définie par $f(y_{hl}|X_{hl},Y_k,u,v)=\dfrac{y_{hl}^{(\alpha_{hl}-1)}\exp{\left(-\dfrac{y_{hl}}{\sigma_{hl}}\right)}}{\sigma_{hl}^{\alpha_{hl}}\Gamma(\alpha_{hl})}$ avec $\alpha_{hl}=\exp{(\xi_{hl})}$ le paramètre de forme, $\xi_{hl}$ le paramètre à estimer associé et $\sigma_{hl}=\exp{(X_{hl}\beta_{hl}+\gamma_{hl}Y_{k}+\rho_{hl}u+\delta_{hl}v)}$ le paramètre d’échelle.

La contribution à la vraisemblance de la personne $j$  s'écrit de la manière suivante :
\begin{align*}
    L(\beta_l,\beta_k,&\beta_{hl},\gamma_l,\gamma_k,\gamma_{hl},\rho_l,\rho_k,\rho_{hl},\delta_l,\delta_k,\delta_{hl},\theta_u,\theta_v,\xi_{hl})=\\
	&\sum_{u=0}^1\sum_{v=0}^1p_u^u(1-p_u)^{1-u} p_v^v(1-p_v)^{1-v} \\
	& \times\left[ \left(1 - \Phi(X_{l}\beta_{l} + \rho_{l}u + \delta_{l}v) \right) \right. \\
    & \left. \times \left(1 - \Phi(X_{k}\beta_{k} + \rho_{k}u + \delta_{k}v) \right)\right]^{(1-Y_{l})\times(1-Y_{k})}\\
	&\times \left[\Phi(X_{l}\beta_{l} + \rho_{l}u + \delta_{l}v) \times f(X_{hl}\beta_{hl} + \rho_{hl}u + \delta_{hl}v) \right. \\
	 &\times\left. \left(1 - \Phi(X_{k}\beta_{k} + \gamma_{k}Y_{hl} +\rho_{k}u + \delta_{k}v) \right)\right]^{Y_{l}\times(1-Y_{k})}\\
	&\times \left[\left(1 - \Phi(X_{l}\beta_{l} + \gamma_{l}Y_{k} + \rho_{l}u + \delta_{l}v) \right)\right.\\
    & \left. \times \ \Phi(X_{k}\beta_{k} + \rho_{k}u + \delta_{k}v) \right]^{(1-Y_{l})\times Y_{k}}\\
	&\times \left[\Phi(X_{l}\beta_{l} + \gamma_{l}Y_{k} + \rho_{l}u + \delta_{l}v) \right.\\
    &\left. \times \ f(X_{hl}\beta_{hl} + \gamma_{hl}Y_{k} + \rho_{hl}u + \delta_{hl}v) \right.  \\
	&\left. \times\ \Phi(X_{k}\beta_{k} + \gamma_{k}Y_{hl} + \rho_{k}u + \delta_{k}v) \right]^{Y_{l} \times Y_{k}}.
\end{align*}

Cette vraisemblance est celle d’un modèle à facteurs latents discrets (DLFM). Ces modèles à mélange fini semi-paramétriques permettent généralement d'approximer des distributions continues. \textcite{mroz1999} montre que ces modèles réduisent le biais d'identification de la distribution des facteurs latents lorsqu'ils ne suivent pas une loi normale (voir aussi \textcite{heckman2007}) et donnent de «bons résultats» lorsque les instruments sont faibles. Dans ce travail, les facteurs latents d'hétérogénéité inobservés $\tau_b$ rendent dépendantes les trois relations associées aux variables à expliquer. Étant données les hypothèses faites sur les variables aléatoires $u$ et $v$, la distribution de l'hétérogénéité inobservée est approximée par une distribution discrète multivariée à 4 points masses définis selon le couple de valeurs $(u,v)$\footnote{$\tau_b=0$ avec une probabilité $Pr(u=0,v=0)=(1-p_u)(1-p_v)$~; $\tau_b=\rho_b$ avec une probabilité $Pr(u=1,v=0)=p_u(1-p_v)$~; $\tau_b=\delta_b$ avec une probabilité $Pr(u=0,v=1)=(1-p_u)p_v$ et $\tau_b=\rho_b+\delta_b$ avec une probabilité $Pr(u=1,v=1)=p_u p_v$~; $b=l,k,hl$. La somme des probabilités est égale à $1$.}. L’identification des paramètres associés à cette distribution des erreurs se fait grâce à la normalisation des points de support de $u$ et $v$. De plus, la distribution discrète permet aux variances des $\tau_b$ et donc aux variances des $\varepsilon_b$ de ne pas être constantes, les variances étant constantes pour chaque point masse mais différentes entre ces points. L’approche d'identification avec hétéroscédasticité (\textcite{lewbel2012}) permet elle aussi d'identifier les paramètres de la distribution des facteurs latents. Concernant les paramètres associés aux aides, lorsque les covariables sont identiques pour les différentes équations, l'identification est basée sur la non linéarité du modèle et donc sur une identification paramétrique. Cette identification obtenue sur la seule forme fonctionnelle peut conduire à des coefficients instables. Pour une meilleure identification il serait préférable d’avoir des conditions d'exclusion. Toutefois, comme pour \textcite{BALIA2014}\footnote{\textcite{BALIA2014} considèrent comme variables d'exclusion le renoncement aux soins pour les équations associées à l'aide formelle et la plus petite distance entre le domicile de la personne de celui de ses enfants pour les équations associées à l'aide informelle. Or, ils s'avèrent que ces informations ne sont pas des variables d'exclusion satisfaisantes puisque qu'elles sont significatives dans le modèle sans variable d'exclusion.} il est difficile de trouver des variables d'exclusion. D'autres modèles, cf. \autoref{robustesse}, avec des ensembles de variables différents pour les équations, suppression d'une variable -~revenu, nombre d'enfants~-, par exemple, ont été estimés. Les résultats associés aux variables d'intérêt étant inchangés, nous avons choisi de présenter le modèle avec le même ensemble de variables explicatives pour les trois équations.

%%%%%%%%%%%%%%%%%%%%%%%%%%%%%%%%%%%%%%%%%%%%%%%%%%%%%
						\section{Résultats et discussion}
%%%%%%%%%%%%%%%%%%%%%%%%%%%%%%%%%%%%%%%%%%%%%%%%%%%%%

%%%%%%%%%%%%%%%%%%%%%%%%%%%%%%%%%%%%%%%%%%%%%%%%%%%%%
\subsection{Statistiques descriptives}
%%%%%%%%%%%%%%%%%%%%%%%%%%%%%%%%%%%%%%%%%%%%%%%%%%%%%

Le \autoref{stat_des_care} donne les statistiques descriptives liées aux aides. 69,3 \% des individus de l'échantillon ne reçoivent aucune aide. Près de 20 \%,  soit un peu moins d'une personne sur cinq, ont au moins un type d'aide (17,7 \% pour l'aide formelle et 18,6 \% pour l'aide informelle) et 5,7 \% des personnes âgées étudiées utilisent à la fois de l'aide formelle et de l'aide informelle. Si l'on distingue l'aide domestique et personnelle, on constate que la première est utilisée presque quatre fois plus que la seconde (30,1 \% contre 7,9 \%). En revanche, 7,2 \% des individus reçoivent ces deux types d'aide, par conséquent, la quasi totalité des personnes âgées ayant recours à l'aide personnelle, a également recours à de l'aide domestique. Que l'aide soit formelle ou informelle les proportions, 4,8 \% pour l'aide personnelle formelle contre 4,1 \% pour l'aide personnelle informelle et 17,2 \% pour l'aide domestique formelle contre 18,1 \% pour l'aide domestique informelle, sont relativement proches.

\begin{table}[t]
\caption{Statistiques descriptives liées aux aides}
\label{stat_des_care}
\centering
\begin{threeparttable}[t]
\begin{tabularx}{\linewidth}{@{}X rrl@{}}
\toprule
	& {Effectif} & \multicolumn{2}{c}{\%} \\\midrule
	Ensemble & 8 799 & 100 & \\\midrule
	Aucune aide (NoA) & 6 100 & 69,3 & (57,2) \\\midrule
	Aide formelle (AF) & 1 561 & 17,7 & \\
	Aide informelle (AI) & 1 638 & 18,6 & (34,0) \\\midrule
	Aide formelle et informelle & 500 &  5,7 & (8,9) \\
	Aide formelle seulement & 1 061 & 12,1 & \\
	Aide informelle seulement & 1 138 & 12,9 & (25,1) \\\midrule
	Aide personnelle (AP) & 691 & 7,9 & (8,4) \\
	Aide domestique (AD) & 2 646 & 30,1 & (42,2) \\\midrule
	Aide personnelle et domestique & 638 & 7,2 & (7,7) \\
	Aide personnelle seulement & 53 &  0,6 &  (0,6) \\
	Aide domestique seulement & 2 008 & 22,8 & (34,5) \\\midrule
	%\textbf{Personal care} & & & \\
	Aide personnelle formelle (APF) & 426 & 4,8 & \\
	Aide personnelle informelle (API) & 358 & 4,1 & (4,7) \\
	Aide personnelle formelle et informelle & 93 & 1,1 & (1,1) \\
	Aide personnelle formelle seulement & 333 & 3,8 & \\
	Aide personnelle informelle seulement & 265 & 3,0 & (3,5) \\\midrule
	%\textbf{Domestic care} & & & \\
	Aide domestique formelle (ADF) & 1 516 & 17,2 & \\
	Aide domestique informelle (ADI) & 1 596 & 18,1 & (33,4) \\
	Aide domestique formelle et informelle & 466 & 5,3 & (8,5) \\
	Aide domestique formelle seulement & 1 050 & 11,9 & \\
	Aide domestique informelle seulement & 1 130 & 12,8 & (25,0) \\\bottomrule
%	\multicolumn{4}{l}{\footnotesize Lecture : $6100$ individus ne reçoivent aucune aide formelle et pas d'aide informelle}\\
%	\multicolumn{4}{l}{\footnotesize au moins une fois par semaine soit $69,3~\%$ de l'échantillon. Si l'on considère l'aide}\\
%	\multicolumn{4}{l}{\footnotesize informelle quelque soit sa récurrence ils ne sont plus que $57,2~\%$ de l'échantillon.}
\end{tabularx}
\begin{tablenotes}[para,flushleft] \footnotesize
\item Lecture : $6100$ individus ne reçoivent aucune aide formelle et pas d'aide informelle au moins une fois par semaine soit $69,3~\%$ de l'échantillon. Si l'on considère l'aide informelle quelque soit sa récurrence ils ne sont plus que $57,2~\%$ de l'échantillon.
\end{tablenotes}
\end{threeparttable}
\end{table}

Le \autoref{stat_des_heures_care} contient le nombre moyen d'heures d'aide formelle par semaine et par objet de l'aide. Les individus qui font appel à l'aide formelle, en utilisent en moyenne 10,4 h par semaine. Ils reçoivent en moyenne 3,4 h d'aide personnelle et 7,5 h d'aide domestique par semaine. Lorsque les individus reçoivent à la fois de l'aide formelle domestique et personnelle, le nombre moyen d'heures d'aide augmente fortement et passe à 25,3 h, les heures sont relativement bien réparties entre personnelle (12,9 h) et domestique (14,7 h). Ces statistiques confirment que la quasi totalité des personnes ayant de l'aide personnelle ont aussi recours à de l'aide domestique. Le \autoref{quali_variablesA} en annexe \ref{Tableaux} montre que les personnes qui reçoivent de l'aide personnelle sont moins autonomes que les autres (état de santé dégradé, problèmes de mobilités, etc.).

\begin{table}[]
\centering
\caption{Statistiques descriptives des heures d'aide formelle}
\label{stat_des_heures_care}
\begin{threeparttable}[t]
\begin{tabularx}{\linewidth}{@{}X rrrrrrr@{}}
	\toprule
	& Effectif & Moyenne & Q1 & Q2 & Q3 \\\midrule
	Aide formelle (AF) & 1 561 & & & & \\
	Heures formelles (hAF) & & 10,4 & 2 & 3 & 7 \\
	Heures personnelles formelles (hAPF) & & 3,4 & 0 & 0 & 1 \\
	Heures domestiques formelles (hADF) & & 7,5 & 2 & 3 & 5 \\\midrule
	Aide formelle personnelle et domestique & 381 & & & & \\
	Heures formelles (hAF) & & 25,3 & 5 & 10 & 30 \\
	Heures personnelles formelles (hAPF) & & 11,9 & 1,5 & 4 & 12 \\
	Heures domestiques formelles (hADF) & & 12,7 & 2 & 5 & 12 \\ \midrule
	Aide formelle personnelle uniquement & 45 & & & & \\
	Heures personnelles formelles (hAPF) & & 8,9 & 1 & 4 & 7 \\\midrule
	Aide formelle domestique uniquement & 1 135 & & & & \\
	Heures domestiques formelles (hADF) & & 5,4 & 2 & 3 & 4 \\\midrule
\end{tabularx}
% \begin{tablenotes}[para,flushleft] \footnotesize
% \item %Lecture : .
% \end{tablenotes}
\end{threeparttable}
\end{table}

Les statistiques descriptives associées aux principales variables explicatives sont données \autoref{quanti_variablesA} et \autoref{quali_variablesA} en annexe \ref{Tableaux}. Ces statistiques montrent que le profil des personnes qui reçoivent de l'aide diffère selon le type d'aide reçue. Les trois quarts de l'échantillon sont des femmes (74,9 \%). On en compte 81,4 \% parmi les personnes qui reçoivent de l’aide informelle, 78,7 \% parmi celles qui reçoivent de l'aide personnelle et 77,2 \% parmi celles qui reçoivent de l'aide domestique. Alors que le profil des personnes qui ne reçoivent aucune aide est relativement proche de celui de l'échantillon, le profil des personnes qui reçoivent de l'aide personnelle est très différent. Il s'agit de personnes peu ou pas diplômées, se déclarant en mauvaise santé, fortement limitées, ayant des problèmes de mobilité importants nécessitant l'utilisation d'aides à la marche et assez isolées. Des spécificités apparaissent également pour les autres types d'aides mais elles sont moins marquées que pour l'aide personnelle (cf. \autoref{quali_variablesA}). Notons que 91,4~\% des personnes qui ne reçoivent aucune aide ne déclarent aucune ADL mais seulement 653 personnes (soit 10,7~\% des personnes sans aide et 7,4~\% de l’échantillon) ne déclarent aucun problème\footnote{Pas d'ADL, ni de problème de mobilité, ni d'aide à la marche, ni de maladie chronique, une bonne santé mentale, une force de préhension forte et pas de limitation ressentie (GALI).}. Ces statistiques ont été recalculées pour le sous échantillon des 75 ans et plus. 59,3~\% déclarent n'avoir aucune aide mais seulement 187 ne déclarent aucun problème\footnote{La même définition que précédemment a été retenue. Les statistiques descriptives associées aux différents sous-groupes sont disponibles auprès des auteurs sur demande.} (soit 6,1~\% des personnes sans aide et 4,2~\% des personnes de 75 ans et plus). Le \autoref{quanti_variablesA} résume les principales statistiques descriptives associées aux variables continues (âge, revenu et nombre d'enfants). L'âge moyen est de 76,9 ans. Les personnes ne recevant aucune aide sont légèrement plus jeunes (75,2 ans) et celles qui bénéficient d'au moins un type d'aide sont plus âgées (plus de 80 ans et près de 83 ans quand il s'agit de soins personnels). Le nombre moyen d'enfants est d'environ 2. Le revenu médian des bénéficiaires des deux types d'aide est plus élevé que pour les autres sous-ensembles d'individus.


%%%%%%%%%%%%%%%%%%%%%%%%%%%%%%%%%%%%%%%%%%%%%%%%%%%%%
\subsection{Résultats des estimations}
%%%%%%%%%%%%%%%%%%%%%%%%%%%%%%%%%%%%%%%%%%%%%%%%%%%%%

Les commentaires sont réalisés toutes choses égales par ailleurs. Une indicatrice de pays a été introduite dans toutes les équations. Les résultats des estimations pour l'aide agrégée formelle et l'aide agrégée informelle sont présentés dans le \autoref{res_fcic}. Les coefficients associés à l'hétérogénéité non observée ($\theta_u, \theta_v, \rho, \delta$) sont significatifs, la double causalité est avérée.

\begin{table}
\centering
\caption{Résultats des estimations : aides non-désagrégées\label{res_fcic}}
\resizebox*{!}{\dimexpr\textheight-4\baselineskip\relax}{
	\sisetup{
		% parse-numbers = true,
		output-decimal-marker={,},
		%  table-number-alignment = right,
		% table-text-alignment = right,
		%table-alignment  = right,
		negative-color = red,
		input-open-uncertainty = ,
		input-close-uncertainty = ,
		minimum-decimal-digits = 0,
		table-format = (+1.3, %)
		table-align-text-before = false,
		table-align-text-after = false,
		round-mode = places,
		round-precision=3,
		round-pad = false,
	}
\begin{threeparttable}
	\begin{tabular}{l@{\quad}l@{} S!{\quad}S!{\quad}S!{\quad}}
		\toprule\midrule
		&& \multicolumn{1}{c}{Aide formelle} & \multicolumn{1}{c}{Heures aide formelle}& \multicolumn{1}{c}{Aide informelle} \\ \hline
		Sexe & homme (réf.) & & & \\
		& femme & -0.19574\sym{***} & -0.07094 & 0.28419 \\\hline
		\^Age& & 0.08386\sym{***} & 0.03234\sym{***} & 0.14053\sym{***} \\\hline
		Nombre d'enfants & & -0.02849 & 0.02918 & 0.21615\sym{***} \\\hline
		Revenu (en logarithme) & & 0.03865 & 0.00297 & -0.00761 \\\hline
		Niveau d'éducation& sans diplôme (réf.) & & &  \\
		& inférieur au bac& 0.12383 & 0.0575 & -0.25303 \\
		& niveau bac & 0.21191 & 0.02232 & -0.60706\sym{*} \\
		& études supérieures & 0.43218\sym{***} & 0.11788 & -0.84429\sym{**} \\\hline
		Zone résidentielle& Très grande ville (réf.)  & & & \\
		& banlieue d'une très grande ville & 0.03655 & -0.01461 & 0.11548 \\
		& grande ville & -0.02378 & -0.02387 & 0.15947 \\
		& petite ville & 0.00575 & 0.0217 & 0.45076\sym{*} \\
		& zone rurale & -0.1147 & 0.11587 & 0.68543\sym{***} \\
		& non renseignée & -0.00267 & 0.16761 & 0.68985\sym{**} \\\hline
		ADL déclarées& aucune (réf.) & & & \\
		& oui, mais pas bain ou douche & 0.46838\sym{***} & 0.26004\sym{**} & 0.80059\sym{***} \\
		& oui, dont bain ou douche & 1.13217\sym{***} & 1.02879\sym{***} & 1.79563\sym{***} \\\hline
		Problèmes de mobilité& aucun ou haut du corps (réf.)& & & \\
		& oui, bas du corps & 0.33521\sym{***} & -0.03591 & 0.6453\sym{***} \\
		& oui, haut et bas du corps & 0.44841\sym{***} & 0.04118 & 0.77488\sym{***} \\\hline
		Maladies chroniques déclarées & aucune (réf.) & & & \\
		& oui, légères & 0.2151\sym{**} & -0.18477 & 0.53613 \\
		& oui, graves & 0.38633\sym{***} & -0.1057 & 1.01413\sym{***} \\ \hline
		Aide à la marche & non ou autre (réf.) & & & \\
		& fauteuil roulant, déambulateur & 1.23829\sym{***} & 0.72776\sym{***} & 1.84458\sym{***} \\\hline
		Force de préhension faible & non (réf.) & & &  \\
		& oui & 0.06993 & 0.21537\sym{***} & 0.00394 \\\hline
		Personnes de confiance  & aucune (réf.) & & &  \\
		& moins de 4 & -0.20298 & 0.05781 & -0.71474 \\
		& 4 et plus & 0.25789\sym{*} & -0.07566 & 1.57071\sym{***} \\\hline
		Santé mentale mauvaise  & non (réf.) & & &  \\
		& oui & 0.22177\sym{***} & 0.08543 & 0.52428\sym{***} \\\hline
		Sentiment de solitude& jamais (réf.) & & & \\
		 & parfois & -0.1032 & -0.14928\sym{**} & -0.16703 \\
		& souvent & 0.12783 & -0.12609 & -0.08582 \\\hline
		GALI & pas limité (réf.)& & & \\
		& limité mais pas fortement & 0.48597\sym{***} & 0.2719\sym{***} & 1.08524\sym{***} \\
		& fortement limité & 0.92611\sym{***} & 0.63327\sym{***} & 2.05048\sym{***} \\\hline
		Pays de résidence & France (réf.) & & & \\
		& Allemagne & -0.73728\sym{***} & 0.28685\sym{*} & 1.16077\sym{**} \\
		& Autriche & -0.31988\sym{**} & 0.46976\sym{***} & 1.70502\sym{***} \\
		& Belgique & 0.37124\sym{***} & 0.34388\sym{***} & 1.08636\sym{***} \\
		& Bulgarie & -1.35088\sym{***} & 0.82978\sym{**} & 2.00312\sym{***} \\
		& Chypre & 0.11559 & 1.81615\sym{***} & 1.1037 \\
		& Croatie & -0.44906\sym{*} & 0.53932\sym{*} & 3.05473\sym{***} \\
		& Danemark & -0.45654\sym{***} & -0.68984\sym{***} & -0.0463 \\
		& Espagne & -0.1356 & 0.6519\sym{***} & 0.55437 \\
		& Estonie & -2.00973\sym{***} & -0.05693 & 1.77461\sym{***} \\
		& Finlande & -0.84051\sym{***} & -0.00686 & 0.76742 \\
		& Grèce & -0.95696\sym{***} & 1.11903\sym{***} & 1.0935\sym{**} \\
		& Hongrie & -0.89597\sym{***} & 0.34749 & 1.51248\sym{***} \\
		& Israël & 0.121 & 1.44968\sym{***} & 1.61007\sym{**} \\
		& Italie & -0.35201\sym{**} & 0.68105\sym{***} & 0.93182\sym{*} \\
		& Lettonie & -1.95049\sym{***} & 0.11217 & 0.52396 \\
		& Lituanie & -1.72032\sym{***} & 0.18492 & 1.4578\sym{***} \\
		& Luxembourg & -0.34817 & 0.18855 & 1.00978 \\
		& Malte & -0.87915\sym{***} & 0.15271 & -0.37466 \\
		& Pologne & -1.71939\sym{***} & 0.5533\sym{*} & 1.05918\sym{**} \\
		& République tchèque & -1.10813\sym{***} & 0.34961\sym{*} & 2.96802\sym{***} \\
		& Roumanie & -1.90634\sym{***} & 0.54175 & 2.2087\sym{***} \\
		& Slovaquie & -1.67352\sym{***} & 0.38149 & 0.79919 \\
		& Slovénie & -1.72788\sym{***} & 0.58398\sym{**} & 1.45113\sym{***} \\
		& Suède & -1.13474\sym{***} & -0.26616\sym{*} & -1.19148\sym{**} \\
		& Suisse & -0.23279\sym{*} & 0.23026 & 0.8216\sym{*} \\\hline
		Constante& & -5.57662\sym{***} & -0.00104 & -10.18565\sym{***} \\\hline\hline
		Heures d'aide formelle & & & & -0.07771\sym{***} \\
		Aide informelle & & -2.03663\sym{***} & -1.34393\sym{***} &\\\hline\hline
		Paramètres & $\theta_h$ & & -0.114 & \\
		& $\rho$ & -2.52369\sym{***} & -1.90072\sym{***} & -8.33471***\\
		& $\delta$ &-1.85687\sym{***} & -0.90756\sym{***} & -4.91781\sym{***}\\\cline{2-5}
		& $\theta_u$ && 1.78627\sym{***} &\\
		& $\theta_v$ && 0.15535\sym{*} &\\\hline
	Nombre d'observations & 8799 &&&\\\hline
	Log vraisemblance & -10591,79 &&&\\\hline\hline
%\multicolumn{5}{l}{Significativité : \sym{*} $p<0,10$, \sym{**} $p<0,05$, \sym{***} $p<0,01$.}
	\end{tabular}
\begin{tablenotes}[para,flushleft] \footnotesize
\item Significativité : \sym{*} $p<0,10$, \sym{**} $p<0,05$, \sym{***} $p<0,01$.
\end{tablenotes}
\end{threeparttable}
}
\end{table}

Les effets du genre, de l'âge, du nombre d'enfants et du niveau d'éducation sont cohérents avec ceux trouvés dans la littérature. La probabilité de recevoir de l'aide (formelle ou informelle) ainsi que le nombre d'heures d'aide formelle utilisées augmentent avec l'âge\footnote{L’âge a été testé de manière linéaire puis non linéaire. Le coefficient associé au carré n’est pas significatif pour l'aide et l'aide domestique (cf. \autoref{robustesse}).}. Les femmes ont une probabilité de recevoir de l'aide formelle plus faible que les hommes. Le nombre d'enfants et le niveau d'études ont des effets opposés sur la probabilité de recevoir de l'aide informelle, augmente avec le nombre d'enfants et diminue avec le niveau d'éducation. La probabilité de recevoir de l'aide formelle ne dépend pas du nombre d'enfants et augmente pour les personnes qui ont fait des études supérieures. La zone de résidence ne semble pas être un facteur déterminant pour expliquer la probabilité de recevoir une aide : seules les personnes vivant dans des zones rurales ont une probabilité plus élevée de recevoir de l'aide informelle. Deux raisons pourraient expliquer ce résultat~: une solidarité envers les ainés plus forte et/ou une offre d’aide formelle plus faible. Notons que le revenu n'a d'effet significatif sur aucune de ces deux probabilités. Ce résultat est conforme avec la littérature qui ne met pas réellement en évidence d’effet revenu sur l’accès à l’aide formelle ou aux heures \parencite{BALIA2014,PERDRIX2021}, par exemple. Cet effet peut être attribué soit au niveau d’études, soit à des effets plus macroéconomiques liés aux pays.

De façon générale, avoir déclaré des problèmes de santé impacte la probabilité de recevoir de l'aide, l'effet étant plus ou moins important selon le problème. Avoir une maladie chronique augmente la probabilité de recevoir de l'aide formelle. Si cette maladie à des conséquences physiques l'augmentation concernera à la fois la probabilité de recevoir de l'aide formelle mais aussi de l'aide informelle. Avoir des difficultés dans les activités de la vie quotidienne (ADL) augmente la probabilité de recevoir de l'aide informelle, de l'aide formelle mais aussi la quantité d'heures d'aide formelle utilisée. Cette prise en charge est particulièrement importante lorsque la personne a déclaré avoir des difficultés pour se laver (bain ou douche). Les problèmes de mobilité ont les mêmes effets que les ADL sur les deux probabilités mais aucun effet sur les heures d'aide formelle reçue. Ces probabilités augmentent très significativement lorsque ces problèmes de mobilité obligent la personne à utiliser des outils d'aide à la marche (déambulateur, fauteuil roulant). Dans ce cas, le nombre d'heures d'aide formelle augmente lui aussi.

Les variables de comportements et de perceptions ne semblent pas impacter de façon comparable la probabilité de recevoir une aide. Avoir plus de 4 personnes de confiance augmente fortement la probabilité d'avoir de l'aide informelle et faiblement celle de recevoir de l'aide formelle. Le sentiment de solitude, n'a globalement aucun effet. La mauvaise santé mentale, mesurée par une indicatrice associée à un indice Euro-D au moins égal à 4 augmente très significativement la probabilité d'avoir de l'aide qu'elle soit formelle ou informelle mais pas le nombre d'heures le nombre d'heures d'aide formelle. Pour terminer avec ces variables, nous pouvons noter que la variable GALI a des effets conformes aux attentes~: d'une part, plus les limitations sont fortes, plus la probabilité de recevoir une aide (quelle soit formelle ou informelle) est forte et, d'autre part, plus le nombre d'heures d'aide formelle est important.

Selon le pays de résidence la probabilité de recevoir de l'aide est différente. Le pays de référence dans le modèle est la France. Un pays est associé a une probabilité plus forte de recevoir de l'aide (formelle et informelle) : la Belgique et trois pays ont une probabilité comparable : Chypre, l'Espagne et le Luxembourg. Pour un ensemble de pays (Allemagne, Autriche, Bulgarie, Croatie, Estonie, Grèce, Hongrie, Israël, Lituanie, République Tchèque, Roumanie, Slovénie) la probabilité de recevoir de l'aide informelle est plus forte et la probabilité de recevoir de l'aide formelle est plus faible (comparable pour Israël). Un autre groupe de pays (Danemark, Finlande, Italie, Lettonie, Malte, Pologne, Slovaquie, Suisse) a une probabilité plus faible de recevoir de l'aide formelle et une probabilité de recevoir de l'aide informelle comparable à celle de la France. Enfin, notons que pour la Suède, la probabilité de recevoir de l'aide formelle mais aussi de l'aide informelle est plus faible. Ces effets fixes de pays ne permettent pas de contrôler précisément les caractéristiques économiques, institutionnelles (aide aux personnes âgées) et culturelles du pays. Toutefois l’échelle culturelle de « solidarité familiale » proposée par \textcite{GEERTS2012}~: l’Allemagne, l’Autriche, l’Espagne et l’Italie ont un niveau de solidarité élevé~; la Belgique et la France ont un niveau moyen et le Danemark et la Suède ont un niveau faible~; est plutôt bien respectée concernant la probabilité d’aide informelle. Les caractéristiques économiques et institutionnelles décrites par \textcite{rostgaard2011} pour l’Allemagne, l’Autriche, le Danemark, la Finlande, l’Italie, et la Suède semblent en accord avec nos résultats sur l'aide formelle.

Terminons le commentaire du \autoref{res_fcic} par les coefficients associés aux aides : ils sont négatifs et très significatifs. Recevoir de l'aide informelle diminue la probabilité de recevoir de l'aide formelle et diminue aussi le nombre d'heures d'aide formelle reçues. Inversement, plus le nombre d'heures d'aide formelle augmente, plus la probabilité de recevoir de l'aide informelle sera faible. Il semblerait donc que les deux types d'aide soient \enquote{substituables}. Toutefois, les deux types d'aide (formelle et informelle) incluent à la fois des travaux domestiques (aide domestique) et des soins à la personne (non médicaux). Par conséquent, étant donné que les services rendus ne sont pas homogènes, parler de \enquote{biens substituables} n'est pas tout à fait correct. Le modèle a été estimé en distinguant l'aide domestique (formelle, informelle et les heures d'aide formelle domestique) de l'aide personnelle (formelle, informelle et les heures d'aide formelle personnelle). Les résultats de ces estimations sont donnés dans le \autoref{resultats}.

\begin{table}
	\centering
	\caption{Résultats des estimations\label{resultats}}
\resizebox*{!}{\dimexpr\textheight-6\baselineskip\relax}{
		\sisetup{
			% parse-numbers = true,
			output-decimal-marker={,},
			%table-number-alignment = right,
			%table-text-alignment = right,
			table-alignment  = right,
			negative-color = red,
			input-open-uncertainty = ,
			input-close-uncertainty = ,
			minimum-decimal-digits = 0,
			table-format = (+1.3),
			table-align-text-before = false,
			table-align-text-after = false,
			round-mode = places,
			round-precision=3,
			round-pad = false,
		}
	%\hspace{-2cm}
\begin{threeparttable}
\begin{tabular}{l@{}l@{} S!{\qquad}S!{\qquad}S!{\quad} |S!{\qquad}S!{\qquad}S!{\quad}}
	\toprule\midrule
		&& \multicolumn{3}{c|}{Aide domestique} & \multicolumn{3}{c}{Aide personnelle}\\
		&&{formelle}&\multicolumn{1}{c}{heures formelles} & \multicolumn{1}{c|}{informelle} &{formelle}&\multicolumn{1}{c}{heures formelles} & \multicolumn{1}{c}{informelle}\\\hline

	Sexe & homme (réf.) && & && &\\
	& femme & -0.19534\sym{***} & -0.08945 & 0.15392 & -0.24521\sym{**} & -0.10613 & -0.33656\sym{*}\\\hline
	\^Age& & 0.07918\sym{***} & 0.03039\sym{***} & 0.082\sym{***} & 0.04599\sym{***} & 0.00905 & 0.05893\sym{***}\\\hline
	Nombre d'enfants & & -0.04071\sym{**} & 0.00875 & 0.09587\sym{***} & -0.01262 & 0.06651 & 0.08002\\\hline
	Revenu (en logarithme) & & 0.03045 & 0.00374 & -0.04292 & 0.00928 & -0.00004 & 0.0396\\\hline
	Niveau d'éducation& sans diplôme (réf.) && & && &\\
	& inférieur au bac & 0.15759 & 0.006 & -0.12843 & -0.28961\sym{*} & 0.28348 & -0.47444\sym{*}\\
	& niveau bac & 0.27509\sym{**} & -0.00979 & -0.28418 & -0.302 & 0.3033 & -0.74226\sym{**}\\
	& études supérieures & 0.47501\sym{***} & 0.11952 & -0.44795\sym{**}& -0.38765\sym{*} & 0.46333 & -0.94932\sym{***}\\\hline
	Zone résidentielle& très grande ville (réf.) && &&& &\\
	& banlieue d'une très grande ville & 0.01271 & -0.01783 & -0.00101 & 0.15968 & 0.08362 & 0.56768\\
	& grande ville & -0.02093 & -0.00596 & 0.091 & 0.05628 & 0.51428\sym{**} & 0.6378\sym{**}\\
	& petite ville & -0.01057 & 0.03914 & 0.2516\sym{*} & -0.08675 & 0.21237 & 0.15726\\
	& zone rurale & -0.11547 & 0.13604 & 0.42691\sym{***} & -0.1795 & 0.24139 & 0.40841\sym{*}\\
	& non renseignée & 0.01677 & 0.19502 & 0.43908\sym{**} & 0.06878 & 0.37694 & 0.1665\\\hline
	ADL déclarées& aucune (réf.) && & && &\\
	& oui, mais pas bain ou douche & 0.39977\sym{***} & 0.16991 & 0.45954\sym{***} & 1.34094\sym{***} & 1.28091\sym{***} & 1.91374\sym{***}\\
	& oui, dont bain ou douche & 0.92118\sym{***} & 0.56693\sym{***} & 0.92495\sym{***} & 2.41344\sym{***} & 1.79407\sym{***} & 3.84145\sym{***}\\\hline
	Problèmes de mobilité& aucun ou haut du corps (réf.)&& & && &\\
	& oui, bas du corps & 0.33464\sym{***} & -0.0863 & 0.38964\sym{***} & 0.02651 & 0.19009 & 0.41435\sym{**}\\
	& oui, haut et bas du corps & 0.43363\sym{***} & 0.00625 & 0.44721\sym{***} & 0.2126 & 0.30552 & 0.61765\sym{***}\\\hline
	Maladies chroniques déclarées & aucune (réf.) && & && &\\
	& oui, légères & 0.24279\sym{**} & -0.11339 & 0.55026\sym{***} & 0.20649 & 0.5723 & 0.07225\\
	& oui, graves & 0.37695\sym{***} & -0.06246 & 0.80063\sym{***} & 0.35555 & 0.47839 & 0.46545\\\hline
	Aide à la marche & non ou autre (réf.) && & && &\\
	& fauteuil roulant, déambulateur & 1.1391\sym{***} & 0.55561\sym{***} & 1.14756\sym{***} & 0.95963\sym{***} & 0.76005\sym{***} & 1.1231\sym{***}\\\hline
	Force de préhension faible & non (réf.) && &  && & \\
	& oui & 0.11812 & 0.21204\sym{***} & 0.10676 & 0.18994\sym{*} & 0.08248 & 0.18607\\\hline
	Personnes de confiance & aucune (réf.) && &  && & \\
	& moins de 4 & -0.13986 & 0.16446 & -0.48812 & -0.81261\sym{**} & -0.2412 & -0.51302\\
	& 4 et plus & 0.24138\sym{*} & 0.01812 & 0.93045\sym{***} & -0.20226 & -0.46381 & 0.60485\sym{*}\\\hline
	Santé mentale mauvaise & non (réf.) && &&  & & \\
	& oui & 0.2015\sym{***} & 0.11178 & 0.32746\sym{***} & 0.06456 & -0.00267 & 0.45995\sym{***}\\\hline
	Sentiment de solitude& jamais (réf.) && &  && & \\
	& parfois & -0.0835 & -0.09515 & -0.01912 & -0.04135 & -0.51326\sym{***} & 0.03849\\
	& souvent & 0.1698\sym{**} & -0.0963 & 0.01246 & 0.07666 & -0.44497\sym{**} & -0.00546\\\hline
	GALI & pas limité (réf.) &  &  &  &  &  & \\
	& limité mais pas fortement & 0.43926\sym{***} & 0.17575\sym{**} & 0.51205\sym{***} & 0.64879\sym{***} & 0.36306 & 0.42814\sym{*}\\
	& fortement limité & 0.87185\sym{***} & 0.42683\sym{***} & 1.05983\sym{***} & 0.99349\sym{***} & 0.41381 & 1.04743\sym{***}\\\hline
	Pays de résidence & France (réf.) && & && &\\
	& Allemagne & -0.78981\sym{***} & 0.28303\sym{*} & 0.45565\sym{*} & -0.32103 & 0.01609 & 1.6023\sym{***}\\
	& Autriche & -0.33664\sym{**} & 0.47442\sym{***} & 0.97337\sym{***} & -0.3062 & 0.26994 & 2.01068\sym{***}\\
	& Belgique & 0.32337\sym{***} & 0.33568\sym{***} & 0.70565\sym{***} & 0.38004\sym{**} & -0.2463 & 0.30332\\
	& Bulgarie & -1.40995\sym{***} & 0.92361\sym{**} & 0.95199\sym{***} & -1.31409\sym{***} & 1.22436 & 2.03236\sym{***}\\
	& Chypre & -0.02361 & 1.74729\sym{***} & 0.15848 & 0.40782 & 2.29884\sym{***} & 1.81299\sym{***}\\
	& Croatie & -0.59148\sym{**} & 0.53443\sym{*} & 1.54926\sym{***} & -0.6419 & 1.01335 & 1.04311\\
	& Danemark & -0.497\sym{***} & -0.67401\sym{***} & -0.02121 &&&\\
	& Espagne & -0.18483 & 0.59705\sym{***} & 0.01374 & 0.20129 & 0.98179\sym{***} & 1.8627\sym{***}\\
	& Estonie & -2.07516\sym{***} & -0.16732 & 0.77558\sym{***} & -1.43115\sym{***} & -0.32956 & 1.97566\sym{***}\\
	& Finlande & -0.92708\sym{***} & -0.22612 & 0.09666 & -0.13775 & 1.24494\sym{**} & 2.4773\sym{***}\\
	& Grèce & -0.92732\sym{***} & 1.03406\sym{***} & 0.50987\sym{**} & -0.48451\sym{*} & 1.70712\sym{***} & 2.0954\sym{***}\\
	& Hongrie & -0.91488\sym{***} & 0.28919 & 0.77829\sym{**} & 0.0485 & 0.76924\sym{*} & 2.36066\sym{***}\\
	& Israël & 0.0216 & 1.50449\sym{***} & 0.72396\sym{*} & 0.48245\sym{*} & 1.30324\sym{***} & 1.69657\sym{**}\\
	& Italie & -0.33089\sym{**} & 0.67596\sym{***} & 0.41698 & -0.27112 & 1.30261\sym{***} & 2.19091\sym{***}\\
	& Lettonie & -1.88691\sym{***} & 0.33888 & 0.15072 & -0.93865\sym{**} & 0.29673 & 1.03339\\
	& Lituanie & -1.73424\sym{***} & 0.18255 & 0.62498\sym{**} & -0.83846\sym{***} & -0.06052 & 2.2323\sym{***}\\
	& Luxembourg & -0.4557\sym{**} & 0.29501 & 0.67699\sym{*} & -0.18291 & 0.15805 & 0.37424\\
	& Malte & -0.83635\sym{**} & 0.35583 & -0.2148 &&&\\
	& Pologne & -1.86504\sym{***} & 0.42469 & 0.5057\sym{*} & -0.985\sym{***} & 1.03911\sym{*} & 0.94002\\
	& République tchèque & -1.19633\sym{***} & 0.28757 & 1.57357\sym{***} & -0.54242\sym{**} & 0.75368\sym{**} & 2.26506\sym{***}\\
	& Roumanie & -2.16317\sym{***} & -0.09397 & 1.17355\sym{***} & -0.91103\sym{**} & -0.13813 & 2.81787\sym{***}\\
	& Slovaquie & -1.67452\sym{***} & 0.02536 & 0.19546 & -0.35725 & 1.32174\sym{*} & 2.09188\sym{***}\\
	& Slovénie & -1.8013\sym{***} & 0.48937\sym{**} & 0.76773\sym{***} & -0.99617\sym{***} & 0.30406 & 2.26156\sym{***}\\
	& Suède & -1.07368\sym{***} & -0.2248 & -0.72004\sym{**} &&&\\
	& Suisse & -0.34955\sym{***} & 0.04267 & 0.18934 &&&\\ \hline
	Constante & & -5.04584\sym{***} & 0.15208 & -4.35047\sym{***} & -0.90573 & 2.63593\sym{**} & 9.96243\\\hline\hline
	Heures d'aide formelle domestique & & & &-0.06441\sym{***} & & & -0.10524\sym{***}\\
	Aide informelle domestique & & -1.63774\sym{***} & -1.03541\sym{***} && -1.36916\sym{***} & -1.57563\sym{***} & \\\hline\hline
	Paramètres & $\theta_h$ & & -0.1320 & & & 0.0240 & \\
	& $\rho$ & -2.53111\sym{***} & -1.86166\sym{***} & -5.91528\sym{***} & -4.4018\sym{***} & -3.46626\sym{***} & -19.8039\sym{***}\\
		& $\delta$ & -1.788\sym{***} & -0.69603\sym{***} & -3.17707\sym{***}& -2.18242\sym{***} & -1.79454\sym{***} & -5.10002\sym{***} \\\cline{2-8}
		& $\theta_u$ & & 2.45323\sym{***} & & & 5.73085\sym{***} &\\
		& $\theta_v$ & & 0.02561 & & & 1.54131\sym{***} &\\\hline
	Nombre d'observations (N) &&& 8799 &&& 7304 &\\\hline
	Log vraisemblance &&& \text{-}10103,22 &&& \text{-}2752,11 &\\\hline\hline
%	\multicolumn{8}{l}{Significativité  ; \sym{*} $p<0,10$, \sym{**} $p<0,05$, \sym{***} $p<0,01$. Pour des problèmes de convergence liés à des effectifs trop faibles (taux d'aide personnelle formelle ou informelle}\\
%	\multicolumn{8}{l}{inférieur à 1 \%) le Danemark, Malte, la Suède et la Suisse ont été écartés.}\\
\end{tabular}
\begin{tablenotes}[para,flushleft] \footnotesize
\item Significativité : \sym{*} $p<0,10$, \sym{**} $p<0,05$, \sym{***} $p<0,01$. Pour des problèmes de convergence liés à des effectifs trop faibles (taux d'aide personnelle formelle ou informelle inférieur à 1 \%) le Danemark, Malte, la Suède et la Suisse ont été écartés.
\end{tablenotes}
\end{threeparttable}
}
\end{table}

Les estimations associées à l'aide domestique (cf. \autoref{resultats}) sont comparables à celles obtenues pour l'aide non-différenciée. Ce résultat s'explique par le fait que la majorité des personnes recevant de l'aide personnelle reçoivent aussi de l'aide domestique (cf. \autoref{stat_des_care}). Les coefficients associés aux variables d'aide domestique sont négatifs et significatifs. Parce que l'aide domestique (qu'elle soit formelle ou informelle) conduit à la même production (ménage, repassage, repas, etc.), ces deux aides peuvent être considérées comme des biens substituts. Il en est de même pour les coefficients associés aux variables d'aide personnelle qui eux aussi sont négatifs et significatifs (cf. \autoref{resultats}) : recevoir de l'aide personnelle informelle diminue significativement à la fois la probabilité de recevoir de l'aide formelle personnelle et le nombre d'heures d'aide formelle personnelle reçues ; heures qui ont elles-mêmes un impact négatif sur la probabilité de recevoir de l'aide informelle personnelle.

Le \autoref{resultats} montre que les aides domestique et personnelle ne sont pas totalement expliquées par le même ensemble de caractéristiques (individuelles et liées à la santé). Les femmes reçoivent moins d’aide formelle (domestique ou personnelle) que les hommes. La probabilité de recevoir de l’aide (domestique ou personnelle) augmente avec l’âge mais, seules les heures d’aide domestique sont positivement corrélées à l’âge. Le nombre d’enfants est corrélé négativement à l’aide formelle domestique et positivement à l’aide informelle domestique, mais il n'a aucun impact sur les heures ou sur l’aide personnelle (formelle ou informelle). Plus le niveau d’éducation de la personne est élevé plus la probabilité qu’elle reçoive de l’aide formelle domestique (et dans une moindre mesure personnelle) est forte alors que sa probabilité de recevoir de l’aide informelle personnelle (et dans une moindre mesure domestique) est faible. Notons que le revenu n’a pas d’effet significatif sur les aides ou les heures. Avoir beaucoup de personnes de confiance augmente les probabilités de recevoir de l’aide domestique informelle, dans une moindre mesure, de l'aide domestique formelle\footnote{Peut-être que les personnes de confiance fournissent de l'information sur les aides institutionnelles et aident pour les démarches administratives.} et de l'aide personnelle informelle.

Les ADL déclarées et l’aide à la marche augmentent la probabilité de recevoir tous les types d'aides. Ces probabilités et le nombre d’heures sont significativement plus élevées lorsque l’ADL est associée à des problèmes d’ablutions. Les problèmes de mobilité et les maladies chroniques augmentent les probabilités de recevoir de l'aide domestique. Seule la probabilité de recevoir de l’aide personnelle informelle est corrélée positivement avec les problèmes de mobilité. Être en mauvaise santé mentale augment les probabilités de recevoir de l’aide domestique et de l’aide personnelle informelle. Enfin, le sentiment de limitation est corrélé avec les différents types d'aides~: plus les personnes se sentent limitées, plus la probabilité de recevoir de l'aide est forte. Notons que seul le nombre d’heures d’aide domestique formelle augmente avec le GALI.

%
%Lorsque l'on compare les estimations associées à l'aide domestique (cf. \autoref{resultats}) et à l'aide personnelle (cf. \autoref{resultats}) des différences concernant l'effet de variables explicatives peuvent être mises en évidence.
%
%L'âge a un effet très significativement positif sur la probabilité de recevoir de l'aide, qu'elle soit formelle ou informelle, domestique ou personnelle. En revanche, il n'a pas d'effet sur le nombre d'heures d'aide formelle personnelle alors qu'il a un effet significativement positif sur le nombre d'heures d'aide formelle domestique. Les femmes ont une probabilité d'utiliser de l'aide formelle, domestique ou personnelle, significativement plus faible que les hommes. Cet effet est plus significatif pour l'aide domestique.
%
%Les ADL déclarées n'ont pas d'effet sur les heures d'aide formelle domestiques reçues lorsque l'on ne considère pas les problèmes liés aux ablutions. Les maladies chroniques n'ont pas d'effet sur les variables associées à l'aide personnelle (formelle ou informelle, y compris les heures). Les problèmes de mobilité augmentent la probabilité de recevoir de l'aide formelle domestique mais pas personnelle.
%
%Les probabilités d'avoir de l'aide personnelle ne sont pas corrélées avec le sentiment de solitude. Une mauvaise santé mentale affecte positivement la probabilité de recevoir de l'aide formelle qu'elle soit domestique ou personnelle et celle de recevoir de l'aide personnelle informelle. Avoir plus de quatre personnes de confiance augmente la probabilité de recevoir de l'aide domestique informelle et dans une moindre mesure la probabilité de recevoir de l'aide domestique formelle. Le sentiment de solitude n'augmente que la probabilité de recevoir de l'aide domestique formelle.
%
%Les limitations perçues, GALI, affectent positivement la probabilité de recevoir de l'aide formelle qu'elle soit domestique ou personnelle et aussi le nombre d'heures d'aide formelle domestique. Contrairement à l'étude sur les aides agrégées, le GALI n'affecte pas positivement toutes les heures d'aide formelle. Les heures d'aide formelle personnelle sont indépendantes du GALI.

%%%%%%%%%%%%%%%%%%%%%%%%%%%%%%%%%%%%%%%%%%%%%%%%%%%%%
						\section{Robustesse}
%%%%%%%%%%%%%%%%%%%%%%%%%%%%%%%%%%%%%%%%%%%%%%%%%%%%%
%\begin{landscape}
\begin{table}[t]
\caption{Robustesse}\label{robustesse}
\centering
\subcaption{Différents modèles}
\label{robustesseA}
\resizebox{\textwidth}{!}{
		\sisetup{
			parse-numbers = true,
			output-decimal-marker={,},
			%table-number-alignment = left,
			%table-text-alignment = right,
			table-alignment-mode = none,
			%negative-color = red,
			%input-open-uncertainty = ,
			%input-close-uncertainty = ,
			%minimum-decimal-digits = 3,
			%table-format = +5.3,
			%table-align-text-before = false,
			%table-align-text-after = false,
			%round-mode = figures,
			%round-precision=3,
			%round-pad = false,
		}\renewcommand{\arraystretch}{1.5}
\begin{threeparttable}
\begin{tabular}{l@{\;} S!{\qquad}S!{\qquad}S!{\qquad}S!{\qquad}S!{\qquad}S!{\qquad}S!{\qquad}S!{\qquad}}%{l|c|c|c|c|c|c|c}
\toprule\midrule
& {$(1)$} & {$(2)$} & {$(3)$} & {$(4)$} & {$(5)$} & {$(6)$} & {$(7)$} & {$(8)$} \\\hline
Caractéristiques individuelles\tnote{\textbf{a}} & oui & oui\tnote{\textbf{b}} & oui & oui & oui & oui & oui &oui \\
Caractéristiques de santé physique\tnote{\textbf{c}} & oui & oui & oui & oui & oui & oui & oui & oui \\
Caractéristiques de la zone géographique\tnote{\textbf{d}} & oui & oui & oui & oui & oui & oui & non &oui\tnote{\textbf{e}}\\
Caractéristiques de santé psychologique\tnote{\textbf{f}} & oui & oui & non & oui & oui & oui & oui & oui \\
Âge & {linéaire} & {linéaire} & {linéaire} & {linéaire} & {$+75$ ans} & {quadratique} & {linéaire} & {linéaire}\\
GALI & oui & oui & oui & non & oui & oui & oui & oui\\\hline
Modèle AF-AI & & & & & & & & \\
AI équation AF & -2.037*** & -2.091*** & -2.001*** & -1.638*** & -1.179*** & -1.981*** & -1.437*** & -1.911***\\
AI équation hAF & -1.344*** & -1.336*** & -1.345*** & -1.071*** & -0.803*** & -1.303*** & -0.201*** & -1.268***\\
hAF équation AI & -0.078*** & -0.078*** & -0.07*** & -0.056*** & -0.031*** & -0.069*** & -0.02*** & -0.081***\\
LL & -10591.8 & -10594 & -10632.8 & -10665.3 & -8038.8 & -10596.1 & -11273.2 & -7050.6\\

Modèle ADF-ADI & & & & & & & & \\
ADI équation ADF & -1.638*** & -1.67*** & -1.729*** & -1.396*** & -1.947*** & -1.634*** & -1.507*** & -1.484***\\
ADI équation hADF & -1.035*** & -1.046*** & -1.099*** & -0.845*** & -1.21*** & -1.028*** & -0.313*** & -0.834***\\
hADF équation ADI & -0.064*** & -0.064*** & -0.065*** & -0.073*** & -0.091*** & -0.066*** & -0.027*** & -0.069***\\
LL & -10103.2 & -10106.1 & -10142.6 & -10168.5 & -7620.1 & -10137.4 & -10747.3 & -6725.85 \\\hline
Effectif & 8799 & 8799 & 8799 & 8799 & 5160 & 8799 & 8799 & 5666\\\hline

Modèle APF-API & & & & & & & & \\
API équation APF & -1.369*** & -1.387*** & -1.256*** & -1.618*** & -1.886*** & -1.392*** & -2.227*** & -1.306*** \\
API équation hAPF & -1.576*** & -1.59*** & -1.62*** & -1.74*** & -1.648*** & -1.536*** & -2.216*** & -1.580***\\
hAPF équation API & -0.105*** & -0.106*** & -0.097*** & -0.13*** & -0.157*** & -0.08*** & -0.093*** & -0.063***\\
LL & -2752.1 & -2753.4 & -2769.3 & -2774.7 & -2185.3 & -2758.4 & -2909.8 & -1770.72\\\hline
Effectif & 7304 & 7304 & 7304 & 7304 & 4292 & 7304 & 7304 & 4665\\\hline
\bottomrule
\end{tabular}
\begin{tablenotes}
	\item[a] Genre, nombre d'enfants, log revenu, diplôme.
	\item[b] Nombre d'enfants utilisé seulement pour expliquer l'aide informelle. Log revenu utilisé seulement pour expliquer l'aide formelle et les heures d'aide formelle.
	\item[c] ADL, mobilité, aide à la marche, maladie chronique.
	\item[d] Localité, pays.
	\item[e] Uniquement les pays où le taux de risque de pauvreté est inférieur à 18 \%.
	\item[f] Sentiment de solitude, santé mentale, contacts sociaux.
\end{tablenotes}
\end{threeparttable}
}
\subcaption{La variable âge}\label{robustesseB}
\resizebox{\textwidth}{!}{
		\sisetup{
			%parse-numbers = true,
			output-decimal-marker={,},
			%table-number-alignment = format,
			%table-text-alignment = center,
			%table-alignment  = center,
			%table-alignment-mode = marker,
			%negative-color = red,
			%minimum-decimal-digits = 0,
			table-format = +5.6,
			%table-align-text-before = false,
			table-align-text-after = false,
			round-mode = places,
			round-precision=3,
			round-pad = false,
		}
\begin{threeparttable}\renewcommand{\arraystretch}{1.5}
\begin{tabular}{l@{}l@{}S@{}S@{}S|@{}S@{}S@{}S|@{}S@{}S@{}S@{}}%{llccc|ccc|ccc}
\toprule\midrule
Modèle && \multicolumn{3}{c|}{Aide} & \multicolumn{3}{c|}{Aide domestique} & \multicolumn{3}{c}{Aide personnelle}\\
&& \multicolumn{1}{c}{formelle}& \multicolumn{1}{c}{heures formelles}& \multicolumn{1}{c|}{informelle}&{formelle} & {heures formelles}&{informelle}&{formelle}&{heures formelles}&{informelle}\\\midrule
 (1)& âge & 0.08386*** & 0.03234*** & 0.14053*** & 0.07918*** & 0.03039*** & 0.082*** & 0.04599*** & 0.00905 & 0.05893*** \\
 (2)& âge & 0.08468*** & 0.03255*** & 0.14059*** & 0.08003*** & 0.03085*** & 0.08357*** & 0.0462*** & 0.00938 & 0.05886*** \\
 (3)& âge & 0.08401*** & 0.03191*** & 0.11914*** & 0.08111*** & 0.03083*** & 0.08431*** & 0.04623*** & 0.01209 & 0.06145*** \\
 (4)& âge & 0.08419*** & 0.0315*** & 0.08581*** & 0.0803*** & 0.03186*** & 0.0753*** & 0.0455*** & 0.00707 & 0.05428*** \\
 (5)& âge & 0.07466*** & 0.04122*** & 0.07573*** & 0.07635*** & 0.04206*** & 0.14474*** & 0.05658*** & 0.01449 & 0.10186*** \\\hline
 \multirow{ 2}{*}{(6)\tnote{a}}& âge & 0.10688* & 0.15324** & 0.16612 & 0.13302* & 0.16278** & -0.09249 & -0.35349*** & 0.20914 & -0.47441** \\
 & (âge)$^2/100$ & -0.01472 & -0.07564* & -0.03065 & -0.03381 & -0.08193 & 0.10723 &0.24946*** & -0.12073 & 0.33924*** \\\hline
  (7) & âge & 0.0776*** & 0.00425 & 0.04239*** & 0.07561*** & 0.00638 & 0.04366*** & 0.06048*** & 0.02994*** & 0.10419*** \\\hline
  (8) & âge & 0.082*** & 0.034*** & 0.133*** & 0.077*** & 0.028*** & 0.069*** & 0.049*** & 0.017 & 0.041*** \\\hline
\bottomrule
\end{tabular}
\begin{tablenotes}
	\item[a] Modèle avec la variable âge et la variable âge au carré.
\end{tablenotes}
\end{threeparttable}
}
\end{table}%\FloatBarrier
%\end{landscape}
Le modèle~(1) dans le \autoref{robustesseA} est le modèle détaillé précédemment (cf. \autoref{res_fcic} et \autoref{resultats}). Dans le modèle~(2), des restrictions d'exclusion sont imposées. Nous utilisons un indicateur de disponibilité -offre potentielle-, le nombre d'enfants, comme instrument pour la probabilité de recevoir de l'aide informelle. Le nombre d'enfants est exclu dans l'estimation de la probabilité de recevoir de l'aide formelle et du nombre d'heures. Un indicateur de renoncement potentiel, le revenu, est utilisé comme instrument pour estimer la probabilité de recevoir de l'aide formelle et le nombre d'heures. Il est donc exclu pour l'estimation de la probabilité de recevoir de l'aide informelle. Le nombre d'enfants est un déterminant fortement significatif, de signe positif, dans l'équation de l'aide informelle, conformément à l'intuition. De même, le revenu impacte positivement mais de manière faiblement significative le recours à l'aide formelle mais n'a pas d'influence significative sur le nombre d'heures d'aide formelle reçue. Le revenu n'a plus d'effet lorsque l'on désagrège notre modèle suivant que l'aide est domestique ou personnelle. Le nombre d'enfant conserve son effet significatif sur l'aide informelle domestique mais n'a aucun impact sur l'aide informelle personnelle. Cette instrumentation ne modifie ni le signe ni la significativité des coefficients de nos variables d'intérêt et cela modifie que très légèrement la valeur des coefficients. Dans le modèle~(3), nous avons exclu les caractéristiques de \enquote{santé psychologique} (sentiment de solitude, santé mentale et nombre de contacts sociaux), ces variables pouvant être suspectées de causalité inverse. Cette exclusion ne modifie ni le signe ni la significativité des coefficients de nos variables d'intérêt, et cela modifie que très légèrement la valeur des coefficients. Si l'on exclut alternativement la variable GALI, modèle~(4), les individus de moins de 75 ans\footnote{Nous avons dans ce modèle exclu les \enquote{young-old} classe définie par \textcite{neugarten1974} et que \textcite{eurostat2020} nomme \enquote{older people} par opposition aux \enquote{very old people} dans la dernière version du rapport Eurostat, Ageing Europe.}, modèle~(5), les caractéristiques de la zone géographique -localité et pays- modèle~(7), le signe et la significativité des coefficients sont inchangés. La substituabilité entre les aides n'est pas invalidée même si l'on considère un sous échantillon plus âgé, ou que l'on ne tient pas compte du pays et de la localité de résidence de l'aidé, ou encore si l'on ne tient pas compte des limitations déclarées. Notons que lorsque l'on exclut les caractéristiques de la zone géographique la variable âge a un effet positif et fortement significatif sur le nombre d'heures d'aide personnelle formelle (cf. \autoref{robustesseB}). Les caractéristiques institutionnelles, les politiques dédiées ou encore les caractéristiques des marchés concernés sont différenciées suivant la zone géographique, voir par exemple \textcite{rostgaard2011}. Ne pas tenir compte de cette information, dans notre cas, conduirait à conclure que l'âge a un effet sur le nombre heures d'aide personnelle formelle alors qu'il n'en n'a pas en moyenne. Dans le modèle~(6), la variable âge est introduite de manière quadratique. Cela ne modifie ni le signe ni la significativité des coefficients de nos variables d'intérêt. En revanche, si l'on considère seulement l'aide personnelle (cf. \autoref{robustesseB}) l'âge a un impact quadratique ce qui n'est ni le cas lorsque l'on considère l'aide domestique ou l'aide agrégée. Par souci de lisibilité dans le \autoref{resultats} la variable âge est introduite de manière linéaire même pour l'étude sur l'aide personnelle\footnote{Les 24 tableaux complets correspondant à ces tests de robustesse sont disponibles sur demande.}. Pour finir, dans le modèle~(8) nous avons exclu les pays où le taux de risque de pauvreté\footnote{Défini comme la part des personnes ayant un revenu disponible équivalent (après transferts sociaux) inférieur au seuil de risque de pauvreté, fixé à 60 \% du revenu disponible équivalent médian national après transferts sociaux.}, calculé sur la population des plus de 65~ans, est supérieur\footnote{Ce seuil, 18 \%, est le taux moyen sur l'ensemble des pays européens. On a donc retiré~: la Bulgarie, Chypre, la Croatie, l'Estonie, la Hongrie, la Lettonie, la Lituanie, Malte et la Roumanie} à 18 \%. Pour ce faire nous avons utilisé les taux calculés par \textcite{ebbinghaus2021} sur les données EU-SILC (2017/18), rappelons que les données SHARE que l'on utilise ont été collectées en 2019. La substituabilité entre les aides n'est pas invalidée même si l'on considère un sous échantillon de pays où le pourcentage de personnes âgées \enquote{pauvres} est relativement faible. Ce taux de risque de pauvreté est calculé en tenant compte des transferts sociaux c'est donc une proxy de la \enquote{générosité} des politiques nationales en faveur des personnes âgées.

%%%%%%%%%%%%%%%%%%%%%%%%%%%%%%%%%%%%%%%%%%%%%%%%%%%%%
						\section{Conclusion}
%%%%%%%%%%%%%%%%%%%%%%%%%%%%%%%%%%%%%%%%%%%%%%%%%%%%%

L'objectif de cette recherche est double. Tout d'abord, il s'agit d'identifier les caractéristiques qui ont un impact sur les probabilités de recevoir de l'aide à domicile, ainsi que sur le nombre d'heures d'aide formelle reçu. Ensuite, nous cherchons à qualifier la relation entre les aides formelles et informelles : sont-elles substituables ou complémentaires ? Afin de répondre à cette question, nous avons jugé essentiel de considérer des services relativement homogènes. Nous avons donc effectué une décomposition des aides formelles et informelles suivant qu'elles sont domestiques ou personnelles. De même, les heures d'aide formelle reçues ont été décomposées en heures d'aide formelle domestique et personnelle.

En ce qui concerne les déterminants associés aux probabilités de recevoir une aide (qu'elle soit formelle, informelle, domestique ou personnelle), il n'est pas surprenant de constater que les personnes les moins autonomes physiquement (ayant des difficultés à effectuer leur toilette, des problèmes de mobilité ou utilisant des aides à la marche) sont celles qui ont les probabilités les plus élevées de recevoir de l'aide, qu'elle soit formelle, informelle, domestique ou personnelle. Ce dernier type d'aide revêt une importance particulière pour ces personnes. Le niveau de limitation déclaré par l'aidé constitue également un déterminant important de la probabilité de recevoir toutes les formes d'aide mais n'a pas d'effet sur le nombre d'heures d'aide personnelle reçu.

L’analyse du nombre d’heures d’aide formelle reçu montre que ces heures sont significativement plus importantes pour les personnes les moins autonomes (difficultés pour effectuer leur toilette, utilisation de matériel d’aide à la marche). Les maladies chroniques n’ont pas d’effet particulier sur le nombre d’heures en revanche plus le sentiment de limitation et la perception de ces limitations sont forts plus le nombre d’heures d’aide formelle reçu sera élevé. Cette analyse nous permet de plus d'estimer l'interaction entre les aides formelles et informelles. Nos résultats montrent de manière systématique une interaction négative et très significative : que ce soit pour l'ensemble des aides, les aides domestiques ou les aides personnelles, le fait de bénéficier d'une aide informelle domestique (respectivement personnelle) diminue la probabilité de recevoir de l'aide formelle domestique (respectivement personnelle), et inversement, un plus grand nombre d'heures d'aide formelle domestique (respectivement personnelle) reçu diminue la probabilité de recevoir de l'aide informelle domestique (respectivement personnelle). Ainsi, les deux types d'aide (formelle et informelle) semblent être substituables. Ce résultat, bien qu'intuitif, n'a jamais été obtenu de manière aussi nette à notre connaissance. Certains travaux ont même mis en évidence une complémentarité entre certaines formes d'aide.

Nous sommes conscients que nos résultats peuvent être influencés par plusieurs facteurs. Premièrement, notre analyse se base uniquement sur les aides informelles régulières, c'est-à-dire celles fournies au moins une fois par semaine. Nous avons choisi de privilégier cette fréquence pour mieux étudier le problème du maintien à domicile des personnes âgées. Deuxièmement, l'absence de données sur le nombre d'heures d'aide informelle constitue une limitation de notre étude. Cette information n'est plus disponible dans les enquêtes SHARE depuis plusieurs vagues. Il est probable que sa fiabilité ait été remise en question, étant donné la difficulté pour une personne âgée - ou même pour quiconque - de se souvenir même approximativement des heures d'aide non contractuelles fournies par son entourage, en particulier sur une période de plusieurs mois. Cette lacune dans nos données constitue une limite à notre travail. Si nous disposions de données fiables sur les heures d'aide informelle, nous aurions pu mesurer le taux marginal de substitution entre les heures d'aide formelle et informelle. De plus, nous ne disposons pas de données sur les politiques publiques ou sur les caractéristiques culturelles spécifiques à chaque pays en matière de maintien à domicile des personnes âgées. Cette absence constitue une autre limitation de nos résultats. La prise de compte des caractéristiques institutionnelles pour l’aide formelle et des caractéristiques culturelles (comme les traditions familiales pour l’aide informelle) permettrait surement d’affiner les résultats.

Ce travail permet de repérer les principales caractéristiques qui impactent le fait de recevoir de l’aide formelle ou informelle. Nos résultats montrent deux choses. D’une part, il est essentiel de tenir compte de la nature spécifique de l’aide (domestique ou personnelle) et, d’autre part, les aides sont clairement des services substituables. Toutefois, il est important de noter que cette substituabilité n’empêche pas que certaines personnes âgées recourent à la fois à l'aide formelle et à l'aide informelle, que ce soit au niveau domestique ou personnel. Plusieurs explications sont possibles. L’aide informelle reçue est insuffisante (problème de disponibilité de l’aidant et donc coût d'opportunité élevé) ou l’aide formelle reçue est trop onéreuse (prix relatif élevé). Les prix (coûts d'opportunité) ne peuvent à eux seuls expliquer le choix de la personne âgée. Ce choix dépend aussi de ses préférences. Une personne âgée peut préférer être aidée par un membre de son entourage plutôt que par une personne étrangère à celui-ci ou inversement faire appel à un professionnel pour ne pas \enquote{déranger} un de ses proches. Dans le premier cas si l'entourage n'est pas disponible et dans le second cas si l'offre d'aide formelle n'est pas accessible, la personne âgée peut sous-consommer l'aide disponible. \'Etant donné ses capacités objectives, elle pourrait prendre le risque de perdre son autonomie future en ne délégant pas certaines tâches domestiques dont l'exécution est au-dessus de ces capacités physiques ou cognitives. Avoir une meilleure compréhension des préférences des séniors permettrait sûrement de mieux adapter l’aide dont ils ont besoin pour les maintenir à domicile le plus longtemps possible.

%
%
%Dans ces cas les politiques modifiant les prix relatifs n'auraient que peu d'impact.
%
%
%Par exemple, une personne âgée peut avoir une forte désutilité à être aidée, préférence pour l'autonomie \enquote{instantanée}, et étant donné ses capacités objectives, prendre le risque de perdre son autonomie future en ne délégant pas certaines tâches domestiques dont l'exécution est au-dessus de ces capacités physiques ou cognitives.
%
%Cette \enquote{myopie} aura en plus des coûts pour la société des conséquences négatives sur son bien-être futur. 
%

%%%%%%%%%%%%%%%%%%%%%%%%%%%%%%%%%%%%%%%%%%%%%%%%%%%%%
%%%%%%%%%%%%%%%%%%%%%%%%%%%%%%%%%%%%%%%%%%%%%%%%%%%%%
%%%%%%%%%%%%%%%%%%%%%%%%%%%%%%%%%%%%%%%%%%%%%%%%%%%%%
\begingroup\sloppy\setlength{\emergencystretch}{3em}
\printbibliography
\endgroup
%%%%%%%%%%%%%%%%%%%%%%%%%%%%%%%%%%%%%%%%%%%%%%%%%%%%%

%%%%%%%%%%%%%%%%%%%%%%%%%%%%%%%%%%%%%%%%%%%%%%%%%%%%%
\begin{appendices}
%%%%%%%%%%%%%%%%%%%%%%%%%%%%%%%%%%%%%%%%%%%%%%%%%%%%%
%%%%%%%%%%%%%%%%%%%%%%%%%%%%%%%%%%%%%%%%%%%%%%%%%%%%%

\section{Construction des variables d’aide formelle informelle}
\label{Construction des variables d’aide formelle informelle}

Questions utilisées pour définir les variables d’aide formelle et informelle~:

\begin{description}
\item[Aide formelle] « Au cours des douze derniers mois, avez-vous reçu à domicile un professionnel ou un des services rémunérés suivants en raison d’un problème de santé d’ordre physique, mental, émotionnel ou de mémoire ? »
\begin{enumerate}
\item Aide pour des besoins personnels (ex. : se lever de ou se coucher dans un lit, s'habiller, se laver).
\item Aide pour des tâches ménagères (ex. : ménage, repassage, cuisine).
\item Repas à domicile (c'est-à-dire repas prêts à la consommation apportés par une structure publique ou privée).
\item Aide pour d'autres activités (ex. : prise de médicaments, soins infirmiers).
\item Aucunes de celles-ci
\end{enumerate}

La personne peut répondre oui ou non pour chacune des modalités (1 à 5). Puis, pour chacune des modalités dont la réponse est oui, la question suivante est posée : « Pendant combien d’heures par semaine, en moyenne, avez-vous reçu à domicile de telles aides pour vos propres besoins~? ».
Pour créer nos variables d’aide formelle nous avons considéré les réponses aux modalités 1) et 2).
\begin{itemize}
\item Une personne reçoit de l’aide formelle domestique si elle a répondu oui à la modalité 2 (aide pour les tâches ménagères). Le nombre moyen d’heures d’aide domestique reçu par semaine est alors connu.
\item Une personne reçoit de l’aide formelle personnelle si elle a répondu oui à la modalité 1 (aide pour les besoins personnels). Le nombre moyen d’heures d’aide personnelle reçu par semaine est alors connu.
\item Une personne reçoit de l’aide formelle si elle a répondu au moins oui à l’une des deux aides décrites ci-dessus (formelle domestique et/ou personnelle). Le nombre moyen d’heures aide est défini comme la somme des heures moyenne d’heures d’aide domestique et personnelle reçues par semaine.
\end{itemize}

\item[Aide informelle] « Pensez aux douze derniers mois. Un membre de votre famille, extérieur à votre ménage, un ami ou un voisin vous a-t-il apporté un type d’aide mentionné ci-dessous ? »
\begin{enumerate}
\item Des soins personnels, comme s'habiller, se laver ou se doucher, manger, aller ou sortir du lit, utiliser les toilettes.
\item Des aides pour le ménage, ou des tâches ménagères, du bricolage, du jardinage.
\item Une aide administrative, comme remplir des formulaires, s'occuper des questions financières ou légales.
\end{enumerate}

La personne peut répondre oui ou non pour chaque modalité (1 à 3). Pour chaque modalité, la question suivante est posée~: « Quel membre de votre famille ne faisant pas partie de votre ménage, ami ou voisin, vous a principalement aidé(e) durant les douze derniers mois~? » 
Pour la personne citée, la question est ensuite « Durant les douze derniers mois, à quelle fréquence avez-vous reçu l’aide de cette personne~? »
\begin{enumerate}
\item Tous les jours ou presque.
\item Toutes les semaines ou presque.
\item Tous les mois ou presque.
\item Moins souvent.
\end{enumerate}

Pour construire les variables d’aide informelle nous avons combiné les réponses au type d’aide et à la fréquence.
\begin{itemize}
\item Une personne bénéficie d’aide informelle domestique si elle a répondu oui à la modalité 2) et si pour ce type d’aide elle a répondu a) ou b).
\item Une personne bénéficie d’aide informelle personnelle si elle a répondu oui à la modalité 1) et si pour ce type d’aide elle a répondu a) ou b).
\item Une personne bénéficie d’aide informelle si elle bénéficie d’au moins une des deux aides (domestique ou personnelle) au sens où nous venons de les définir.
\end{itemize}

Vivre seul recoupe plusieurs situations : célibataire, veuf ou séparé (avec ou sans enfants). La question posée sur l’aidant permet de dire que 68\% des aidés répondent qu’au moins un enfant au sens large (enfant biologique ou pas, gendre ou belle-fille, petit-enfant) les aide. Pour ceux ne répondant pas un enfant les réponses qui reviennent le plus sont voisins, amis, proches de la personne aidée (frères ou sœurs, neveux ou nièces, petits-neveux ou nièces).
\end{description}

\newpage
%%%%%%%%%%%%%%%%%%%%%%%%%%%%%%%%%%%%%%%%%%%%%%%%%%%%%%%%
\section{Statistiques descriptives}\label{Tableaux}
\begin{table}[!h]
	\centering
	\caption{Principales statistiques descriptives - Variables quantitatives}
	\sisetup{
		% parse-numbers = true,
		output-decimal-marker={,},
		table-alignment-mode = format,
		%  table-number-alignment = right,
		% table-text-alignment = right,
		%table-alignment  = right,
		negative-color = red,
		input-open-uncertainty = ,
		input-close-uncertainty = ,
		minimum-decimal-digits = 0,
		group-digits = integer,
		group-minimum-digits = 4,
		table-format = 6.1,
		table-align-text-before = false,
		table-align-text-after = false,
		round-mode = places,
		round-precision=3,
		round-pad = false,
		round-zero-positive = false,
		drop-zero-decimal = false,
	}
	\rotatebox{90}{
	\begin{tabular}{l@{\;}l@{\;}S!{}S!{}S!{}S!{}S!{}S!{}S!{}S!{}}
		\hline\hline
		Variable & & \multicolumn{1}{c}{Ens.} & \multicolumn{1}{c}{AF} & \multicolumn{1}{c}{AI} & \multicolumn{1}{c}{ADF} & \multicolumn{1}{c}{ADI} & \multicolumn{1}{c}{APF} & \multicolumn{1}{c}{API} & \multicolumn{1}{c}{NoA}\\\hline
		\^Age & minimum & 65 & 65 & 65 & 65 & 65 & 65 & 65 & 65 \\
		 & maximum & 100 & 100 & 100 & 100 & 100 & 98 & 96 & 96 \\
		 & moyenne & 76.9 & 81.7 & 80.6 & 81.6 & 80.6 & 83.2 & 82.2 & 75.2 \\
		 & médiane & 76 & 82 & 81 & 82 & 81 & 84 & 84 & 75 \\
		 & écart-type & 7.4 & 7.5 & 7.4 & 7.4 & 7.5 & 7.8 &7.3 & 6.8 \\\hline
		Nombre d'enfants & minimum & 0 & 0 & 0 & 0 & 0 & 0 & 0 & 0 \\
		 & maximum & 14 & 9 & 11 & 9 & 11 & 8 & 7& 14 \\
		 & moyenne & 1.9 & 1.9 & 2.1 & 1.9 & 2.1 & 2.0 &2.1 & 1.9 \\
		 & médiane & 2 & 2 & 2 & 2 & 2 & 2 & 2 & 2 \\
		 & écart-type & 1.3 & 1.4 & 1.4 & 1.4 & 1.4 & 1.5 & 1.4 & 1.3 \\\hline
		Revenu & minimum & 1 & 1 & 1 & 1 & 1 & 1 & 1 & 1 \\
		 & maximum & 425691 & 425691 & 180062 & 425691 & 180062 & 194501 & 159621 & 343556 \\
		 & moyenne & 15518 & 18930 & 12941 & 18939 & 12844 & 16837 & 11514 & 15434 \\
		 & médiane & 11500 & 14067 & 9763 & 14032 & 9724 & 13298 & 8817 & 11395 \\
		 & écart-type & 18312 & 23718 & 14048 & 23748 & 13866 & 19072 & 14365 & 17482 \\\hline
		Logarithme du revenu & minimum & 0 & 0 & 0 & 0 & 0 & 0 & 0 & 0 \\
		 & maximum & 12.9 & 13.0 & 12.1 & 12.9 & 12.1 & 12.2 & 12.0 & 12.7 \\
		 & moyenne & 9.2 & 9.4 & 9.1 & 9.4 & 9.1 & 9.3 & 9.0 & 9.3 \\
		 & médiane & 9.3 & 9.6 & 9.2 & 9.5 & 9.2 & 9.5 & 9.1 & 9.3 \\
		 & écart-type & 1.2 & 1.2 & 1.1 & 1.2 & 1.1 & 1.2 & 1.1 & 1.2 \\\hline
		Nombre d'observations && 8 799 & 1 561 & 1 638 & 1 516 & 1 596 & 426 & 358 & 6 100 \\\hline \hline
	\end{tabular}
	}
	\label{quanti_variablesA}
\end{table}

%%%%%%%%%%%%%%%%%%%%%%%%%%%%%%%%%%%%%%%%%%%%%%%%%%%%%%
\begin{table}[!h]
	\centering
	\caption{Principales statistiques descriptives - Variables qualitatives}
	\resizebox{\dimexpr\textwidth-2\fboxsep-2\fboxrule}{!}{%
	\sisetup{
		% parse-numbers = true,
		output-decimal-marker={,},
		%  table-number-alignment = right,
		% table-text-alignment = right,
		table-alignment  = right,
		negative-color = red,
		input-open-uncertainty = ,
		input-close-uncertainty = ,
		minimum-decimal-digits = 0,
		group-digits = integer,
		group-minimum-digits = 4,
		table-format = 2.1,
		table-align-text-before = false,
		table-align-text-after = false,
		round-mode = places,
		round-precision=1,
		round-pad = false,
		round-zero-positive = false,
		drop-zero-decimal = false,
	}
		\begin{tabular}{l@{\;}l@{\;}S!{}S!{}S!{}S!{}S!{}S!{}S!{}S!{}}
		\hline\hline
		Variable & & \multicolumn{1}{c}{Ens.} & \multicolumn{1}{c}{AF} & \multicolumn{1}{c}{AI}& \multicolumn{1}{c}{ADF}& \multicolumn{1}{c}{ADI}& \multicolumn{1}{c}{APF}& \multicolumn{1}{c}{API}& \multicolumn{1}{c}{NoA}\\ \hline
		Sexe & homme & 25.1 & 27.2 & 18.6 & 27.2 & 18.2 & 24.9 & 18.4 & 26.0 \\
		 & femme & 74.9 & 72.8 & 81.4 & 72.8 & 81.8 & 75.1 &81.6 & 74.0 \\\hline
		Niveau d'éducation & sans diplôme & 4.6 & 6.8 & 7.1 & 6.7 & 6.9 & 10.1 &12.0 & 3.6 \\
		 & inférieur au bac & 36.8 & 41.2 & 46.9 & 41.1 & 47.2 & 46.5 &50.6 & 33.8 \\
		 & niveau bac & 32.3 & 26.8 & 29.0 & 26.9 & 29.1 & 24.7 &25.4 & 33.8 \\
		 & études supérieure & 25.8 & 24.3 & 16.3 & 24.5 & 16.2 & 16.9 & 11.5 & 28.3 \\
		 & non renseigné & 0.6 & 0.9 & 0.6 & 0.9 & 0.6 & 1.9 & 0.6 & 0.5 \\\hline
		Zone résidentielle & très grande ville & 18.3 & 18.8 & 15.2 & 18.9 & 15.4 & 20.2 &16.8 & 18.9 \\
		 & banlieue & 9.0 & 11.5 & 7.8 & 11.5 & 7.6 & 10.6 &6.4 & 8.8 \\
		 & grande ville & 17.2 & 16.9 & 14.9 & 16.8 & 14.7 & 17.1 & 15.6 & 17.7 \\
		 & petite ville & 21.3 & 23.2 & 22.5 & 23.3 & 22.8 & 22.5 & 19.8 & 20.8 \\
		 & zone rurale & 26.5 & 23.1 & 31.9 & 23.0 & 31.8 & 23.2 & 35.5 & 25.9 \\
		 & non renseigné & 7.7 & 6.4 & 7.7 & 6.5 & 7.7 & 6.3 & 5.9 & 8.0 \\\hline
		ADL déclarées & aucune & 83.7 & 61.4 & 64.4 & 62.1 & 64.5 & 21.1 &31.0 & 91.4 \\
		 & oui, mais pas bain et douche & 7.3 & 12.2 & 12.0 & 12.0 & 12.0 & 12.9 &15.4 & 5.5 \\
		 & oui, dont bain et douche & 9.0 & 26.3 & 23.7 & 25.9 & 23.7 & 66.0 &53.6 & 3.1 \\\hline
		Problèmes de mobilité & aucun ou haut du corps & 52.1 & 30.2 & 27.3 & 30.2 & 27.3 & 14.8 & 12.0 & 61.8 \\
		 & oui, bas du corps & 31.2 & 38.3 & 40.7 & 38.5 & 40.8 & 36.6 &38.6 & 27.4 \\
		 & oui, haut et bas du corps & 16.7 & 31.5 & 32.0 & 31.3 & 31.9 & 48.6 & 49.4 & 10.8 \\\hline
		Maladies chroniques déclarées & aucune & 12.6 & 5.7 & 4.2 & 5.7 & 4.2 & 2.6 & 3.1 & 15.9 \\
		 & oui, légères & 46.1 & 38.1 & 38.3 & 38.4 & 38.4 & 33.8 & 31.0 & 49.0 \\
		 & oui, graves & 41.4 & 56.2 & 57.5 & 55.9 & 57.4 & 63.6 & 65.9 & 35.1 \\ \hline
		Aide à la marche & non ou autre & 90.4 & 68.0 & 76.0 & 68.2 & 76.1 & 41.8 &59.8 & 97.0 \\
		 & fauteuil roulant, déambulateur & 9.6 & 32.0 & 24.0 & 31.8 & 23.9 & 58.2 & 40.2 & 3.0 \\\hline
		Force de préhension & faible  & 11.1 & 21.0 & 18.8 & 21.1 & 18.5 & 31.2 &29.6 & 7.7 \\\hline
		Personnes de confiance & aucun & 5.0 & 4.4 & 3.1 & 4.2 & 3.1 & 6.1 &4.5 & 5.5 \\
		 & moins de 4 & 3.3 & 3.1 & 1.3 & 3.2 & 1.4 & 3.1 &1.4 & 3.8 \\
		 & 4 et plus & 91.7 & 92.5 & 95.5 & 92.6 & 95.6 & 90.1 & 94.1 & 90.8 \\\hline
		Santé mentale & mauvaise & 34.6 & 46.4 & 51.3 & 46.3 & 51.1 & 58.5 & 67.0 & 28.8 \\\hline
		Sentiment de solitude & jamais & 49.8 & 44.3 & 41.7 & 44.4 & 42.0 & 38.4 & 34.9 & 52.4 \\
		 & parfois & 33.3 & 32.2 & 35.0 & 31.9 & 34.7 & 31.9 & 33.5 & 33.1 \\
		 & souvent & 17.0 & 23.6 & 23.2 & 23.8 & 23.3 & 28.6 & 31.6 & 14.5 \\\hline
		GALI & pas limité & 41.1 & 20.6 & 19.1 & 20.7 & 19.1 & 8.9 & 8.1 & 50.0 \\
		 & limité mais pas fortement & 37.7 & 39 & 38.8 & 38.5 & 38.9 & 26.8 & 27.9 & 36.8 \\
		 & fortement limité & 21.2 & 40.4 & 42.1 & 40.8 & 42 & 64.3 & 64 & 13.2 \\\hline
		Pays de résidence & Allemagne & 5.6 & 6.3 & 5.5 & 6.4 & 5.5 & 7.5 &7.3 & 5.5 \\
		& Autriche & 4.7 & 6.2 & 5.7 & 6.4 & 5.6 & 4.2 &7.8 & 4.3 \\
		& Belgique & 5.0 & 11.7 & 5.4 & 11.6 & 5.5 & 12.0 & 3.1 & 3.6 \\
		& Bulgarie & 2.2 & 0.7 & 2.8 & 0.7 & 2.8 & 0.5 & 3.6 & 2.3 \\
		& Chypre & 1.2 & 2.6 & 1.1 & 2.4 & 0.9 & 4.7 &2.0 & 1.0 \\
		& Croatie & 1.6 & 1.1 & 2.5 & 1.1 & 2.6 & 0.7 & 1.4 & 1.4 \\
		& Danemark & 4.7 & 6.1 & 3.0 & 5.9 & 3.1 & 5.6 & 0.6 & 4.8 \\
		& Espagne & 3.6 & 6.1 & 3.1 & 6.1 & 3.0 & 6.6 &6.2 & 3.3 \\
		& Estonie & 9.6 & 2.2 & 11.7 & 2.2 & 11.8 & 2.1 & 11.7 & 10.4 \\
		& Finlande & 1.9 & 1.4 & 1.5 & 1.3 & 1.4 & 0.9 & 1.7 & 2.1 \\
		& France & 6.8 & 10.3 & 5.7 & 10.4 & 5.6 & 9.4 & 3.4 & 6.4 \\
		& Grèce & 5.7 & 4.3 & 5.3 & 4.4 & 5.3 & 4.2 & 7.5 & 6.1 \\
		& Hongrie & 2.2 & 1.5 & 2.6 & 1.5 & 2.6 & 2.4 & 3.4 & 2.2 \\
		& Israël & 1.8 & 3.9 & 1.7 & 4.0 & 1.6 & 6.1 & 1.7 & 1.4 \\
		& Italie & 2.8 & 3.7 & 2.9 & 3.8 & 2.8 & 2.8 & 3.6 & 2.5 \\
		& Lettonie & 2.0 & 0.6 & 1.3 & 0.6 & 1.4 & 0.9 & 1.7 & 2.4 \\
		& Lituanie & 3.6 & 1.4 & 3.8 & 1.4 & 3.9 & 2.1 & 6.1 & 4.0 \\
		& Luxembourg & 1.4 & 1.6 & 1.2 & 1.5 & 1.2 & 1.2 & 0.8 & 1.4 \\
		& Malte & 0.9 & 0.5 & 0.3 & 0.5 & 0.3 & 0  & 0.3 & 1.1 \\
		%& Pays-Bas & 4.3 & 9.7 & 3.0 & 9.8 & 2.9 & 6.8 &1.1 & 3.5 \\
		& Pologne & 2.6 & 0.8 & 2.8 & 0.7 & 2.9 & 1.4 & 2.5 & 2.9 \\
		& République Tchèque & 7.3 & 2.8 & 11.4 & 2.8 & 11.8 & 3.1 & 7.5 & 7.2 \\
		& Roumanie & 1.6 & 0.3 & 2.8 & 0.3 & 2.8 & 0.7 & 5.0 & 1.6 \\
		& Slovaquie & 1.0 & 0.4 & 0.8 & 0.4 & 0.8 & 0.7 &1.4 & 1.2 \\
		& Slovénie & 4.7 & 1.5 & 5.6 & 1.4 & 5.6 & 1.9 &7.0 & 5.0 \\
		& Suède & 6.7 & 6.3 & 3.3 & 6.4 & 3.3 & 5.6 & 0.8 & 7.6 \\
		& Suisse & 4.7 & 6.2 & 3.4 & 6.0 & 3.4 & 5.9 & 0.8 & 4.8 \\\hline
		Nombre d'observations && 8 799 & 1 561 & 1 638 & 1 516 & 1 596 & 426 & 358 & 6 100 \\\hline \hline
%			GALI estimé & pas limité & 47.6 & 22.2 & 20.8 & 22.4 & 20.9 & 4.5 & 6.2 & 58.2 \\
%			 & limité mais pas fortement & 37.8 & 43.5 & 43.7 & 43.5 & 43.7 & 31.5 & 30.5 & 35.0 \\
%			 & fortement limité & 14.6 & 34.3 & 35.5 & 34.1 & 35.3 & 64.1 & 63.4 & 6.8 \\\hline
%			$\Delta$GALI & correcte & 60.0 & 57.4 & 58.6 & 57.4 & 58.7 & 67.1 & 66.2 & 61.3 \\
%			 & sous évaluée & 14.5 & 18.1 & 17.0 & 17.8 & 16.9 & 18.1 & 17.0 & 13.3 \\
%			 & sur évaluée & 25.5 & 24.5 & 24.4 & 24.9 & 24.5 & 14.8 & 16.8 & 25.5 \\\hline\hline
		\end{tabular}
	}
	\label{quali_variablesA}
\end{table}

%%%%%%%%%%%%%%%%%%%%%%%%%%%%%%%%%%%%%%%%%%%%%%%%%%%%%



%%%%%%%%%%%%%%%%%%%%%%%%%%%%%%%%%%%%%%%%%%%%%%%%%%%%%
\end{appendices}
%%%%%%%%%%%%%%%%%%%%%%%%%%%%%%%%%%%%%%%%%%%%%%%%%%%%%

%%%%%%%%%%%%%%%%%%%%%%%%%%%%%%%%%%%%%%%%%%%%%%%%
\begin{table}
	\centering
	\caption{Résultats des estimations : modèle avec l'aide domestique formelle 
	et l'aide domestique informelle}
	\scalebox{0.5}{
		\sisetup{
			% parse-numbers = true,
			output-decimal-marker={,},
			%  table-number-alignment = right,
			% table-text-alignment = right,
			table-alignment  = right,
			negative-color = red,
			input-open-uncertainty = ,
			input-close-uncertainty = ,
			minimum-decimal-digits = 0,
			table-format = (+1.3, %)
			table-align-text-before = false,
			table-align-text-after = false,
			round-mode = places,
			round-precision=3,
			round-pad = false,
		}
	\hspace{-2cm}
\begin{tabular}{l@{\;}l@{\;} S!{\qquad}S!{\qquad}S!{\qquad}}
	\toprule\midrule
		  &  &\multicolumn{1}{c}{Aide formelle }& \multicolumn{1}{c}{Heures aide formelle } & \multicolumn{1}{c}{Aide informelle } \\ 
		  &  & \multicolumn{1}{c}{domestique}& \multicolumn{1}{c}{domestique} & \multicolumn{1}{c}{domestique} \\ \hline

	Sexe  & homme (réf.)  &\sym{}&  \sym{}&\sym{} \\
	& femme & -0.19534\sym{***} & -0.08945 & 0.15392 \\\hline
	\^Age& & 0.07918\sym{***} & 0.03039\sym{***} & 0.082\sym{***} \\\hline
	Nombre d'enfants  & & -0.04071\sym{**} & 0.00875 & 0.09587\sym{***} \\\hline
	Revenu (en logarithme) & & 0.03045 & 0.00374 & -0.04292 \\\hline
	Niveau d'éducation& sans diplôme (réf.) &\sym{}&  \sym{}&\sym{} \\
	& inférieur au bac & 0.15759 & 0.006 & -0.12843 \\
	& niveau bac & 0.27509\sym{**} & -0.00979 & -0.28418 \\
	& études supérieures & 0.47501\sym{***} & 0.11952 & -0.44795\sym{**} \\\hline
	Zone résidentielle& très grande ville (réf.)  &\sym{}&  \sym{}&\sym{} \\
	& banlieue d'une très grande ville & 0.01271 & -0.01783 & -0.00101 \\
	& grande ville & -0.02093 & -0.00596 & 0.091 \\
	& petite ville & -0.01057 & 0.03914 & 0.2516\sym{*} \\
	& zone rurale & -0.11547 & 0.13604 & 0.42691\sym{***} \\
	& non renseignée & 0.01677 & 0.19502 & 0.43908\sym{**} \\\hline
	ADL déclarées& aucune (réf.) &\sym{}&  \sym{}&\sym{} \\
	& oui, mais pas bain ou douche & 0.39977\sym{***} & 0.16991 & 0.45954\sym{***} \\
	& oui, dont bain ou douche & 0.92118\sym{***} & 0.56693\sym{***} & 0.92495\sym{***} \\\hline
	Problèmes de mobilité& aucun ou haut du corps (réf.)&\sym{}&  \sym{}&\sym{} \\
	& oui, bas du corps & 0.33464\sym{***} & -0.0863 & 0.38964\sym{***} \\
	& oui, haut et bas du corps & 0.43363\sym{***} & 0.00625 & 0.44721\sym{***} \\\hline
	Maladies chroniques déclarées & aucune (réf.) &\sym{}&  \sym{}&\sym{} \\
	& oui, autre & 0.24279\sym{**} & -0.11339 & 0.55026\sym{***} \\
	& oui, Parkinson/Alzheimer/arthrose & 0.37695\sym{***} & -0.06246 & 0.80063\sym{***} \\\hline
	Aide à la marche & non ou autre (réf.) &\sym{}&  \sym{}&\sym{} \\
	& fauteuil roulant, déambulateur & 1.1391\sym{***} & 0.55561\sym{***} & 1.14756\sym{***} \\\hline
	Pays de résidence & France (réf.) &\sym{}&  \sym{}&\sym{} \\
	& Autriche & -0.33664\sym{**} & 0.47442\sym{***} & 0.97337\sym{***} \\
	& Allemagne & -0.78981\sym{***} & 0.28303\sym{*} & 0.45565\sym{*} \\
	& Suède & -1.07368\sym{***} & -0.2248 & -0.72004\sym{**} \\
	& Espagne & -0.18483 & 0.59705\sym{***} & 0.01374 \\
	& Italie & -0.33089\sym{**} & 0.67596\sym{***} & 0.41698 \\
	& Danemark & -0.497\sym{***} & -0.67401\sym{***} & -0.02121 \\
	& Grèce & -0.92732\sym{***} & 1.03406\sym{***} & 0.50987\sym{**} \\
	& Suisse & -0.34955\sym{***} & 0.04267 & 0.18934 \\
	& Belgique & 0.32337\sym{***} & 0.33568\sym{***} & 0.70565\sym{***} \\
	& Israël & 0.0216 & 1.50449\sym{***} & 0.72396\sym{*} \\
	& République tchèque & -1.19633\sym{***} & 0.28757 & 1.57357\sym{***} \\
	& Luxembourg & -0.4557\sym{**} & 0.29501 & 0.67699\sym{*} \\
	& Slovénie & -1.8013\sym{***} & 0.48937\sym{**} & 0.76773\sym{***} \\
	& Estonie & -2.07516\sym{***} & -0.16732 & 0.77558\sym{***} \\
	& Pologne & -1.86504\sym{***} & 0.42469 & 0.5057\sym{*} \\
	& Hongrie & -0.91488\sym{***} & 0.28919 & 0.77829\sym{**} \\
	& Croatie & -0.59148\sym{**} & 0.53443\sym{*} & 1.54926\sym{***} \\
	& Lituanie & -1.73424\sym{***} & 0.18255 & 0.62498\sym{**} \\
	& Bulgarie & -1.40995\sym{***} & 0.92361\sym{**} & 0.95199\sym{***} \\
	& Chypre & -0.02361 & 1.74729\sym{***} & 0.15848 \\
	& Finlande & -0.92708\sym{***} & -0.22612 & 0.09666 \\
	& Lettonie & -1.88691\sym{***} & 0.33888 & 0.15072 \\
	& Malte & -0.83635\sym{**} & 0.35583 & -0.2148 \\
	& Roumanie & -2.16317\sym{***} & -0.09397 & 1.17355\sym{***} \\
	& Slovaquie & -1.67452\sym{***} & 0.02536 & 0.19546 \\\hline
	Force de préhension faible & non (réf.) &\sym{}&  \sym{}& \sym{} \\
	& oui & 0.11812 & 0.21204\sym{***} & 0.10676 \\\hline
	Personnes de confiance  & aucune (réf.) &\sym{}&  \sym{}&  \sym{} \\
	& moins de 4 & -0.13986 & 0.16446 & -0.48812 \\
	& 4 et plus & 0.24138\sym{*} & 0.01812 & 0.93045\sym{***} \\\hline
	Santé mentale mauvaise  & non (réf.) &\sym{}& \sym{}&\sym{} \\
	& oui & 0.2015\sym{***} & 0.11178 & 0.32746\sym{***} \\\hline
	Sentiment de solitude& jamais (réf.) &\sym{}& \sym{}& \sym{} \\
	& parfois & -0.0835 & -0.09515 & -0.01912 \\
	& souvent & 0.1698\sym{**} & -0.0963 & 0.01246 \\\hline
	GALI & pas limité (réf.) & \sym{} & \sym{} & \sym{} \\
	& limité mais pas fortement & 0.43926\sym{***} & 0.17575\sym{**} & 0.51205\sym{***} \\
	& fortement limité & 0.87185\sym{***} & 0.42683\sym{***} & 1.05983\sym{***} \\\hline
	Constante & & -5.04584\sym{***} & 0.15208 & -4.35047\sym{***} \\\hline\hline
	Heures d'aide formelle domestique & & & &-0.06441\sym{***} \\
	Aide informelle domestique & & -1.63774\sym{***} & -1.03541\sym{***} & \\\hline\hline
	Paramètres  & $\rho$ & -2.53111\sym{***} & -1.86166\sym{***} & -5.91528\sym{***} \\
		& $\delta$ & -1.788\sym{***} & -0.69603\sym{***} & -3.17707\sym{***} \\\cline{2-5}
		& $\theta_u$ & & 2.45323\sym{***} & \\
		& $\theta_v$ & & 0.02561 & \\\hline
	Nombre d'observations (N) & 8799 &  & &\\\hline
	Log vraisemblance & -10103,22 & & & \\\hline\hline
	\multicolumn{5}{l}{\footnotesize Significativité  ; \sym{*} \(p<0,10\), \sym{**} \(p<0,05\), \sym{***} \(p<0,01\)}	
\end{tabular}
	}
\label{res_fdcidc}
\end{table}

%%%%%%%%%%%%%%%%%%%%%%%%%%%%%%%%%%%%%%%%%%%%%%

\begin{table}
	\centering
	\caption{Résultats des estimations : modèle avec l'aide personnelle formelle 
	et l'aide personnelle informelle}
	\scalebox{0.5}{
		\sisetup{
			% parse-numbers = true,
			output-decimal-marker={,},
			%  table-number-alignment = right,
			% table-text-alignment = right,
			table-alignment  = right,
			negative-color = red,
			input-open-uncertainty = ,
			input-close-uncertainty = ,
			minimum-decimal-digits = 0,
			table-format = (+1.3, %)
			table-align-text-before = false,
			table-align-text-after = false,
			round-mode = places,
			round-precision=3,
			round-pad = false,
		}
\begin{tabular}{l@{\;}l@{\;} S!{\qquad}S!{\qquad}S!{\qquad}}
	\toprule\midrule
		  &  &\multicolumn{1}{c}{Aide formelle}&\multicolumn{1}{c}{Heures d'aide formelle} & \multicolumn{1}{c}{Aide informelle} \\
		  &  & \multicolumn{1}{c}{personnelle}& \multicolumn{1}{c}{personnelle} & \multicolumn{1}{c}{personnelle} \\ \hline
	Sexe & homme (réf.)  &\sym{}&  \sym{}&\sym{} \\
	& femme & -0.24521\sym{**} & -0.10613 & -0.33656\sym{*} \\\hline
	\^Age& & 0.04599\sym{***} & 0.00905 & 0.05893\sym{***} \\\hline
	Nombre d'enfants & & -0.01262 & 0.06651 & 0.08002 \\\hline
	Revenu (en logarithme) & & 0.00928 & -0.00004 & 0.0396 \\\hline
	Niveau d'éducation& sans diplôme (réf.) &\sym{}&  \sym{}&\sym{} \\
	& inférieur au bac & -0.28961\sym{*} & 0.28348 & -0.47444\sym{*} \\
	& niveau bac & -0.302 & 0.3033 & -0.74226\sym{**} \\
	& études supérieures & -0.38765\sym{*} & 0.46333 & -0.94932\sym{***} \\\hline
	Zone résidentielle& Très grande ville (réf.)  &\sym{}&  \sym{}&\sym{} \\
	& banlieue d'une très grande ville & 0.15968 & 0.08362 & 0.56768 \\
	& grande ville & 0.05628 & 0.51428\sym{**} & 0.6378\sym{**} \\
	& petite ville & -0.08675 & 0.21237 & 0.15726 \\
	& zone rurale & -0.1795 & 0.24139 & 0.40841\sym{*} \\
	& non renseignée & 0.06878 & 0.37694 & 0.1665 \\\hline
	ADL déclarées& aucune (réf.) &\sym{}&  \sym{}&\sym{} \\
	& oui, mais pas bain ou douche & 1.34094\sym{***} & 1.28091\sym{***} & 1.91374\sym{***} \\
	& oui, dont bain ou douche & 2.41344\sym{***} & 1.79407\sym{***} & 3.84145\sym{***} \\\hline
	Problèmes de mobilité& aucun ou haut du corps (réf.)&\sym{}&  \sym{}&\sym{} \\
	& oui, bas du corps & 0.02651 & 0.19009 & 0.41435\sym{**} \\
	& oui, haut et bas du corps & 0.2126 & 0.30552 & 0.61765\sym{***} \\\hline
	Maladies chroniques déclarées & aucune (réf.) &\sym{}&  \sym{}&\sym{} \\
	& oui, autre & 0.20649 & 0.5723 & 0.07225 \\
	& oui, Parkinson/Alzheimer/arthrose & 0.35555 & 0.47839 & 0.46545 \\\hline
	Aide à la marche & non ou autre (réf.) &\sym{}&  \sym{}&\sym{} \\
	& fauteuil roulant, déambulateur & 0.95963\sym{***} & 0.76005\sym{***} & 1.1231\sym{***} \\\hline
	Pays de résidence & France (réf.) &\sym{}&  \sym{}&\sym{} \\
	& Autriche & -0.3062 & 0.26994 & 2.01068\sym{***} \\
	& Allemagne & -0.32103 & 0.01609 & 1.6023\sym{***} \\
	& Espagne & 0.20129 & 0.98179\sym{***} & 1.8627\sym{***} \\
	& Italie & -0.27112 & 1.30261\sym{***} & 2.19091\sym{***} \\
	& Grèce & -0.48451\sym{*} & 1.70712\sym{***} & 2.0954\sym{***} \\
	& Belgique & 0.38004\sym{**} & -0.2463 & 0.30332 \\
	& Israël & 0.48245\sym{*} & 1.30324\sym{***} & 1.69657\sym{**} \\
	& République tchèque & -0.54242\sym{**} & 0.75368\sym{**} & 2.26506\sym{***} \\
	& Luxembourg & -0.18291 & 0.15805 & 0.37424 \\
	& Slovénie & -0.99617\sym{***} & 0.30406 & 2.26156\sym{***} \\
	& Estonie & -1.43115\sym{***} & -0.32956 & 1.97566\sym{***} \\
	& Pologne & -0.985\sym{***} & 1.03911\sym{*} & 0.94002 \\
	& Hongrie & 0.0485 & 0.76924\sym{*} & 2.36066\sym{***} \\
	& Croatie & -0.6419 & 1.01335 & 1.04311 \\
	& Lituanie & -0.83846\sym{***} & -0.06052 & 2.2323\sym{***} \\
	& Bulgarie & -1.31409\sym{***} & 1.22436 & 2.03236\sym{***} \\
	& Chypre & 0.40782 & 2.29884\sym{***} & 1.81299\sym{***} \\
	& Finlande & -0.13775 & 1.24494\sym{**} & 2.4773\sym{***} \\
	& Lettonie & -0.93865\sym{**} & 0.29673 & 1.03339 \\
	& Roumanie & -0.91103\sym{**} & -0.13813 & 2.81787\sym{***} \\
	& Slovaquie & -0.35725 & 1.32174\sym{*} & 2.09188\sym{***} \\\hline
	Force de préhension faible & non (réf.) &\sym{}& \sym{}& \sym{} \\
	& oui & 0.18994\sym{*} & 0.08248 & 0.18607 \\\hline
	Personnes de confiance  & aucune (réf.) &\sym{}& \sym{}& \sym{} \\
	& moins de 4 & -0.81261\sym{**} & -0.2412 & -0.51302 \\
	& 4 et plus & -0.20226 & -0.46381 & 0.60485\sym{*} \\\hline
	Santé mentale mauvaise & non (réf.) & \sym{} & \sym{}& \sym{} \\
	& oui & 0.06456 & -0.00267 & 0.45995\sym{***} \\\hline
	Sentiment de solitude& jamais (réf.) &\sym{}& \sym{}&  \sym{}  \\
	& parfois & -0.04135 & -0.51326\sym{***} & 0.03849 \\
	& souvent & 0.07666 & -0.44497\sym{**} & -0.00546 \\\hline
	GALI & pas limité (réf.) & \sym{} & \sym{} & \sym{} \\
	& limité mais pas fortement & 0.64879\sym{***} & 0.36306 & 0.42814\sym{*} \\
	& fortement limité & 0.99349\sym{***} & 0.41381 & 1.04743\sym{***} \\\hline
	Constante & & -0.90573 & 2.63593\sym{**} & 9.96243 \\\hline\hline
	Heures d'aide formelle personnelle & & & & -0.10524\sym{***} \\
	Aide informelle personnelle & & -1.36916\sym{***} & -1.57563\sym{***} & \\\hline\hline
	Paramètres & $\rho$ & -4.4018\sym{***} & -3.46626\sym{***} & -19.8039\sym{***} \\
		& $\delta$ & -2.18242\sym{***} & -1.79454\sym{***} & -5.10002\sym{***} \\\cline{2-5}
		& $\theta_u$ & & 5.73085\sym{***} & \\
		& $\theta_v$ & & 1.54131\sym{***} & \\\hline
	Nombre d'observations & 7304 & & & \\\hline
	Log vraisemblance & -2752,11 & & & \\\hline\hline
	\multicolumn{5}{l}{\footnotesize Significativité ; \sym{*} \(p<0,10\), \sym{**} \(p<0,05\), \sym{***} \(p<0,01\). Pour des problèmes de convergence liés à des effectifs trop faibles (taux d'aide personnelle formelle } \\
	\multicolumn{5}{l}{ou informelle- inférieur à 1\%), la Suède, le Danemark, la Suisse et Malte ont été écartés.}
\end{tabular}
	}
\label{res_fpcipc}
\end{table}

\begin{figure}[h!] % Fichiers manquants
		% \includegraphics[scale = 0.6]{plot1.pdf}
		% \includegraphics[scale = 0.6]{plot2.pdf}
\end{figure}
\end{refsection}

\end{Article}